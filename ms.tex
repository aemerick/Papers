
\documentclass[twocolumn]{aastex62}

\newcommand{\vdag}{(v)^\dagger}
\newcommand\aastex{AAS\TeX}
\newcommand\latex{La\TeX}

\graphicspath{{./}{figures/}}
\usepackage{natbib}
\bibliographystyle{apj}

\submitjournal{ApJ}

\shorttitle{Radiation Feedback in Dwarf Galaxies}
\shortauthors{Emerick et. al.}


\begin{document}

\title{Stellar Radiation is Critical for Regulating Star Formation and Driving Outflows in Low Mass Dwarf Galaxies}

\correspondingauthor{Andrew Emerick}
\email{emerick@astro.columbia.edu}

\author{Andrew Emerick}
\affil{Columbia University}
\affil{American Museum of Natural History (AMNH)}

\author{Greg L. Bryan}
\affiliation{Columbia University}
\affiliation{Flatiron Institute}

\author{Mordecai-Mark Mac Low}
\affiliation{American Museum of Natural History (AMNH)}
\nocollaboration

\begin{abstract}
\textit{Will complete at end}
%{\bf Ignore: Ionizing radiation from recently formed massive stars is important in destroying cold, dense gas around new regions of star formation that would otherwise continue to form stars. Applying a local prescription for this pre-supernova feedback can be sufficient in matching star formation rates in a simulation with full radiative transfer on short timescales. However, the long-range effects of ionizing radiation are important in producing a volume-filling diffuse, warm-ionized medium in the ISM and in the disk-halo interface. This is essential for allowing supernovae to drive significant gas outflows that travel throughout and beyond the virial radius of the galaxy. A simulation with local radiation feedback only is unable to drive significant outflows as the supernovae are trapped well within the inner halo by higher density cold and warm neutral gas.}
\end{abstract}

%% Keywords should appear after the \end{abstract} command. 
%% See the online documentation for the full list of available subject
%% keywords and the rules for their use.
\keywords{galaxies -- feedback -- galaxy evolution -- galactic winds}

\section{Introduction} \label{sec:intro}
Historically, simulations of galaxy formation have suffered from the ``overcooling'' problem, whereby the action of self-gravity and radiative cooling alone produces galaxies with far too many stars. This problem has been addressed by employing various models of stellar feedback physics which are capable of generating self-regulating star formation in galaxies. Recently, it has become clear that the simplest models for stellar feedback, which account for supernovae (SNe) alone, are insufficient \citep{Smith2018} to fully regulate star formation and set galaxy properties consistent with observations (see \cite{NaabOstriker2017} for a recent review). Processes, such as stellar radiation, cosmic rays, or AGN, are often employed as additional sources of feedback. Modeling these processes in detail, however, is challenging. As a result, the relative importance of these additional physics is still uncertain.

Radiation from massive stars dominates the total feedback energy output of a stellar population 
%mm added Abbott1982
\citep[e.g.][]{Abbott1982,Leitherer1999,Agertz2013}, surpassing the energy ejection of supernova ($\sim 10^{51}$~erg) by two orders of magnitude. If the radiation couples effectively to the interstellar medium (ISM), it can be a substantial source of additional feedback. %Lower mass stars (size range) primarily drive photoelectric heating (cite) and H$_2$ dissociation through radiation in the far ultraviolet and Lyman-Werner bands, while more massive stars (range) emit significant amounts of ionizing radiation. 
Simulations including stellar radiation feedback followed through radiative transfer or radiation-hydrodynamics schemes have found it to be effective in regulating star formation and driving galactic winds \citep[e.g.][]{WiseAbel2012,Kim2013a,Sales2014,Oshea2015,Rosdahl2015,Ocvirk2015,Pawlik2015,Peters2017}. This occurs in four ways: 1) heating 
%mm 
   of diffuse 
gas and preventing the formation of cold, dense star formation regions, 2) destruction of cold, dense gas around recently formed stars, preventing further star formation, 3) momentum input by direct absorption of UV radiation by gas and (in some cases) dust through re-emission and scattering in the infrared, and 4) lowering the typical ISM densities in which SNe occur and greatly increasing their effectiveness. % the relevance / importance of RP from dust in the IR is beyond the scope of this work... 

However, most works that employ stellar radiation feedback to account for these effects do so using various forms of sub-grid, approximate models to avoid the substantial additional cost of full radiative transfer. Many works use a Str{\"o}mgren approximation whereby the particles / cells within the Str{\"o}mgren radius around a radiating star are heated and kept ionized, with additional approximations made to account for overlapping ionized regions \citep[e.g.][]{HQM2011,Hu2016,Hu2017}. Other works employ some form of energy or momentum injection localized to the region immediately around a star particle \citep[e.g.][(\textit{need more})]{Agertz2013,Roskar2014,Ceverino2014,Forbes2016}. Although some of these approximate methods account for long-range effects of diffuse radiation \citep{HQM2012,Hopkins2018} most cases treat local radiation {\it only}, confined to energy or momentum injection in a limited physical region around a star particle. Although the approximate radiation feedback models are capable of driving significant galactic outflows, it is unclear if they capture the full effects of stellar ionizing radiation. 
%The 
%It is unclear what, if any, effect the long-range ionization and heating processes play in regulating star formation and driving galactic winds. 

In this work we present a comparison of three simulations of the evolution of an isolated, low mass dwarf galaxy that account for detailed stellar feedback. Our fiducial model uses an adaptive ray-tracing radiative transfer method to follow stellar ionizing radiation. We test its effectiveness in regulating star formation and driving galactic winds by comparison to a simulation without ionizing radiation and a simulation with only localized radiation around each star. We discuss our methods in Section~\ref{sec:methods} and present our results in Section~\ref{sec:results}.

\section{Methods and Initial Conditions} \label{sec:methods}
We refer the reader to Paper I for a more detailed description of our methods, briefly summarized here. We use the adaptive mesh refinement hydrodynamics code \textsc{Enzo} \citep{Enzo2014} to evolve an idealized, isolated low mass dwarf galaxy. The galaxy is initialized as a smooth exponential disk set in hydrostatic equillibrium with a static dark matter potential \citep{Burkert1995} with $M_{\rm gas} = 2 \times 10^6~$~M$_{\odot}$, radial and vertical gas scale heights of 250~pc and 100~pc respectively, $M_{\rm vir} = 2.48 \times 10^9$~M$_{\odot}$, on a grid with a maximum physical resolution of 1.8~pc. We do not model an initial stellar population, but do include random driving from supernovae as an initial source of feedback up until the first explicit star formation in our model.

We include a UV background, tabulated metal line cooling, and a 9 species non-equilibrium chemistry solver using \textsc{Grackle} \citep{GrackleMethod}. Star formation occurs stochastically in cold, dense regions ($n > 200$~cm$^{-3}$, $T < 200$~K) by sampling the \cite{Salpeter1955} IMF from 1~M$_{\odot}$ to 100~M$_{\odot}$ and depositing \textit{individual} star particles over this mass range. For each star, we include feedback from stellar winds, AGB winds, FUV and LW band radiation which drives photoelectric heating and H$_2$ dissociation respectively, HI and HeI ionizing radiation, and core collapse and Type Ia SNe using thermal energy injection. FUV and LW band radiation are both taken to be optically thin, with local (cell-by-cell) attenuation alone. Ionizing radiation is followed using radiative transfer, as discussed below. Stellar lifetimes, surface gravity, effective temperature, and radii are set by the initial stellar mass and metallicity through interpolation over the PARSEC \citep{Bressan2012} zero age main sequence values. These values for surface gravity, effective temperature, and radii are used to set the FUV, LW, and ionizing photon fluxes from each star through interpolation on the OSTAR2002 \citep{LanzHubeny2003} grid. We only follow the radiation from stars with masses above 8~M$_{\odot}$. 
%tellar wind and supernovae mass loss are determined with the  NuGrid yield tables for $M_* < 25$~M$_{\odot}$ and from Slemer et. al. (in prep) for $M_* > 25$~M$_{\odot}$. 


%\textbf{Stellar Ionizing Radiation Models} $-$
In our fiducial simulation, we follow photoionizing radiation using the adaptive ray-tracing radiative transfer method of \cite{WiseAbel2011}. This method places and evolves 48 rays on a HEALPix grid around each emitting source star for both HI and HeI ionizing radiation. Rays are adaptively split as they propagate away from their source to increase the angular resolution such that the solid angle of the ray remains smaller than 1/3 of the cell area. Rays begin on HEALPix refinement level 2, with a maximum refinement level of 13 (or $8.05 \times 10^8$ rays).

We additionally present a simulation run without any stellar ionizing radiation (``noRT'') and a second simulation that includes ionizing radiation, but deletes all photons that travel more than 20~pc from their source (``shortrad''). This second simulation tests the relative importance of short-range vs. long-range effects of stellar ionizing radiation as a form of feedback. Although this is still more accurate than an approximate method, this is meant to function similarly to methods that include only the local effects of stellar radiation feedback. Each of our three simulations is identical up until the formation of the first star particles. At this point, the differing feedback physics causes each to diverge. We note that the stars from the very first star formation event in each run (21 star particles with a total mass of 114 M$_{\odot}$) are the same in each run. Of these, one particle emits ionizing radiation ($M_* > 8 $~M$_{\odot}$). 

\section{Results} \label{sec:results}

\begin{figure*}
\centering
\includegraphics[width=0.49\linewidth]{sfr}
\includegraphics[width=0.49\linewidth]{mass}
\caption{The star formation rate (left) and gas and stellar mass (right) of each of our three simulations over time.
%mm
   Time bins with no star formation are left empty. 
   {\bf legend on first plot has Shortard rather than Shortrad.}}
\label{fig:sfr_mass_evolution}
\end{figure*}

We compare the resulting star formation rate (left) and gas mass properties (right) of our three simulations in Figure~\ref{fig:sfr_mass_evolution}. There is a clear, significant contrast between the runs with and without ionizing radiation. Stellar ionizing radiation leads to a factor of $\sim$ 5 reduction in the resulting SFR, as compared to the noRT run. Since the shortrad and fiducial simulations are so similar over the first $\sim$~100~Myr, it is clear that stellar radiation acts to significantly reduce the local star formation efficiency around young, massive stars. Radiation from these stars quickly ionizes and dissipates surrounding dense gas that would otherwise have gone to fuel a significant amount of additional star formation. We are unable to follow the long-term evolution of the noRT galaxy due to computational constraints 
%mm
   from the large number of stars formed. 
Although radiative transfer is computationally expensive, this run is substantially more expensive than the fiducial and shortrad runs due to the significantly higher star formation rate.

Looking again at the first $\sim$100~Myr of simulation time, the effects of ionizing radiation beyond our 20~pc cutoff radius are not significant. However, these two simulations begin to diverge significantly after this point. The shortrad simulation has continual, steady star formation for the entire simulation time, while the star formation rate in our fiducial run is bursty, with periods of active, low-level star formation interspersed with periods (sometimes up to 50~Myr) of no star formation. The driver of these differences is clear in the right hand panel. Our fiducial run 
%mm looses a significant amount of 
    loses a factor of $\sim 2$ more 
total gas and HI mass (black and blue lines) as compared to the shortrad simulation. 
%mm by a factor of $\sim$2. 
Clearly, galactic winds and outflows are much more effective in the fiducial case, with full accounting of stellar ionizing radiation feedback, than in the shortrad simulation.

This can be confirmed by examining the metal retention fraction (the fraction of produced metals retained within the disk of the galaxy) and mass outflow rates in each simulation. As shown in Figure~\ref{fig:metal_retention}, the mass outflow rate in the fiducial run peaks at an order of magnitude higher at 0.25~R$_{\rm vir}$ than the shortrad simulation, declining only due to a comparative drop off in star formation. Given this, the fiducial simulation has an even higher mass loading factor, since the shortrad simulation has a typically higher SFR. The fiducial run is also the only 
%mm simulation 
   one of the three 
with any significant outflow beyond the virial radius, though we note the noRT run has not been simulated long enough for gas to have reached the virial radius. The outflow in noRT is nearly the same as the fiducial run, but it requires a five times higher supernova rate in this simulation to match the same outflow seen with full stellar radiation feedback. We conclude that radiation feedback allows SN to be substantially more effective in driving outflows. We provide a physical
%mm intuition 
   explanation
for these differences in the following section.

\begin{figure}
\centering
\includegraphics[width=0.95\linewidth]{metal_retention}\\
\includegraphics[width=0.95\linewidth]{mass_loading_comparison}
\caption{{\bf Top:} The fraction of metals contained within the ISM of each galaxy (blue) and the fraction ejected beyond the virial radius (orange). As in Figure~\ref{fig:sfr_mass_evolution}, the fiducial run is given as solid lines, while the noRT and shortrad are dash-dotted and dashed respectively. {\bf Bottom:} The mass outflow rate for each run at 0.25~R$_{\rm vir}$ (solid) and R$_{\rm vir}$ (dashed).
%mm
   {\bf Could you add a panel explicitly showing mass loading factor as discussed in the text?}
}
\label{fig:metal_retention}
\end{figure}

\textbf{We may have room for one more figure and a little more text to explain it. Possibly a comparison of mass / volume fractions in certain phases? But maybe not necessary.... 

What about vertical profiles of density, temperature, ionization?  These would support your discussion below - MM
}

\section{Discussion} \label{sec:discussion}
%mm Again, it is known that 
Accounting for feedback from stellar radiation plays a significant role in determining the ability for SN energy to couple to the ISM and therefore drive outflows. 
%mm However, the specific 
     We believe this work is novel in examining the
importance of localized ionization vs.\ ionization from a diffuse radiation field far from a single star
%mm has not been examined. As shown above, accounting for 
    Only modeling 
stellar radiation feedback locally 
%mm alone 
is insufficient to describe the long-term evolution of an isolated dwarf galaxy. To explore the cause of this difference we 
%mm present two panel-plots in Figure~\ref{fig:panel1} and Figure~\ref{fig:panel2} which 
compare in Figures~\ref{fig:panel1} and~Figure~\ref{fig:panel2} 
the gas number density (left), temperature (middle), and hydrogen ionization fraction (right) in edge-on slices in each of our simulations at two different times.\footnote{See ({\it insert link here once posted}) for a movie of this comparison.} 

Figure~\ref{fig:panel1}, at 17~Myr, compares each simulation just after the first few SNe. Already there are significant differences between the runs. Gas outside the galaxy is warm ($\sim$~10$^{4}$~K) and ionized up to $\sim$500~pc above/below the plane of the disk in our fiducial run. This same gas is cold ($<10^4$~K) and neutral in both other runs. The contrast between the effect of ionizing radiation in the fiducial and shortrad runs is seen most clearly in the ionized region in-plane and to the right of center. This region contains massive stars that are capable of generating enough ionizing radiation to carve a channel out to the halo of our fiducial simulation; this does not occur in the shortrad case. Instead, the HII region is confined by surrounding cold, neutral gas. 

\begin{figure*}
\centering
\includegraphics[width=0.99\linewidth]{DD0136_fiducial_shortrad_nort}
\caption{\textbf{In the final version of this figure I will add labels for each row marking the simulation, and a line giving the scale of each figure.But for now: TOP: Fiducial. MIDDLE: noRT. BOTTOM: shortrad} Edge-on slices of each simulation showing gas number density (left), temperature (middle), and hydrogen ionization fraction (right) 17~Myr after the formation of the first star in each run.}
\label{fig:panel1}
\end{figure*}

Although the ISM properties within the HII region in each case are quite similar between the simulations, the SNe that eventually go off in this region are confined by the neutral gas in the shortrad simulation, but 
%mm are readily ejected
   readily escape through the lower density ionized gas
into the galaxy halo in our fiducial case. As these simulations evolve, the existence of these diffuse, ionized channels in the fiducial run easily allow continual and significant outflows from SNe,
%mm
   as shown in  Figure~\ref{fig:panel2}, which shows each simulation 40~Myr after the 
   formation of the first stars. 
In contrast, the same SNe in the other two simulations are well contained, surrounded by shells of denser, neutral gas. Though they make a significant local impact on the ISM, they are unable to drive significant mass loss from the galaxy. In the noRT case, an outflow does eventually develop, but it takes the factor of five increase in SFR, and corresponding increase in supernova rate, for SNe to finally break out from the neutral gas surrounding the galaxy. 
%mm These differences are shown more clearly in Figure~\ref{fig:panel2}, which shows each simulation 40~Myr after the formation of the first stars. 

\begin{figure*}
\centering
\includegraphics[width=0.99\linewidth]{DD0160_fiducial_shortrad_nort}
\caption{Same as Figure~\ref{fig:panel1}, but at 40~Myr.}
\label{fig:panel2}
\end{figure*}

\section{Conclusion}  \label{sec:conclusion}
In agreement with previous work we find that stellar radiation feedback is effective in regulating star formation and driving outflows in our simulations of an isolated, low mass, dwarf galaxy. Simulations run without ionizing radiation feedback have star formation rates a factor of five higher than our fiducial simulation. In spite of this difference, SNe in our fiducial run are capable of driving larger galactic outflows, aided significantly by the ionizing radiation feedback.

We 
%mm also 
demonstrate
%mm
   for the first time 
that simple prescriptions 
%mm which treat stellar radiation feedback locally do not fully reproduce the same
%evolution as our fiducial model. 
    of local stellar radiation feedback fail to reproduce the evolution of our fiducial 
    model.
Our simulation with radiation localized to 20~pc around each star particle does effectively regulate star formation on short time scales, predominately by quickly destroying cold, dense gas around young, hot stars. However, this model does not drive the significant outflows seen in our fiducial simulation. Long-range ionizing radiation is important for carving channels allowing the release of significant amounts mass and metals from SNe. Our simulation with localized radiation feedback retains metals at a significantly higher fraction than expected observationally for low mass dwarf galaxies.

Finally, we note that we have performed this experiment on only one possible type of galaxy. Its low virial temperature ($\sim10^{4}$~K) makes this galaxy particularly sensitive to the effects of stellar feedback and ionizing radiation in particular. Examining the role of long-range, diffuse stellar ionizing radiation on star formation and galactic winds in more massive galaxies is an important avenue of future research. 

\bibliography{msbib} % if your bibtex file is called example.bib

%% This command is needed to show the entire author+affilation list when
%% the collaboration and author truncation commands are used.  It has to
%% go at the end of the manuscript.
%\allauthors

%% Include this line if you are using the \added, \replaced, \deleted
%% commands to see a summary list of all changes at the end of the article.
%\listofchanges

\end{document}

% End of file `sample62.tex'.
