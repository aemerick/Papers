
\documentclass[twocolumn]{aastex62}

\newcommand{\vdag}{(v)^\dagger}
\newcommand\aastex{AAS\TeX}
\newcommand\latex{La\TeX}

\graphicspath{{./}{figures/}}
\usepackage{natbib}
\bibliographystyle{apj}

\submitjournal{ApJ}

\shorttitle{Radiation Feedback in Dwarf Galaxies}
\shortauthors{Emerick et. al.}


\begin{document}

\title{Stellar Radiation is Critical for Regulating Star Formation and Driving Outflows in Low Mass Dwarf Galaxies}

\correspondingauthor{Andrew Emerick}
\email{emerick@astro.columbia.edu}

\author{Andrew Emerick}
\affil{Columbia University}
\affil{American Museum of Natural History (AMNH)}

\author{Greg L. Bryan}
\affiliation{Columbia University}
\affiliation{Flatiron Institute}

\author{Mordecai-Mark Mac Low}
\affiliation{American Museum of Natural History (AMNH)}
\nocollaboration

\begin{abstract}
%{\bf Ignore: Ionizing radiation from recently formed massive stars is important in destroying cold, dense gas around new regions of star formation that would otherwise continue to form stars. Applying a local prescription for this pre-supernova feedback can be sufficient in matching star formation rates in a simulation with full radiative transfer on short timescales. However, the long-range effects of ionizing radiation are important in producing a volume-filling diffuse, warm-ionized medium in the ISM and in the disk-halo interface. This is essential for allowing supernovae to drive significant gas outflows that travel throughout and beyond the virial radius of the galaxy. A simulation with local radiation feedback only is unable to drive significant outflows as the supernovae are trapped well within the inner halo by higher density cold and warm neutral gas.}
\end{abstract}

%% Keywords should appear after the \end{abstract} command. 
%% See the online documentation for the full list of available subject
%% keywords and the rules for their use.
\keywords{galaxies -- feedback -- galaxy evolution -- galactic winds}

\section{Introduction} \label{sec:intro}
Historically, simulations of galaxy formation have suffered from the ``overcooling'' problem, whereby simulations including self-gravity and radiative cooling alone produce galaxies with far too many stars. This problem has been addressed by employing various models of feedback physics which produces self-regulating star formation in galaxies. Recently, it has become clear that simple models for stellar feedback, which account for supernovae (SNe) alone, are insufficient to fully regulate star formation and set galaxy properties consistent with observations (see \cite{NaabOstriker2017} for a recent review). Additional processes, such as stellar radiation, cosmic rays, or AGN, are often employed as additional sources of feedback. Modeling these processes in detail, however, is challenging.

Radiation from massive stars dominates the total feedback energy output of a stellar population \citep[e.g.][]{Leitherer1999,Agertz2013}, surpassing the energy ejection of supernova ($\sim 10^{51}$~erg) by two orders of magnitude. If the radiation couples effectively to the interstellar medium (ISM), it can be a substantial source of additional feedback. %Lower mass stars (size range) primarily drive photoelectric heating (cite) and H$_2$ dissociation through radiation in the far ultraviolet and Lyman-Werner bands, while more massive stars (range) emit significant amounts of ionizing radiation. 
Simulations including stellar radiation feedback followed through radiative transfer or radiation-hydrodynamics schemes have found it to be effective in regulating star formation and driving galactic winds \citep[e.g.][]{WiseAbel2012,Kim2013a,Sales2014,Oshea2015,Rosdahl2015,Ocvirk2015,Pawlik2015,Peters2017}. This occurs for four reasons: 1) heating gas and preventing the formation of cold, dense star formation regions, 2) the destruction of cold, dense gas around recently formed stars, preventing further star formation, 3) momentum input by direct absorption of UV radiation by gas and (in some cases) dust through re-emission and scattering in the infrared, and 4) lowering the typical ISM densities in which SNe occur and greatly increasing their effectiveness. % the relevance / importance of RP from dust in the IR is beyond the scope of this work... 

However, most works that employ stellar radiation feedback to account for these effects do so using various forms of sub-grid, approximate models to avoid the substantial additional cost of full radiative transfer. Many works use a Stromgren approximation whereby the particles / cells within the Stromgren radius a radiating star are heated and kept ionized, with additional approximations made to account for overlapping ionized regions \citep[e.g.][]{HQM2011,Hu2016,Hu2017}. Other works employ some form of energy or momentum injection localized to the region immediately around a star particle \citep[e.g.][(\textit{need more})]{Agertz2013,Roskar2014,Ceverino2014,Forbes2016}. Although some of these approximate methods account for long-range effects of diffuse radiation \citep{HQM2012,Hopkins2018} most cases treat local radiation {\it only}, confined to energy or momentum injection in a limited physical region around a star particle. It is unclear what, if any, effect the long-range ionization and heating processes play in regulating star formation and driving galactic winds. Although the approximate radiation feedback models are capable of driving significant galactic outflows from galaxies, it is unclear if they capture the full effects of stellar ionizing radiation.

In this work we present a comparison of three simulations of the evolution of an isolated, low mass dwarf galaxy that account for detailed stellar feedback. Our fiducial model uses an adaptive ray-tracing radiative transfer method to follow stellar ionizing radiation. We test its effectiveness in regulating star formation and driving galactic winds by comparison to a simulation without any ionizing radiation and a simulation with only localized radiation around each star. We discuss our methods in Section~\ref{sec:methods} and present our results in Section~\ref{sec:results}.

\section{Methods and Initial Conditions} \label{sec:methods}
We refer the reader to Paper I for a more detailed description of our methods, briefly summarized here. We use the adaptive mesh refinement hydrodynamics code \textsc{Enzo} \citep{Enzo2014} to evolve an idealized, isolated low mass dwarf galaxy. The galaxy is initialized as a smooth exponential disk set in hydrostatic equillibrium with a static dark matter potential \citep{Burkert1995} with $M_{\rm gas} = 2 \times 10^6~$~M$_{\odot}$, radial and vertical gas scale heights of 250~pc and 100~pc respectively, $M_{\rm vir} = 2.48 \times 10^9$~M$_{\odot}$, and a maximum physical resolution of 1.8~pc. We do not model an initial stellar population, but do include random driving from supernovae as an initial source of initial and feedback up until first star formation.

We include a UV background, tabulated metal line cooling, and a 9 species non-equilibrium chemistry solver using \textsc{Grackle} \citep{GrackleMethod}. Star formation occurs stochastically in cold, dense regions ($n > 200$~cm$^{-3}$, $T < 200$~K) by sampling the \cite{Salpeter1955} IMF from 1~M$_{\odot}$ to 100~M$_{\odot}$ and depositing \textit{individual} star particles over this mass range. For each star, we include feedback from stellar winds, AGB winds, FUV and LW band radiation which drives photoelectric heating and H$_2$ dissociation respectively, HI and HeI ionizing radiation, and core collapse and Type Ia SNe using thermal energy injection. FUV and LW band radiation are both taken to be optically thin, with local (cell-by-cell) attenuation alone. Ionizing radiation is followed using radiative transfer, as discussed below. \textit{Maybe can leave this out:} Stellar lifetimes, surface gravity, effective temperature, and radii are set by the initial stellar mass and metallicity through interpolation over the PARSEC \citep{Bressan2012} zero age main sequence values. These values for surface gravity, effective temperature, and radii are used to set the FUV, LW, and ionizing photon fluxes from each star through interpolation on the OSTAR2002 \citep{LanHubeny2003} grid. We only follow the radiation from stars above 8~M$_{\odot}$. 
%tellar wind and supernovae mass loss are determined with the  NuGrid yield tables for $M_* < 25$~M$_{\odot}$ and from Slemer et. al. (in prep) for $M_* > 25$~M$_{\odot}$. 


%\textbf{Stellar Ionizing Radiation Models} $-$
In our fiducial simulation, we follow photoionizing radiation using the adaptive ray-tracing radiative transfer method of \cite{WiseAbel2011}. This method places and evolves 48 rays on a HEALPix grid around each emitting source star for both HI and HeI ionizing radiation. Rays are adaptively split as they propagate away from their source to increase the angular resolution such that the solid angle of the ray remains smaller than 1/3 of the cell area. Rays begin on HEALPix refinement level 2, with a maximum refinement level of 13 (or $8.05 \times 10^8$ rays).

We additionally present a simulation run without any stellar ionizing radiation (``noRT'') and a second simulation that includes ionizing radiation, but deletes all photons that travel more than 20~pc from their source (``shortrad''). This second simulation tests the relative importance of short-range vs. long-range effects of stellar ionizing radiation as a form of feedback. Although this is still more accurate than an approximate method, this is meant to function similarly to methods that include only the local effects of stellar radiation feedback. Each of our three simulations is identical up until the formation of the first star particles. At this point, the differing feedback physics causes each to diverge. We note that the stars from the very first star formation event in each run (21 star particles with a total mass of 114 M$_{\odot}$) are the same in each run. Of these, one particle emits ionizing radiation ($M_* > 8 $~M$_{\odot}$). 

\section{Results} \label{sec:results}

\begin{figure*}
\centering
\includegraphics[width=0.49\linewidth]{sfr}
\includegraphics[width=0.49\linewidth]{mass}
\caption{The star formation rate (left) and gas and stellar mass (right) of each of our three simulations over time.}
\label{fig:sfr_mass_evolution}
\end{figure*}

We compare the resulting star formation rate (left) and gas mass properties (right) of our three simulations in Figure~\ref{fig:sfr_mass_evolution}. There is a clear, significant contrast between the runs with and without ionizing radiation. Stellar ionizing radiation leads to a factor of $\sim$ 5 reduction in the resulting SFR, as compared to the noRT run. Since the shortrad and fiducial simulations are so similar over the first $\sim$~100~Myr, it is clear that stellar radiation acts to significantly reduce the local star formation efficiency around young, massive stars. Radiation from these stars quickly ionizes and dissipates surrounding dense gas that would otherwise have gone to fuel a significant amount of additional star formation. We are unable to follow the long-term evolution of the noRT galaxy due to computational constraints. Although radiative transfer is expensive, this run is substantially more computationally expensive than the fiducial and shortrad runs due to the significantly higher star formation rate.

Looking again at the first $\sim$100~Myr of simulation time, the effects of ionizing radiation beyond our 20~pc cutoff radius are not significant. However, these two simulations begin to diverge significantly after this point. The shortrad simulation has continual, steady star formation for the entire simulation time, while the star formation rate in our fiducial run is bursty, with periods of active, low-level star formation interspersed with periods (sometimes up to 50~Myr) of no star formation. The driver of these differences is clear in the right hand panel. Our fiducial run looses a significant amount of total gas and HI mass (black and blue lines) as compared to the shortrad simulation by a factor of $\sim$2. Clearly, galactic winds and outflows are much more effective in the fiducial case, with full accounting of stellar ionizing radiation feedback, then in the shortrad simulation.

This can be confirmed by examining the mass loading factor and metal retention fraction (the fraction of produced metals retained within the disk of the galaxy) in each simulation. As shown in Figure~\ref{fig:metal_retention}, the mass loading factor in the fiducial run is a factor of XXX higher at 0.25~R$_{\rm vir}$ than the shortrad simulation, and is the only simulation with any significant outflow beyond the virial radius. This is especially interesting considering the factor 5 increase in SFR of the noRT run. The noRT run does have a significant amount of gas mass outflowing at XXXX, but note that its low mass loading factor is due primarily to the significantly larger SFR in this galaxy. Radiation feedback, both locally and over longer ranges, is important in determining the ability for SN to drive significant outflows. We discuss the reasons for this in the following section.

\begin{figure}
\centering
\includegraphics[width=0.98\linewidth]{metal_retention}
\caption{The fraction of metals contained within the ISM of each galaxy (blue) and the fraction ejected beyond the virial radius (orange). As in Figure~\ref{fig:sfr_mass_evolution}, the fiducial run is given as solid lines, while the noRT and shortrad are dash-dotted and dashed respectively.}
\label{fig:metal_retention}
\end{figure}

\textbf{We have room for one more figure and a little more text to explain it. Possibly a comparison of mass / volume fractions in certain phases? But maybe not necessary....}

\section{Discussion} \label{sec:discussion}
Again, it is known that accounting for feedback from stellar radiation plays a significant role in determining the ability for SN energy to couple to the ISM and therefore drive outflows. However, the specific importance of localized ionization vs. ionization from a diffuse radiation field far from a single star has not been examined. As shown above, accounting for stellar radiation feedback locally alone is insufficient to describe the long-term evolution of an isolated dwarf galaxy. To explore the cause of this difference we present two panel-plots in Figure~\ref{fig:panel1} and Figure~\ref{fig:panel2} which compare the gas number density (left), temperature (middle), and hydrogen ionization fraction (right) in edge-on slices in each of our simulations at two different times.\footnote:{See ({\it insert link here once posted}) for a movie of this comparison.} 

Figure~\ref{fig:panel1}, at 17~Myr, compares each simulation just after the first few SNe. Already there are significant differences across runs. Gas outside the galaxy is warm ($\sim$~10$^{4}$~K) and ionized up to $\sim$500~pc above/below the plane of the disk in our fiducial run. This same gas is cold ($<10^4$~K) and neutral in both other runs. The contrast between the effect of ionizing radiation in the fiducial and shortrad runs is seen most clearly in the ionized region in-plane and to the right of center. This region contains massive stars that are capable of generating enough ionizing radiation to carve a channel out to the halo of our fiducial simulation; this does not occur in the shortrad case. Instead, the HII region is confined by surrounding cold, neutral gas. 

\begin{figure*}
\centering
\includegraphics[width=0.99\linewidth]{DD0136_fiducial_shortrad_nort}
\caption{\textbf{In the final version of this figure I will add labels for each row marking the simulation, and a line giving the scale of each figure.} Edge-on slices of each simulation showing gas number density (left), temperature (middle), and hydrogen ionization fraction (right) 17~Myr after the formation of the first star in each run.}
\label{fig:panel1}
\end{figure*}

Although the ISM properties within the HII region in each case are quite similar between the simulations, the SNe that will eventually go off in this region are confined by this neutral gas in the shortrad simulation, but are readily ejected into the galaxy halo in our fiducial case. As these simulations evolve, the existence of these diffuse, ionized channels in the fiducial run easily allow continual and significant outflows from SNe. In contrast, the same SNe in the other two simulations are well contained, surrounded by shells of denser, neutral gas. Though they make a significant local impact on the ISM, they are unable to drive significant mass loss from the galaxy. In the noRT case, outflow does eventually develop, but it takes the factor of 5 increase in SFR, and corresponding increase in supernova rate, for SNe to finally break out from the neutral gas surrounding the galaxy. These differences are shown more clearly in Figure~\ref{fig:panel2}, which gives each simulation 40~Myr after the formation of the first stars. 

\begin{figure*}
\centering
\includegraphics[width=0.99\linewidth]{DD0160_fiducial_shortrad_nort}
\caption{Same as Figure~\ref{fig:panel1}, but at 40~Myr.}
\label{fig:panel2}
\end{figure*}

\section{Conclusion}  \label{sec:conclusion}
In agreement with previous works, we find that stellar radiation is an important source of feedback that helps regulate star formation in dwarf galaxies. (more)

\bibliography{msbib} % if your bibtex file is called example.bib

%% This command is needed to show the entire author+affilation list when
%% the collaboration and author truncation commands are used.  It has to
%% go at the end of the manuscript.
%\allauthors

%% Include this line if you are using the \added, \replaced, \deleted
%% commands to see a summary list of all changes at the end of the article.
%\listofchanges

\end{document}

% End of file `sample62.tex'.
