%%
%% Beginning of file 'sample62.tex'
%%
%% Modified 2018 January
%%
%% This is a sample manuscript marked up using the
%% AASTeX v6.2 LaTeX 2e macros.
%%
%% AASTeX is now based on Alexey Vikhlinin's emulateapj.cls 
%% (Copyright 2000-2015).  See the classfile for details.

%% AASTeX requires revtex4-1.cls (http://publish.aps.org/revtex4/) and
%% other external packages (latexsym, graphicx, amssymb, longtable, and epsf).
%% All of these external packages should already be present in the modern TeX 
%% distributions.  If not they can also be obtained at www.ctan.org.

%% The first piece of markup in an AASTeX v6.x document is the \documentclass
%% command. LaTeX will ignore any data that comes before this command. The 
%% documentclass can take an optional argument to modify the output style.
%% The command below calls the preprint style  which will produce a tightly 
%% typeset, one-column, single-spaced document.  It is the default and thus
%% does not need to be explicitly stated.
%%
%%
%% using aastex version 6.2
\documentclass{aastex62}

%% The default is a single spaced, 10 point font, single spaced article.
%% There are 5 other style options available via an optional argument. They
%% can be envoked like this:
%%
%% \documentclass[argument]{aastex62}
%% 
%% where the layout options are:
%%
%%  twocolumn   : two text columns, 10 point font, single spaced article.
%%                This is the most compact and represent the final published
%%                derived PDF copy of the accepted manuscript from the publisher
%%  manuscript  : one text column, 12 point font, double spaced article.
%%  preprint    : one text column, 12 point font, single spaced article.  
%%  preprint2   : two text columns, 12 point font, single spaced article.
%%  modern      : a stylish, single text column, 12 point font, article with
%% 		  wider left and right margins. This uses the Daniel
%% 		  Foreman-Mackey and David Hogg design.
%%  RNAAS       : Preferred style for Research Notes which are by design 
%%                lacking an abstract and brief. DO NOT use \begin{abstract}
%%                and \end{abstract} with this style.
%%
%% Note that you can submit to the AAS Journals in any of these 6 styles.
%%
%% There are other optional arguments one can envoke to allow other stylistic
%% actions. The available options are:
%%
%%  astrosymb    : Loads Astrosymb font and define \astrocommands. 
%%  tighten      : Makes baselineskip slightly smaller, only works with 
%%                 the twocolumn substyle.
%%  times        : uses times font instead of the default
%%  linenumbers  : turn on lineno package.
%%  trackchanges : required to see the revision mark up and print its output
%%  longauthor   : Do not use the more compressed footnote style (default) for 
%%                 the author/collaboration/affiliations. Instead print all
%%                 affiliation information after each name. Creates a much
%%                 long author list but may be desirable for short author papers
%%
%% these can be used in any combination, e.g.
%%
%% \documentclass[twocolumn,linenumbers,trackchanges]{aastex62}
%%
%% AASTeX v6.* now includes \hyperref support. While we have built in specific
%% defaults into the classfile you can manually override them with the
%% \hypersetup command. For example,
%%
%%\hypersetup{linkcolor=red,citecolor=green,filecolor=cyan,urlcolor=magenta}
%%
%% will change the color of the internal links to red, the links to the
%% bibliography to green, the file links to cyan, and the external links to
%% magenta. Additional information on \hyperref options can be found here:
%% https://www.tug.org/applications/hyperref/manual.html#x1-40003
%%
%% If you want to create your own macros, you can do so
%% using \newcommand. Your macros should appear before
%% the \begin{document} command.
%%
\newcommand{\vdag}{(v)^\dagger}
\newcommand\aastex{AAS\TeX}
\newcommand\latex{La\TeX}

%% Tells LaTeX to search for image files in the 
%% current directory as well as in the figures/ folder.
\graphicspath{{./}{figures/}}

%% Reintroduced the \received and \accepted commands from AASTeX v5.2
\received{January 1, 2018}
\revised{January 7, 2018}
\accepted{\today}
%% Command to document which AAS Journal the manuscript was submitted to.
%% Adds "Submitted to " the arguement.
\submitjournal{ApJ}

%% Mark up commands to limit the number of authors on the front page.
%% Note that in AASTeX v6.2 a \collaboration call (see below) counts as
%% an author in this case.
%
%\AuthorCollaborationLimit=3
%
%% Will only show Schwarz, Muench and "the AAS Journals Data Scientist 
%% collaboration" on the front page of this example manuscript.
%%
%% Note that all of the author will be shown in the published article.
%% This feature is meant to be used prior to acceptance to make the
%% front end of a long author article more manageable. Please do not use
%% this functionality for manuscripts with less than 20 authors. Conversely,
%% please do use this when the number of authors exceeds 40.
%%
%% Use \allauthors at the manuscript end to show the full author list.
%% This command should only be used with \AuthorCollaborationLimit is used.

%% The following command can be used to set the latex table counters.  It
%% is needed in this document because it uses a mix of latex tabular and
%% AASTeX deluxetables.  In general it should not be needed.
%\setcounter{table}{1}

%%%%%%%%%%%%%%%%%%%%%%%%%%%%%%%%%%%%%%%%%%%%%%%%%%%%%%%%%%%%%%%%%%%%%%%%%%%%%%%%
%%
%% The following section outlines numerous optional output that
%% can be displayed in the front matter or as running meta-data.
%%
%% If you wish, you may supply running head information, although
%% this information may be modified by the editorial offices.
\shorttitle{Sample article}
\shortauthors{Schwarz et al.}
%%
%% You can add a light gray and diagonal water-mark to the first page 
%% with this command:
% \watermark{text}
%% where "text", e.g. DRAFT, is the text to appear.  If the text is 
%% long you can control the water-mark size with:
%  \setwatermarkfontsize{dimension}
%% where dimension is any recognized LaTeX dimension, e.g. pt, in, etc.
%%
%%%%%%%%%%%%%%%%%%%%%%%%%%%%%%%%%%%%%%%%%%%%%%%%%%%%%%%%%%%%%%%%%%%%%%%%%%%%%%%%

%% This is the end of the preamble.  Indicate the beginning of the
%% manuscript itself with \begin{document}.

\begin{document}

\title{An Example Article using \aastex v6.2\footnote{Released on January, 8th, 2018}}

%% LaTeX will automatically break titles if they run longer than
%% one line. However, you may use \\ to force a line break if
%% you desire. In v6.2 you can include a footnote in the title.

%% A significant change from earlier AASTEX versions is in the structure for 
%% calling author and affilations. The change was necessary to implement 
%% autoindexing of affilations which prior was a manual process that could 
%% easily be tedious in large author manuscripts.
%%
%% The \author command is the same as before except it now takes an optional
%% arguement which is the 16 digit ORCID. The syntax is:
%% \author[xxxx-xxxx-xxxx-xxxx]{Author Name}
%%
%% This will hyperlink the author name to the author's ORCID page. Note that
%% during compilation, LaTeX will do some limited checking of the format of
%% the ID to make sure it is valid.
%%
%% Use \affiliation for affiliation information. The old \affil is now aliased
%% to \affiliation. AASTeX v6.2 will automatically index these in the header.
%% When a duplicate is found its index will be the same as its previous entry.
%%
%% Note that \altaffilmark and \altaffiltext have been removed and thus 
%% can not be used to document secondary affiliations. If they are used latex
%% will issue a specific error message and quit. Please use multiple 
%% \affiliation calls for to document more than one affiliation.
%%
%% The new \altaffiliation can be used to indicate some secondary information
%% such as fellowships. This command produces a non-numeric footnote that is
%% set away from the numeric \affiliation footnotes.  NOTE that if an
%% \altaffiliation command is used it must come BEFORE the \affiliation call,
%% right after the \author command, in order to place the footnotes in
%% the proper location.
%%
%% Use \email to set provide email addresses. Each \email will appear on its
%% own line so you can put multiple email address in one \email call. A new
%% \correspondingauthor command is available in V6.2 to identify the
%% corresponding author of the manuscript. It is the author's responsibility
%% to make sure this name is also in the author list.
%%
%% While authors can be grouped inside the same \author and \affiliation
%% commands it is better to have a single author for each. This allows for
%% one to exploit all the new benefits and should make book-keeping easier.
%%
%% If done correctly the peer review system will be able to
%% automatically put the author and affiliation information from the manuscript
%% and save the corresponding author the trouble of entering it by hand.

\correspondingauthor{August Muench}
\email{greg.schwarz@aas.org, gus.muench@aas.org}

\author[0000-0002-0786-7307]{Greg J. Schwarz}
\affil{American Astronomical Society \\
2000 Florida Ave., NW, Suite 300 \\
Washington, DC 20009-1231, USA}

\author{August Muench}
\affiliation{American Astronomical Society \\
2000 Florida Ave., NW, Suite 300 \\
Washington, DC 20009-1231, USA}
\collaboration{(AAS Journals Data Scientists collaboration)}

\author{Butler Burton}
\affiliation{National Radio Astronomy Observatory}
\affiliation{AAS Journals Associate Editor-in-Chief}
\nocollaboration

\author{Amy Hendrickson}
\altaffiliation{Creator of AASTeX v6.2}
\affiliation{TeXnology Inc.}
\collaboration{(LaTeX collaboration)}

\author{Julie Steffen}
\affiliation{AAS Director of Publishing}
\affiliation{American Astronomical Society \\
2000 Florida Ave., NW, Suite 300 \\
Washington, DC 20009-1231, USA}

\author{Jeff Lewandowski}
\affiliation{IOP Senior Publisher for the AAS Journals}
\affiliation{IOP Publishing, Washington, DC 20005}

%% Note that the \and command from previous versions of AASTeX is now
%% depreciated in this version as it is no longer necessary. AASTeX 
%% automatically takes care of all commas and "and"s between authors names.

%% AASTeX 6.2 has the new \collaboration and \nocollaboration commands to
%% provide the collaboration status of a group of authors. These commands 
%% can be used either before or after the list of corresponding authors. The
%% argument for \collaboration is the collaboration identifier. Authors are
%% encouraged to surround collaboration identifiers with ()s. The 
%% \nocollaboration command takes no argument and exists to indicate that
%% the nearby authors are not part of surrounding collaborations.

%% Mark off the abstract in the ``abstract'' environment. 
\begin{abstract}

This example manuscript is intended to serve as a tutorial and template for
authors to use when writing their own AAS Journal articles. The manuscript
includes a history of \aastex\ and documents the new features in the
previous versions as well as the new features in version 6.2. This
manuscript includes many figure and table examples to illustrate these new
features.  Information on features not explicitly mentioned in the article
can be viewed in the manuscript comments or more extensive online
documentation. Authors are welcome replace the text, tables, figures, and
bibliography with their own and submit the resulting manuscript to the AAS
Journals peer review system.  The first lesson in the tutorial is to remind
authors that the AAS Journals, the Astrophysical Journal (ApJ), the
Astrophysical Journal Letters (ApJL), and Astronomical Journal (AJ), all
have a 250 word limit for the abstract\footnote{Note that manuscripts 
submitted to the new Research Notes of the American Astronomical Society 
(RNAAS) do \textbf{not} have abstracts.}.  If you exceed this length the
Editorial office will ask you to shorten it.

\end{abstract}

%% Keywords should appear after the \end{abstract} command. 
%% See the online documentation for the full list of available subject
%% keywords and the rules for their use.
\keywords{editorials, notices --- 
miscellaneous --- catalogs --- surveys}

%% From the front matter, we move on to the body of the paper.
%% Sections are demarcated by \section and \subsection, respectively.
%% Observe the use of the LaTeX \label
%% command after the \subsection to give a symbolic KEY to the
%% subsection for cross-referencing in a \ref command.
%% You can use LaTeX's \ref and \label commands to keep track of
%% cross-references to sections, equations, tables, and figures.
%% That way, if you change the order of any elements, LaTeX will
%% automatically renumber them.
%%
%% We recommend that authors also use the natbib \citep
%% and \citet commands to identify citations.  The citations are
%% tied to the reference list via symbolic KEYs. The KEY corresponds
%% to the KEY in the \bibitem in the reference list below. 

\section{Introduction} \label{sec:intro}

\latex\ \footnote{\url{http://www.latex-project.org/}} is a document markup
language that is particularly well suited for the publication of
mathematical and scientific articles \citep{lamport94}. \latex\ was written
in 1985 by Leslie Lamport who based it on the \TeX\ typesetting language
which itself was created by Donald E. Knuth in 1978.  In 1988 a suite of
\latex\ macros were developed to investigate electronic submission and
publication of AAS Journal articles \citep{1989BAAS...21..780H}.  Shortly
afterwards, Chris Biemesdefer merged these macros and more into a \latex\
2.08 style file called \aastex.  These early \aastex\ versions introduced
many common commands and practices that authors take for granted today.
Substantial revisions
were made by Lee Brotzman and Pierre Landau when the package was updated to
v4.0.  AASTeX v5.0, written in 1995 by Arthur Ogawa, upgraded to \latex\ 2e
which uses the document class in lieu of a style file.  Other improvements
to version 5 included hypertext support, landscape deluxetables and
improved figure support to facilitate electronic submission.  
\aastex\ v5.2 was released in 2005 and introduced additional graphics
support plus new mark up to identifier astronomical objects, datasets and
facilities.

In 1996 Maxim Markevitch modified the AAS preprint style file, aaspp4.sty,
to closely emulate the very tight, two column style of a typeset
Astrophysical Journal article.  The result was emulateapj.sty.  A year
later Alexey Vikhlinin took over development and maintenance.  In 2001 he
converted emulateapj into a class file in \latex\ 2e and in 2003 Vikhlinin
completely rewrote emulateapj based on the APS Journal's RevTEX class.

During this time emulateapj gained growing acceptance in the astronomical
community as it filled an author need to obtain an approximate number of
manuscript pages prior to submission for cost and length estimates. The
tighter typeset also had the added advantage of saving paper when printing
out hard copies.

Even though author publication charges are no longer based on print pages
\footnote{see Section \ref{sec:pubcharge} in the Appendix for more details
about how current article costs are calculated.} the emulateapj class file
has proven to be extremely popular with AAS Journal authors.  An informal
analysis of submitted \latex\ manuscripts in 2015 revealed that $\sim$65\%
either called emulateapj or have a commented emulateapj classfile call
indicating it was used at some stage of the manuscript construction.
Clearly authors want to have access to a tightly typeset version of the
article when corresponding with co-authors and for preprint submissions.

When planning the next \aastex\ release the popularity of emulateapj played
an important roll in the decision to drop the old base code and adopt and
modify emulateapj for \aastex\ v6.+ instead.  The change brings \aastex\
inline with what the majority of authors are already using while still
delivering new and improved features.  \aastex\ v6.0 through v6.2 were written
by Amy Hendrickson and released in January 2016 (v6.0), October 2016 (v6.1),
and January 2018 (v6.2), respectively.  Some of the new features in v6.0 
included:

\begin{enumerate}
\item improved citations for third party data repositories and software,
\item easier construction of matrix figures consisting of multiple 
encapsulated postscript (EPS) or portable document format (PDF) files,
\item figure set mark up for large collections of similar figures,
\item color mark up to easily enable/disable revised text highlighting,
\item improved url support, and
\item numerous table options such as the ability to hide columns, column 
decimal alignment, automatic column math mode and numbering, plus splitting of
wide tables.
\end{enumerate}

The features in v6.1 were:

\begin{enumerate}
\item ORCID support for preprints,
\item improved author, affiliation and collaboration mark up,
\item reintroduced the old AASTeX v5.2 {\tt\string\received}, 
      {\tt\string\revised}, {\tt\string\accepted}, and
      {\tt\string\published} commands plus
      added the new {\tt\string\submitjournal} command to document
      which AAS Journal the manuscript was submitted to, plus
\item new typeset style options.
\end{enumerate}

The new features in v6.2 are:

\begin{enumerate}
\item A new RNAAS style option for Research Note manuscripts,
\item Titles no longer put in all caps,
\item No page skip between the title page and article body,
\item re-introduce RevTeX's widetext environment for long lines in
      two column style formats, and
\item upgrade to the {\tt\string\doi} command.
\end{enumerate}

The rest of this article provides information and examples on how to create
your own AAS Journal manuscript with v6.2.  Special emphasis is placed on
how to use the full potential of \aastex\ v6+.  The next section describes
the different manuscript styles available and how they differ from past
releases.  Section \ref{sec:floats} describes how tables and figures are
placed in a \latex\ document. Specific examples of tables, Section
\ref{subsec:tables}, and figures, Section \ref{subsec:figures}, are also
provided.  Section \ref{sec:displaymath} discusses how to display math and
incorporate equations in a manuscript while Section \ref{sec:highlight}
discuss how to use the new revision mark up.  The last section,
\ref{sec:cite}, shows how recognize software and external data as first
class references in the manuscript bibliography.  An appendix is included
to show how to construct one and provide some information on how article
charges are calculated.  Additional information is available both embedded
in the comments of this \latex\ file and in the online documentation at
\url{http://journals.aas.org/authors/aastex.html}.

\section{Manuscript styles} \label{sec:style}

The default style in \aastex\ v6.2 is a tight single column style, e.g.  10
point font, single spaced.  The single column style is very useful for
article with wide equations. It is also the easiest to style to work with
since figures and tables, see Section \ref{sec:floats}, will span the
entire page, reducing the need for address float sizing.

To invoke a two column style similar to the what is produced in
the published PDF copy use \\

\noindent {\tt\string\documentclass[twocolumn]\{aastex62\}}. \\

\noindent Note that in the two column style figures and tables will only
span one column unless specifically ordered across both with the ``*'' flag,
e.g. \\

\noindent{\tt\string\begin\{figure*\}} ... {\tt\string\end\{figure*\}}, \\
\noindent{\tt\string\begin\{table*\}} ... {\tt\string\end\{table*\}}, and \\
\noindent{\tt\string\begin\{deluxetable*\}} ... {\tt\string\end\{deluxetable*\}}. \\

\noindent This option is ignored in the onecolumn style.

Some other style options are outlined in the commented sections of this 
article.  Any combination of style options can be used.

Two style options that are needed to fully use the new revision tracking
feature, see Section \ref{sec:highlight}, are {\tt\string linenumbers} which 
uses the lineno style file to number each article line in the left margin and 
{\tt\string trackchanges} which controls the revision and commenting highlight
output.

There is also a new {\tt\string modern} option that uses a Daniel
Foreman-Mackey and David Hogg design to produce stylish, single column
output that has wider left and right margins. It is designed to have fewer
words per line to improve reader retention. It also looks better on devices
with smaller displays such as smart phones.

For a Research Note use the {\tt\string RNAAS} option which will produce a
manuscript with no abstract and in the {\tt\string modern} style.

\section{Floats} \label{sec:floats}

Floats are non-text items that generally can not be split over a page.
They also have captions and can be numbered for reference.  Primarily these
are figures and tables but authors can define their own. \latex\ tries to
place a float where indicated in the manuscript but will move it later if
there is not enough room at that location, hence the term ``float''.

Authors are encouraged to embed their tables and figures within the text as
they are mentioned.  Please do not place the figures and text at the end of
the article as was the old practice.  Editors and the vast majority of
referees find it much easier to read a manuscript with embedded figures and
tables.

Depending on the number of floats and the particular amount of text and
equations present in a manuscript the ultimate location of any specific
float can be hard to predict prior to compilation. It is recommended that
authors textbf{not} spend significant time trying to get float placement
perfect for peer review.  The AAS Journal's publisher has sophisticated
typesetting software that will produce the optimal layout during
production.

Note that authors of Research Notes are only allowed one float, either one
table or one figure.

\startlongtable
\begin{deluxetable}{c|cc}
\tablecaption{ApJ costs from 1991 to 2013\tablenotemark{a} \label{tab:table}}
\tablehead{
\colhead{Year} & \colhead{Subscription} & \colhead{Publication} \\
\colhead{} & \colhead{cost} & \colhead{charges\tablenotemark{b}}\\
\colhead{} & \colhead{(\$)} & \colhead{(\$/page)}
}
\colnumbers
\startdata
1991 & 600 & 100 \\
1992 & 650 & 105 \\
1993 & 550 & 103 \\
1994 & 450 & 110 \\
1995 & 410 & 112 \\
1996 & 400 & 114 \\
1997 & 525 & 115 \\
1998 & 590 & 116 \\
1999 & 575 & 115 \\
2000 & 450 & 103 \\
2001 & 490 &  90 \\
2002 & 500 &  88 \\
2003 & 450 &  90 \\
2004 & 460 &  88 \\
2005 & 440 &  79 \\
2006 & 350 &  77 \\
2007 & 325 &  70 \\
2008 & 320 &  65 \\
2009 & 190 &  68 \\
2010 & 280 &  70 \\
2011 & 275 &  68 \\
2012 & 150 &  56 \\
2013 & 140 &  55 \\
\enddata
\tablenotetext{a}{Adjusted for inflation}
\tablenotetext{b}{Accounts for the change from page charges to digital quanta in April, 2011}
\tablecomments{Note that {\tt \string \colnumbers} does not work with the 
vertical line alignment token. If you want vertical lines in the headers you
can not use this command at this time.}
\end{deluxetable}

For authors that do want to take the time to optimize the locations of
their floats there are some techniques that can be used.  The simplest
solution is to placing a float earlier in the text to get the position
right but this option will break down if the manuscript is altered, see
Table \ref{tab:table}.  A better method is to force \latex\ to place a
float in a general area with the use of the optional {\tt\string [placement
specifier]} parameter for figures and tables. This parameter goes after
{\tt\string \begin\{figure\}}, {\tt\string \begin\{table\}}, and
{\tt\string \begin\{deluxetable\}}.  The main arguments the specifier takes
are ``h'', ``t'', ``b'', and ``!''.  These tell \latex\ to place the float
\underline{h}ere (or as close as possible to this location as possible), at
the \underline{t}op of the page, and at the \underline{b}ottom of the page.
The last argument, ``!'', tells \latex\ to override its internal method of
calculating the float position.  A sequence of rules can be created by
using multiple arguments.  For example, {\tt\string \begin\{figure\}[htb!]}
tells \latex\ to try the current location first, then the top of the page
and finally the bottom of the page without regard to what it thinks the
proper position should be.  Many of the tables and figures in this article
use a placement specifier to set their positions.

Note that the \latex\ {\tt\string tabular} environment is not a float.  Only
when a {\tt\string tabular} is surrounded by {\tt\string\begin\{table\}} ...
{\tt\string\end\{table\}} is it a true float and the rules and suggestions
above apply.

In AASTeX v6.2 all deluxetables are float tables and thus if they are
longer than a page will spill off the bottom. Long deluxetables should
begin with the {\tt\string\startlongtable} command. This initiates a
longtable environment.  Authors might have to use {\tt\string\clearpage} to
isolate a long table or optimally place it within the surrounding text.

\begin{deluxetable*}{ccCrlc}[b!]
\tablecaption{Column math mode in an observation log \label{tab:mathmode}}
\tablecolumns{6}
\tablenum{2}
\tablewidth{0pt}
\tablehead{
\colhead{UT start time\tablenotemark{a}} &
\colhead{MJD start time\tablenotemark{a}} &
\colhead{Seeing} & \colhead{Filter} & \colhead{Inst.} \\
\colhead{(YYYY-mm-dd)} & \colhead{(d)} &
\colhead{(arcsec)} & \colhead{} & \colhead{}
}
\startdata
2012-03-26 & 56012.997 & \sim 0.\arcsec5 & H$\alpha$ & NOT \\
2012-03-27 & 56013.944 & 1.\arcsec5 & grism & SMARTS \\
2012-03-28 & 56014.984 & \nodata & F814M & HST \\
2012-03-30 & 56016.978 & 1.\arcsec5\pm0.25 & B\&C & Bok \\
\enddata
\tablenotetext{a}{At exposure start.}
\tablecomments{The ``C'' command column identifier in the 3 column turns on
math mode for that specific column. One could do the same for the next
column so that dollar signs would not be needed for H$\alpha$
but then all the other text would also be in math mode and thus typeset
in Latin Modern math and you will need to put it back to Roman by hand.
Note that if you do change this column to math mode the dollar signs already
present will not cause a problem. Table \ref{tab:mathmode} is published 
in its entirety in the machine readable format.  A portion is
shown here for guidance regarding its form and content.}
\end{deluxetable*}

\subsection{Tables} \label{subsec:tables}

Tables can be constructed with \latex's standard table environment or the
\aastex's deluxetable environment. The deluxetable construct handles long
tables better but has a larger overhead due to the greater amount of
defined mark up used set up and manipulate the table structure.  The choice
of which to use is up to the author.  Examples of both environments are
used in this manuscript. Table \ref{tab:table} is a simple deluxetable
example that gives the approximate changes in the subscription costs and
author publication charges from 1991 to 2013.

Tables longer than 200 data lines and complex tables should only have a
short example table with the full data set available in the machine
readable format.  The machine readable table will be available in the HTML
version of the article with just a short example in the PDF. Authors are
required to indicate to the reader where the data can be obtained in the
table comments.  Suggested text is given in the comments of Table
\ref{tab:mathmode}.  Authors are encouraged to create their own machine
readable tables using the online tool at
\url{http://authortools.aas.org/MRT/upload.html}.

\aastex\ v6 introduces five new table features that are designed to make
table construction easier and the resulting display better for AAS Journal
authors.  The items are:

\begin{enumerate}
\item Declaring math mode in specific columns,
\item Column decimal alignment, 
\item Automatic column header numbering,
\item Hiding columns, and
\item Splitting wide tables into two or three parts.
\end{enumerate}

Each of these new features are illustrated in following Table examples.
All five features work with the regular \latex\ tabular environment and in
\aastex's deluxetable environment.  The examples in this manuscript also
show where the two process differ.

\subsubsection{Column math mode}

Both the \latex\ tabular and \aastex\ deluxetable require an argument to
define the alignment and number of columns.  The most common values are
``c'', ``l'' and ``r'' for \underline{c}enter, \underline{l}eft, and
\underline{r}ight justification.  If these values are capitalized, e.g.
``C'', ``L'', or ``R'', then that specific column will automatically be in math
mode meaning that \$s are not required.  Note that having embedded dollar
signs in the table does not affect the output.  The third and forth columns
of Table \ref{tab:mathmode} shows how this math mode works.

\subsubsection{Decimal alignment}

Aligning a column by the decimal point can be difficult with only center,
left, and right justification options.  It is possible to use phantom calls
in the data, e.g. {\tt\string\phn}, to align columns by hand but this can
be tedious in long or complex tables.  To address this \aastex\ introduces
the {\tt\string\decimals} command and a new column justification option,
``D'', to align data in that column on the decimal.  In deluxetable the
{\tt\string\decimals} command is invoked before the {\tt\string\startdata}
call but can be anywhere in \latex's tabular environment.  

Two other important thing to note when using decimal alignment is that each
decimal column \textit{must end with a space before the ampersand}, e.g.
``\&\&'' is not allowed.  Empty decimal columns are indicated with a decimal,
e.g. ``.''.  Do not use deluxetable's {\tt\string\nodata} command.

The ``D'' alignment token works by splitting the column into two parts on the
decimal.  While this is invisible to the user one must be aware of how it
works so that the headers are accounted for correctly.  All decimal column
headers need to span two columns to get the alignment correct. This can be
done with a multicolumn call, e.g {\tt\string\multicolumn2c\{\}} or
{\tt\string\multicolumn\{2\}\{c\}\{\}}, or use the new
{\tt\string\twocolhead\{\}} command in deluxetable.  Since \latex\ is
splitting these columns into two it is important to get the table width
right so that they appear joined on the page.  You may have to run the
\latex\ compiler twice to get it right.  Table \ref{tab:decimal}
illustrates how decimal alignment works in the tabular environment with a
$\pm$ symbol embedded between the last two columns.

%% Note that the \setcounter and \renewcommand are needed here because
%% this example is using a mix of deluxetable and tabular.  Here the
%% deluxetable counters are set with \tablenum but the situation is a bit
%% more complex for tabular.  Use the first command to set the Table number
%% to ONE LESS than it should be.  The next command will auto increment it
%% to the desired number.
\setcounter{table}{2}
\begin{table}[h!]
\renewcommand{\thetable}{\arabic{table}}
\centering
\caption{Decimal alignment made easy} \label{tab:decimal}
\begin{tabular}{cD@{$\pm$}D}
\tablewidth{0pt}
\hline
\hline
Column & \multicolumn2c{Value} & \multicolumn2c{Uncertainty}\\
\hline
\decimals
A & 1234     & 100.0     \\
B &  123.4   &  10.1     \\
C &  12.34   &   1.01    \\
D &   1.234  &   0.101   \\
E &    .1234 &   0.01001 \\
F &   1.0    &    .      \\
\hline
\multicolumn{5}{c}{NOTE. - Two decimal aligned columns}
\end{tabular}
\end{table}

\subsubsection{Automatic column header numbering} \label{subsubsec:autonumber}

The command {\tt\string\colnumbers} can be included to automatically number
each column as the last row in the header. Per the AAS Journal table format
standards, each column index numbers will be surrounded by parentheses. In
a \latex\ tabular environment the {\tt\string\colnumbers} should be invoked
at the location where the author wants the numbers to appear, e.g. after
the last line of specified table header rows. In deluxetable this command
has to come before {\tt\string\startdata}.  {\tt\string\colnumbers} will
not increment for columns hidden by the ``h'' command, see Section
\ref{subsubsec:hide}.  Table \ref{tab:table} uses this command to
automatically generate column index numbers.

Note that when using decimal alignment in a table the command 
{\tt\string\decimalcolnumbers} must be used instead of 
{\tt\string\colnumbers} and {\tt\string\decimals}. Table \ref{tab:messier}
illustrates this specific functionality.

\subsubsection{Hiding columns} \label{subsubsec:hide}

Entire columns can be \underline{h}idden from display simply by changing
the specified column identifier to ``h''.  In the \latex\ tabular environment
this column identifier conceals the entire column including the header
columns.   In \aastex's deluxetables the header row is specifically
declared with the {\tt\string\tablehead} call and each header column is
marked with {\tt\string\colhead} call.  In order to make a specific header
disappear with the ``h'' column identifier in deluxetable use 
{\tt\string\nocolhead} instead to suppress that particular column header.

Authors can use this option in many different ways.  Since column data can
be easily suppressed authors can include extra information and hid it
based on the comments of co-authors or referees.  For wide tables that will
have a machine readable version, authors could put all the information in
the \latex\ table but use this option to hid as many columns as needed until
it fits on a page. This concealed column table would serve as the
example table for the full machine readable version.  Regardless of how
columns are obscured, authors are responsible for removing any unneeded
column data or alerting the editorial office about how to treat these
columns during production for the final typeset article.

Table \ref{tab:messier} provides some basic information about the first ten
Messier Objects and illustrates how many of these new features can be used
together.  It has automatic column numbering, decimal alignment of the
distances, and one concealed column.  The Common name column
is the third in the \latex\ deluxetable but does not appear when the article
is compiled. This hidden column can be shown simply by changing the ``h'' in
the column identifier preamble to another valid value.  This table also
uses {\tt\string\tablenum} to renumber the table because a \latex\ tabular
table was inserted before it.

\begin{deluxetable*}{cchlDlc}
\tablenum{4}
\tablecaption{Fun facts about the first 10 messier objects\label{tab:messier}}
\tablewidth{0pt}
\tablehead{
\colhead{Messier} & \colhead{NGC/IC} & \nocolhead{Common} & \colhead{Object} &
\multicolumn2c{Distance} & \colhead{} & \colhead{V} \\
\colhead{Number} & \colhead{Number} & \nocolhead{Name} & \colhead{Type} &
\multicolumn2c{(kpc)} & \colhead{Constellation} & \colhead{(mag)}
}
\decimalcolnumbers
\startdata
M1 & NGC 1952 & Crab Nebula & Supernova remnant & 2 & Taurus & 8.4 \\
M2 & NGC 7089 & Messier 2 & Cluster, globular & 11.5 & Aquarius & 6.3 \\
M3 & NGC 5272 & Messier 3 & Cluster, globular & 10.4 & Canes Venatici &  6.2 \\
M4 & NGC 6121 & Messier 4 & Cluster, globular & 2.2 & Scorpius & 5.9 \\
M5 & NGC 5904 & Messier 5 & Cluster, globular & 24.5 & Serpens & 5.9 \\
M6 & NGC 6405 & Butterfly Cluster & Cluster, open & 0.31 & Scorpius & 4.2 \\
M7 & NGC 6475 & Ptolemy Cluster & Cluster, open & 0.3 & Scorpius & 3.3 \\
M8 & NGC 6523 & Lagoon Nebula & Nebula with cluster & 1.25 & Sagittarius & 6.0 \\
M9 & NGC 6333 & Messier 9 & Cluster, globular & 7.91 & Ophiuchus & 8.4 \\
M10 & NGC 6254 & Messier 10 & Cluster, globular & 4.42 & Ophiuchus & 6.4 \\
\enddata
\tablecomments{This table ``hides'' the third column in the \latex\ when compiled.
The Distance is also centered on the decimals.  Note that when using decimal
alignment you need to include the {\tt\string\decimals} command before
{\tt\string\startdata} and all of the values in that column have to have a
space before the next ampersand.}
\end{deluxetable*}

\subsubsection{Splitting a table into multiple horizontal components}

Since the AAS Journals are now all electronic with no print version there is
no reason why tables can not be as wide as authors need them to be.
However, there are some artificial limitations based on the width of a
print page.  The old way around this limitation was to rotate into 
landscape mode and use the smallest available table font
sizes, e.g. {\tt\string\tablewidth}, to get the table to fit.
Unfortunately, this was not alway enough but now along with the hide column
option outlined in Section \ref{subsubsec:hide} there is a new way to break
a table into two or three components so that it flows down a page by
invoking a new table type, splittabular or splitdeluxetable. Within these
tables a new ``B'' column separator is introduced.  Much like the vertical
bar option, ``$\vert$'', that produces a vertical table lines, e.g. Table
\ref{tab:table}, the new ``B'' separator indicates where to \underline{B}reak
a table.  Up to two ``B''s may be included.

Table 5 % \ref{tab:deluxesplit} this freaks it out when it is used!
shows how to split a wide deluxetable into three parts with
the {\tt\string\splitdeluxetable} command.  The {\tt\string\colnumbers}
option is on to show how the automatic column numbering carries through the
second table component, see Section \ref{subsubsec:autonumber}.

The last example, Table \ref{tab:tablesplit}, shows how to split the same
table but with a regular \latex\ tabular call and into two parts. Decimal
alignment is included in the third column and the ``Component'' column is
hidden to illustrate the new features working together.

\begin{splitdeluxetable*}{lccccBcccccBcccc}
\tabletypesize{\scriptsize}
\tablewidth{0pt} 
\tablenum{5}
\tablecaption{Measurements of Emission Lines: two breaks \label{tab:deluxesplit}}
\tablehead{
\colhead{Model} & \colhead{Component}& \colhead{Shift} & \colhead{FWHM} &
\multicolumn{10}{c}{Flux} \\
\colhead{} & \colhead{} & \colhead{($\rm
km~s^{-1}$)}& \colhead{($\rm km~s^{-1}$)} & \multicolumn{10}{c}{($\rm
10^{-17}~erg~s^{-1}~cm^{-2}$)} \\
\cline{5-14}
\colhead{} & \colhead{} &
\colhead{} & \colhead{} & \colhead{Ly$\alpha$} & \colhead{N\,{\footnotesize
V}} & \colhead{Si\,{\footnotesize IV}} & \colhead{C\,{\footnotesize IV}} &
\colhead{Mg\,{\footnotesize II}} & \colhead{H$\gamma$} & \colhead{H$\beta$}
& \colhead{H$\alpha$} & \colhead{He\,{\footnotesize I}} &
\colhead{Pa$\gamma$}
} 
\colnumbers
\startdata 
{       }& BELs& -97.13 &    9117$\pm      38$&    1033$\pm      33$&$< 35$&$<     166$&     637$\pm      31$&    1951$\pm      26$&     991$\pm 30$&    3502$\pm      42$&   20285$\pm      80$&    2025$\pm     116$& 1289$\pm     107$\\ 
{Model 1}& IELs& -4049.123 & 1974$\pm      22$&    2495$\pm      30$&$<     42$&$<     109$&     995$\pm 186$&      83$\pm      30$&      75$\pm      23$&     130$\pm      25$& 357$\pm      94$&     194$\pm      64$& 36$\pm      23$\\
{       }& NELs& \nodata &     641$\pm       4$&     449$\pm 23$&$<      6$&$<       9$&       --            &     275$\pm      18$& 150$\pm      11$&     313$\pm      12$&     958$\pm      43$&     318$\pm 34$& 151$\pm       17$\\
\hline
{       }& BELs& -85 &    8991$\pm      41$& 988$\pm      29$&$<     24$&$<     173$&     623$\pm      28$&    1945$\pm 29$&     989$\pm      27$&    3498$\pm      37$&   20288$\pm      73$& 2047$\pm     143$& 1376$\pm     167$\\
{Model 2}& IELs& -51000 &    2025$\pm      26$& 2494$\pm      32$&$<     37$&$<     124$&    1005$\pm     190$&      72$\pm 28$&      72$\pm      21$&     113$\pm      18$&     271$\pm      85$& 205$\pm      72$& 34$\pm      21$\\
{       }& NELs& 52 &     637$\pm      10$&     477$\pm 17$&$<      4$&$<       8$&       --            &     278$\pm      17$& 153$\pm      10$&     317$\pm      15$&     969$\pm      40$&     325$\pm 37$&
     147$\pm       22$\\
\enddata
\tablecomments{This is an example of how to split a deluxetable. You can
split any table with this command into two or three parts.  The location of
the split is given by the author based on the placement of the ``B''
indicators in the column identifier preamble.  For more information please
look at the new \aastex\ instructions.}
\end{splitdeluxetable*}

%\clearpage

\setcounter{table}{5}
\begin{table*}[h!]
\renewcommand{\thetable}{\arabic{table}}
\caption{Measurements of Emission Lines: one break\label{tab:tablesplit}}
\begin{splittabular}{lhDccccBccccccc}
%\multicolumn{5}{c}{Table 6} \\
%\multicolumn{5}{c}{Measurements of Emission Lines} \\
\hline 
\hline 
Model & Component & \multicolumn2c{Shift} & FWHM & 
\multicolumn{10}{c}{Flux} \\
 & & \multicolumn2c{($\rm km~s^{-1}$)} & {($\rm km~s^{-1}$)} & 
\multicolumn{10}{c}{($\rm 10^{-17}~erg~s^{-1}~cm^{-2}$)} \\
\cline{5-15}
 & & & & & {Ly$\alpha$} & {N\,{\footnotesize V}} & 
{Si\,{\footnotesize IV}} & {C\,{\footnotesize IV}} &
{Mg\,{\footnotesize II}} & {H$\gamma$} & {H$\beta$}
& {H$\alpha$} & {He\,{\footnotesize I}} & {Pa$\gamma$} \\
%\hline
\decimalcolnumbers
 & BELs& -97.13 &    9117$\pm      38$&    1033$\pm      33$&$< 35$&$<     166$&     637$\pm      31$&    1951$\pm      26$&     991$\pm 30$&    3502$\pm      42$&   20285$\pm      80$&    2025$\pm     116$& 1289$\pm     107$\\
Model 1 & IELs& -4049.123 & 1974$\pm      22$&    2495$\pm      30$&$<     42$&$<     109$&     995$\pm 186$&      83$\pm      30$&      75$\pm      23$&     130$\pm      25$& 357$\pm      94$&     194$\pm      64$& 36$\pm      23$\\
 & NELs& . &     641$\pm       4$&     449$\pm 23$&$<      6$&$<       9$&       --            &     275$\pm      18$& 150$\pm      11$&     313$\pm      12$&     958$\pm      43$&     318$\pm 34$& 151$\pm       17$\\
\hline
 & BELs& -85 &    8991$\pm      41$& 988$\pm      29$&$<     24$&$<     173$&     623$\pm      28$&    1945$\pm 29$&     989$\pm      27$&    3498$\pm      37$&   20288$\pm      73$& 2047$\pm     143$& 1376$\pm     167$\\
Model 2 & IELs& -51000 &    2025$\pm      26$& 2494$\pm      32$&$<     37$&$<     124$&    1005$\pm     190$&      72$\pm 28$&      72$\pm      21$&     113$\pm      18$&     271$\pm      85$& 205$\pm      72$& 34$\pm      21$\\
 & NELs& 52 &     637$\pm      10$&     477$\pm 17$&$<      4$&$<       8$&       --            &     278$\pm      17$& 153$\pm      10$&     317$\pm      15$&     969$\pm      40$&     325$\pm 37$& 147$\pm       22$\\
\hline
\end{splittabular}
\end{table*}

\subsection{Figures\label{subsec:figures}}

%% The "ht!" tells LaTeX to put the figure "here" first, at the "top" next
%% and to override the normal way of calculating a float position
\begin{figure}[ht!]
\plotone{cost-eps-converted-to.pdf}
\caption{The subscription and author publication costs from 1991 to 2013.
The data comes from Table \ref{tab:table}.\label{fig:general}}
\end{figure}

Authors can include a wide number of different graphics with their articles
in  encapsulated postscript (EPS) or portable document format (PDF).  These
range from general figures all authors are familiar with to new enhanced
graphics that can only be fully experienced in HTML.  The later include
animations, figure sets and interactive figures.  This portion of the
article provides examples for setting up all these graphics in with the
latest version of \aastex.

\subsection{General figures\label{subsec:general}}

\aastex\ has a {\tt\string\plotone} command to display a figure
consisting of one EPS/PDF file.  Figure \ref{fig:general} is an example
which uses the data from Table \ref{tab:table}. For a general figure
consisting of two EPS/PDF files the {\tt\string\plottwo} command can be
used to position the two image files side by side.  Figure \ref{fig:f2}
shows the Swift/XRT X-ray light curves of two recurrent novae.  The data
from Figures \ref{fig:f2} through \ref{fig:fig4} are taken from Table 2 of
\citet{2011ApJS..197...31S}. 

\begin{figure}
\plottwo{RS_Oph-eps-converted-to.pdf}{U_Sco-eps-converted-to.pdf}
\caption{Swift/XRT X-ray light curves of RS Oph and U Sco which represent
the two canonical recurrent types, a long period system with a red giant
secondary and a short period system with a dwarf/sub-dwarf secondary,
respectively.\label{fig:f2}}
\end{figure}

Both {\tt\string\plotone} and {\tt\string\plottwo} take a
{\tt\string\caption} and an optional {\tt\string\figurenum} command to
specify the figure number\footnote{It is better to not use
{\tt\string\figurenum} and let LaTeX auto-increment all the figures. If you
do use this command you need to mark all of them accordingly.}.  Each is
based on the {\tt\string graphicx} package command,
{\tt\string\includegraphics}.  Authors are welcome to use
{\tt\string\includegraphics} along with its optional arguments that control
the height, width, scale, and position angle of a file within the figure.
More information on the full usage of {\tt\string\includegraphics} can be
found at \break
\url{https://en.wikibooks.org/wiki/LaTeX/Importing\_Graphics\#Including\_graphics}.

\subsection{Grid figures}

Including more than two EPS/PDF files in a single figure call can be tricky
easily format.  To make the process easier for authors \aastex\ v6 offers
{\tt\string\gridline} which allows any number of individual EPS/PDF file
calls within a single figure.  Each file cited in a {\tt\string\gridline}
will be displayed in a row.  By adding more {\tt\string\gridline} calls an
author can easily construct a matrix X by Y individual files as a
single general figure.

For each {\tt\string\gridline} command a EPS/PDF file is called by one of
four different commands.  These are {\tt\string\fig},
{\tt\string\rightfig}, {\tt\string\leftfig}, and {\tt\string\boxedfig}.
The first file call specifies no image position justification while the
next two will right and left justify the image, respectively.  The
{\tt\string\boxedfig} is similar to {\tt\string\fig} except that a box is
drawn around the figure file when displayed. Each of these commands takes
three arguments.  The first is the file name.  The second is the width that
file should be displayed at.  While any natural \latex\ unit is allowed, it
is recommended that author use fractional units with the
{\tt\string\textwidth}.  The last argument is text for a subcaption.

Figure \ref{fig:pyramid} shows an inverted pyramid of individual
figure constructed with six individual EPS files using the
{\tt\string\gridline} option.

\begin{figure*}
\gridline{\fig{V2491_Cyg-eps-converted-to.pdf}{0.3\textwidth}{(a)}
          \fig{CSS081007-eps-converted-to.pdf}{0.3\textwidth}{(b)}
          \fig{LMC_2009-eps-converted-to.pdf}{0.3\textwidth}{(c)}
          }
\gridline{\fig{RS_Oph-eps-converted-to.pdf}{0.3\textwidth}{(d)}
          \fig{U_Sco-eps-converted-to.pdf}{0.3\textwidth}{(e)}
          }
\gridline{\fig{KT_Eri-eps-converted-to.pdf}{0.3\textwidth}{(f)}}
\caption{Inverted pyramid figure of six individual files. The nova are
(a) V2491 Cyg, (b) HV Cet, (c) LMC 2009, (d) RS Oph, (e) U Sco, and
(f) KT Eri.\label{fig:pyramid}}
\end{figure*}

\subsection{Figure sets}

A large collection of similar style figures should be grouped together as a
figure set.  The derived PDF article will only shows an example figure
while the enhanced content is available in the figure set in the electronic
edition.  The advantage of a figure set gives the reader the ability to
easily sort through the figure collection to find individual component
figures.  All of the figure set components, along with their html framework,
are also available for download in a .tar.gz package.

Special \latex\ mark up is required to create a figure set.  Prior to
\aastex\ v6 the underlying mark up commands had to be inserted by hand
but is now included.  Note that when an article with figure set is compiled
in \latex\ none of the component figures are shown and a floating Figure
Set caption will appear in the resulting PDF.

\figsetstart
\figsetnum{4}
\figsettitle{Swift X-ray light curves}

\figsetgrpstart
\figsetgrpnum{1.1}
\figsetgrptitle{KT Eri}
\figsetplot{KT_Eri-eps-converted-to.pdf}
\figsetgrpnote{The Swift/XRT X-ray light curve for the first year after
outburst.}
\figsetgrpend

\figsetgrpstart
\figsetgrpnum{1.2}
\figsetgrptitle{RS Oph}
\figsetplot{RS_Oph-eps-converted-to.pdf}
\figsetgrpnote{The Swift/XRT X-ray light curve for the first year after
outburst.}
\figsetgrpend

\figsetgrpstart
\figsetgrpnum{1.3}
\figsetgrptitle{U Sco}
\figsetplot{U_Sco-eps-converted-to.pdf}
\figsetgrpnote{The Swift/XRT X-ray light curve for the first year after
outburst.}
\figsetgrpend

\figsetgrpstart
\figsetgrpnum{1.4}
\figsetgrptitle{V2491 Cyg}
\figsetplot{V2491_Cyg-eps-converted-to.pdf}
\figsetgrpnote{The Swift/XRT X-ray light curve for the first year after
outburst.}
\figsetgrpend

\figsetgrpstart
\figsetgrpnum{1.5}
\figsetgrptitle{Nova LMC 2009}
\figsetplot{LMC_2009-eps-converted-to.pdf}
\figsetgrpnote{The Swift/XRT X-ray light curve for the first year after
outburst.}
\figsetgrpend

\figsetgrpstart
\figsetgrpnum{1.6}
\figsetgrptitle{HV Cet}
\figsetplot{CSS081007-eps-converted-to.pdf}
\figsetgrpnote{The Swift/XRT X-ray light curve for the first year after
outburst.}
\figsetgrpend

\figsetgrpstart
\figsetgrpnum{1.7}
\figsetgrptitle{V2672 Oph}
\figsetplot{V2672_Oph-eps-converted-to.pdf}
\figsetgrpnote{The Swift/XRT X-ray light curve for the first year after
outburst.}
\figsetgrpend

\figsetgrpstart
\figsetgrpnum{1.8}
\figsetgrptitle{V407 Cyg}
\figsetplot{V407_Cyg-eps-converted-to.pdf}
\figsetgrpnote{The Swift/XRT X-ray light curve for the first year after
outburst.}
\figsetgrpend

\figsetend

\begin{figure}
\plotone{KT_Eri-eps-converted-to.pdf}
\caption{The Swift/XRT X-ray light curve for the first year after
outburst of the suspected recurrent nova KT Eri. At a maximum count rate of 
328 ct/s, KT Eri was the brightest nova in X-rays observed to date. All 
the component figures are available in the Figure Set. \label{fig:fig4}}
\end{figure}

Authors are encouraged to use an online tool at
\url{http://authortools.aas.org/FIGSETS/make-figset.html} to generate their
own specific figure set mark up to incorporate into their \latex\ articles.

\subsection{Animations}

Authors may include animations in their articles.  A single still frame from 
the animation should be included as a regular figure to serve as an example.
The associated figure caption should indicate to the reader exactly what the
animation shows and that the animation is available online.

\begin{figure}
\plotone{video3-eps-converted-to.pdf}
\caption{Example image from the animation which is available in the electronic
edition.}
\end{figure}

\subsection{Interactive figures}

Interactive figures give the reader the ability to manipulate the
information contained in an image which can add clarity or help further the
author's narrative.  These figures consist of two parts, the figure file in
a specific format and a javascript and html frame work that provides the
interactive control.  An example of an interactive figure is a 3D model.
The underlying figure is a X3D file while x3dom.js is the javascript driver
that displays it. An author created interface is added via a html wrapper.
The first 3D model published by the AAS Journals using this technique was
\citet{2014ApJ...793..127V}.  Authors should consult the online tutorials
for more information on how to construct their own interactive figures.

As with animations authors should include a non-interactive regular figure
to use as an example.  The example figure should also indicate to the reader
that the enhanced figure is interactive and can be accessed online.

\section{Displaying mathematics} \label{sec:displaymath}

The most common mathematical symbols and formulas are in the amsmath
package.  \aastex\ requires this package so there is no need to
specifically call for it in the document preamble.  Most modern \latex\
distributions already contain this package.  If you do not have this
package or the other required packages, revtex4-1, latexsym, graphicx,
amssymb, longtable, and epsf, they can be obtained from 
\url{http://www.ctan.org}

Mathematics can be displayed either within the text, e.g. $E = mc^2$, or
separate from in an equation.  In order to be properly rendered, all inline
math text has to be declared by surrounding the math by dollar signs (\$).

A complex equation example with inline math as part of the explanation
follows.

\begin{equation}
\bar v(p_2,\sigma_2)P_{-\tau}\hat a_1\hat a_2\cdots
\hat a_nu(p_1,\sigma_1) ,
\end{equation}
where $p$ and $\sigma$ label the initial $e^{\pm}$ four-momenta
and helicities $(\sigma = \pm 1)$, $\hat a_i=a^\mu_i\gamma_\nu$
and $P_\tau=\frac{1}{2}(1+\tau\gamma_5)$ is a chirality projection
operator $(\tau = \pm1)$.  This produces a single line formula.  \latex\ will
auto-number this and any subsequent equations.  If no number is desired then
the {\tt\string equation} call should be replaced with {\tt\string displaymath}.

\latex\ can also handle a a multi-line equation.  Use {\tt\string eqnarray}
for more than one line and end each line with a
\textbackslash\textbackslash.  Each line will be numbered unless the
\textbackslash\textbackslash\ is preceded by a {\tt\string\nonumber}
command.  Alignment points can be added with ampersands (\&).  There should be
two ampersands per line. In the examples they are centered on the equal
symbol.

\begin{eqnarray}
\gamma^\mu  & = &
 \left(
\begin{array}{cc}
0 & \sigma^\mu_+ \\
\sigma^\mu_- & 0
\end{array}     \right) ,
 \gamma^5= \left(
\begin{array}{cc}
-1 &   0\\
0 &   1
\end{array}     \right)  , \\
\sigma^\mu_{\pm}  & = &   ({\bf 1} ,\pm \sigma) , 
\end{eqnarray}

\begin{eqnarray}
\hat a & = & \left(
\begin{array}{cc}
0 & (\hat a)_+\\
(\hat a)_- & 0
\end{array}\right), \nonumber \\
(\hat a)_\pm & = & a_\mu\sigma^\mu_\pm 
\end{eqnarray}

%% Putting eqnarrays or equations inside the mathletters environment groups
%% the enclosed equations by letter. For instance, the eqnarray below, instead
%% of being numbered, say, (4) and (5), would be numbered (4a) and (4b).
%% LaTeX the paper and look at the output to see the results.

\section{Revision tracking and color highlighting} \label{sec:highlight}

Authors sometimes use color to highlight changes to their manuscript in
response to editor and referee comments.  In \aastex\ new commands
have been introduced to make this easier and formalize the process. 

The first method is through a new set of editing mark up commands that
specifically identify what has been changed.  These commands are
{\tt\string\added\{<text>\}}, {\tt\string\deleted\{<text>\}}, and
{\tt\string\replaced\{<old text>\}\{<replaced text>\}}. To activate these
commands the {\tt\string trackchanges} option must be used in the
{\tt\string\documentclass} call.  When compiled this will produce the
marked text in red.  The {\tt\string\explain\{<text>\}} can be used to add
text to provide information to the reader describing the change.  Its
output is purple italic font. To see how {\tt\string\added\{<important
added info>\}}, {\tt\string\deleted\{<this can be deleted text>\}},
{\tt\string\replaced\{<old data>\}\{<replaced data>\}}, and \break
{\tt\string\explain\{<text explaining the change>\}} commands will produce
\added{important added information}\deleted{, deleted text, and }
\replaced{old data}{and replaced data,} toggle between versions compiled with
and without the {\tt\string trackchanges} option.\explain{text explaining
the change}

A summary list of all these tracking commands can be produced at the end of
the article by adding the {\tt\string\listofchanges} just before the
{\tt\string\end\{document\}} call.  The page number for each change will be
provided. If the {\tt\string linenumbers} option is also included in the
documentcall call then not only will all the lines in the article be
numbered for handy reference but the summary list will also include the
line number for each change.

The second method does not have the ability to highlight the specific
nature of the changes but does allow the author to document changes over
multiple revisions.  The commands are {\tt\string\edit1\{<text>\}},
{\tt\string\edit2\{<text>\}} and {\tt\string\edit3\{<text>\}} and they
produce {\tt\string<text>} that is highlighted in bold red, italic blue and
underlined purple, respectively.  Authors should use the first command to
\edit1{indicated which text has been changed from the first revision.}  The
second command is to highlight \edit2{new or modified text from a second
revision}.  If a third revision is needed then the last command should be used 
\edit3{to show this changed text}.  Since over 90\% of all manuscripts are
accepted after the 3rd revision these commands make it easy to identify
what text has been added and when.  Once the article is accepted all the
highlight color can be turned off simply by adding the
{\tt\string\turnoffediting} command in the preamble. Likewise, the new commands
{\tt\string\turnoffeditone}, {\tt\string\turnoffedittwo}, and
{\tt\string\turnoffeditthree} can be used to only turn off the 
{\tt\string\edit1\{<text>\}}, {\tt\string\edit2\{<text>\}} and 
{\tt\string\edit3\{<text>\}}, respectively.

Similar to marking editing changes with the {\tt\string\edit} options there
are also the {\tt\string\authorcomments1\{<text>\}}, 
{\tt\string\authorcomments2\{<text>\}} and
{\tt\string\authorcomments3\{<text>\}} commands.  These produce the same
bold red, italic blue and underlined purple text but when the
{\tt\string\turnoffediting} command is present the {\tt\string<text>}
material does not appear in the manuscript.  Authors can use these commands
to mark up text that they are not sure should appear in the final
manuscript or as a way to communicate comments between co-authors when
writing the article.

\section{Software and third party data repository citations} \label{sec:cite}

The AAS Journals would like to encourage authors to change software and
third party data repository references from the current standard of a
footnote to a first class citation in the bibliography.  As a bibliographic
citation these important references will be more easily captured and credit
will be given to the appropriate people.

The first step to making this happen is to have the data or software in
a long term repository that has made these items available via a persistent
identifier like a Digital Object Identifier (DOI).  A list of repositories
that satisfy this criteria plus each one's pros and cons are given at \break
\url{https://github.com/AASJournals/Tutorials/tree/master/Repositories}.

In the bibliography the format for data or code follows this format: \\

\noindent author year, title, version, publisher, prefix:identifier\\

\citet{2015ApJ...805...23C} provides a example of how the citation in the
article references the external code at
\doi{10.5281/zenodo.15991}.  Unfortunately, bibtex does
not have specific bibtex entries for these types of references so the
``@misc'' type should be used.  The Repository tutorial explains how to
code the ``@misc'' type correctly.  The most recent aasjournal.bst file,
available with \aastex\ v6, will output bibtex ``@misc'' type properly.

%% If you wish to include an acknowledgments section in your paper,
%% separate it off from the body of the text using the \acknowledgments
%% command.
\acknowledgments

We thank all the people that have made this AASTeX what it is today.  This
includes but not limited to Bob Hanisch, Chris Biemesderfer, Lee Brotzman,
Pierre Landau, Arthur Ogawa, Maxim Markevitch, Alexey Vikhlinin and Amy
Hendrickson. Also special thanks to David Hogg and Daniel Foreman-Mackey
for the new "modern" style design. Considerable help was provided via bug
reports and hacks from numerous people including Patricio Cubillos, Alex
Drlica-Wagner, Sean Lake, Michele Bannister, Peter Williams, and Jonathan
Gagne.

%% To help institutions obtain information on the effectiveness of their 
%% telescopes the AAS Journals has created a group of keywords for telescope 
%% facilities.
%
%% Following the acknowledgments section, use the following syntax and the
%% \facility{} or \facilities{} macros to list the keywords of facilities used 
%% in the research for the paper.  Each keyword is check against the master 
%% list during copy editing.  Individual instruments can be provided in 
%% parentheses, after the keyword, but they are not verified.

\vspace{5mm}
\facilities{HST(STIS), Swift(XRT and UVOT), AAVSO, CTIO:1.3m,
CTIO:1.5m,CXO}

%% Similar to \facility{}, there is the optional \software command to allow 
%% authors a place to specify which programs were used during the creation of 
%% the manusscript. Authors should list each code and include either a
%% citation or url to the code inside ()s when available.

\software{astropy \citep{2013A&A...558A..33A},  
          Cloudy \citep{2013RMxAA..49..137F}, 
          SExtractor \citep{1996A&AS..117..393B}
          }

%% Appendix material should be preceded with a single \appendix command.
%% There should be a \section command for each appendix. Mark appendix
%% subsections with the same markup you use in the main body of the paper.

%% Each Appendix (indicated with \section) will be lettered A, B, C, etc.
%% The equation counter will reset when it encounters the \appendix
%% command and will number appendix equations (A1), (A2), etc. The
%% Figure and Table counter will not reset.

\appendix

\section{Appendix information}

Appendices can be broken into separate sections just like in the main text.
The only difference is that each appendix section is indexed by a letter
(A, B, C, etc.) instead of a number.  Likewise numbered equations have
the section letter appended.  Here is an equation as an example.

\begin{equation}
I = \frac{1}{1 + d_{1}^{P (1 + d_{2} )}}
\end{equation}

Appendix tables and figures should not be numbered like equations. Instead
they should continue the sequence from the main article body.

\section{Author publication charges} \label{sec:pubcharge}

Finally some information about the AAS Journal's publication charges.
In April 2011 the traditional way of calculating author charges based on 
the number of printed pages was changed.  The reason for the change
was due to a recognition of the growing number of article items that could not 
be represented in print. Now author charges are determined by a number of
digital ``quanta''.  A single quantum is 350 words, one figure, one table,
and one enhanced digital item.  For the latter this includes machine readable
tables, figure sets, animations, and interactive figures.  The current cost
is \$27 per word quantum and \$30 for all other quantum type.

\section{Rotating tables} \label{sec:rotate}

The process of rotating tables into landscape mode is slightly different in
\aastex v6.2. Instead of the {\tt\string\rotate} command, a new environment
has been created to handle this task. To place a single page table in a
landscape mode start the table portion with
{\tt\string\begin\{rotatetable\}} and end with
{\tt\string\end\{rotatetable\}}.

Tables that exceed a print page take a slightly different environment since
both rotation and long table printing are required. In these cases start
with {\tt\string\begin\{longrotatetable\}} and end with
{\tt\string\end\{longrotatetable\}}. Table \ref{chartable} is an
example of a multi-page, rotated table.

\begin{longrotatetable}
\begin{deluxetable*}{lllrrrrrrll}
\tablecaption{Observable Characteristics of 
Galactic/Magellanic Cloud novae with X-ray observations\label{chartable}}
\tablewidth{700pt}
\tabletypesize{\scriptsize}
\tablehead{
\colhead{Name} & \colhead{V$_{max}$} & 
\colhead{Date} & \colhead{t$_2$} & 
\colhead{FWHM} & \colhead{E(B-V)} & 
\colhead{N$_H$} & \colhead{Period} & 
\colhead{D} & \colhead{Dust?} & \colhead{RN?} \\ 
\colhead{} & \colhead{(mag)} & \colhead{(JD)} & \colhead{(d)} & 
\colhead{(km s$^{-1}$)} & \colhead{(mag)} & \colhead{(cm$^{-2}$)} &
\colhead{(d)} & \colhead{(kpc)} & \colhead{} & \colhead{}
} 
\startdata
CI Aql & 8.83 (1) & 2451665.5 (1) & 32 (2) & 2300 (3) & 0.8$\pm0.2$ (4) & 1.2e+22 & 0.62 (4) & 6.25$\pm5$ (4) & N & Y \\
{\bf CSS081007} & \nodata & 2454596.5 & \nodata & \nodata & 0.146 & 1.1e+21 & 1.77 (5) & 4.45$\pm1.95$ (6) & \nodata & \nodata \\
GQ Mus & 7.2 (7) & 2445352.5 (7) & 18 (7) & 1000 (8) & 0.45 (9) & 3.8e+21  & 0.059375 (10) & 4.8$\pm1$ (9) & N (7) & \nodata \\
IM Nor & 7.84 (11) & 2452289 (2) & 50 (2) & 1150 (12) & 0.8$\pm0.2$ (4) & 8e+21 & 0.102 (13) & 4.25$\pm3.4$ (4) & N & Y \\
{\bf KT Eri} & 5.42 (14) & 2455150.17 (14) & 6.6 (14) & 3000 (15) & 0.08 (15) & 5.5e+20 & \nodata & 6.5 (15) & N & M \\
{\bf LMC 1995} & 10.7 (16) & 2449778.5 (16) & 15$\pm2$ (17) & \nodata & 0.15 (203) & 7.8e+20  & \nodata & 50 & \nodata & \nodata \\
LMC 2000 & 11.45 (18) & 2451737.5 (18) & 9$\pm2$ (19) & 1700 (20) & 0.15 (203) & 7.8e+20  & \nodata & 50 & \nodata & \nodata \\
{\bf LMC 2005} & 11.5 (21) & 2453700.5 (21) & 63 (22) & 900 (23) & 0.15 (203) & 1e+21 & \nodata & 50  & M (24) & \nodata \\
{\bf LMC 2009a} & 10.6 (25) & 2454867.5 (25) & 4$\pm1$  & 3900 (25) & 0.15 (203)  & 5.7e+20 & 1.19 (26) & 50 & N & Y \\
{\bf SMC 2005} & 10.4 (27) & 2453588.5 (27) & \nodata & 3200 (28) & \nodata & 5e+20  & \nodata & 61 & \nodata & \nodata \\
{\bf QY Mus} & 8.1 (29) & 2454739.90 (29) & 60:  & \nodata & 0.71 (30) & 4.2e+21  & \nodata & \nodata & M & \nodata \\
{\bf RS Oph} & 4.5 (31) & 2453779.44 (14) & 7.9 (14) & 3930 (31) & 0.73 (32) & 2.25e+21 & 456 (33) & 1.6$\pm0.3$ (33) & N (34) & Y \\
{\bf U Sco} & 8.05 (35) & 2455224.94 (35) & 1.2 (36) & 7600 (37) & 0.2$\pm0.1$ (4) & 1.2e+21 & 1.23056 (36) & 12$\pm2$ (4) & N & Y \\
{\bf V1047 Cen} & 8.5 (38) & 2453614.5 (39) & 6 (40) & 840 (38) & \nodata & 1.4e+22  & \nodata & \nodata & \nodata & \nodata \\
{\bf V1065 Cen} & 8.2 (41) & 2454123.5 (41) & 11 (42) & 2700 (43) & 0.5$\pm0.1$ (42) & 3.75e+21 & \nodata & 9.05$\pm2.8$ (42) & Y (42) & \nodata \\
V1187 Sco & 7.4 (44) & 2453220.5 (44) & 7: (45) & 3000 (44) & 1.56 (44) & 8.0e+21 & \nodata & 4.9$\pm0.5$ (44) & N & \nodata \\
{\bf V1188 Sco} & 8.7 (46) & 2453577.5 (46) & 7 (40) & 1730 (47) & \nodata & 5.0e+21  & \nodata & 7.5 (39) & \nodata & \nodata \\
{\bf V1213 Cen} & 8.53 (48) & 2454959.5 (48) & 11$\pm2$ (49) & 2300 (50) & 2.07 (30) & 1.0e+22 & \nodata & \nodata & \nodata & \nodata \\
{\bf V1280 Sco} & 3.79 (51) & 2454147.65 (14) & 21 (52) & 640 (53) & 0.36 (54) & 1.6e+21  & \nodata & 1.6$\pm0.4$ (54) & Y (54) & \nodata \\
{\bf V1281 Sco} & 8.8 (55) & 2454152.21 (55) & 15:& 1800 (56) & 0.7 (57) & 3.2e+21 & \nodata & \nodata & N & \nodata \\
{\bf V1309 Sco} & 7.1 (58) & 2454714.5 (58) & 23$\pm2$ (59) & 670 (60) & 1.2 (30) & 4.0e+21 & \nodata & \nodata & \nodata & \nodata \\
{\bf V1494 Aql} & 3.8 (61) & 2451515.5 (61) & 6.6$\pm0.5$ (61) & 1200 (62) & 0.6 (63) & 3.6e+21  & 0.13467 (64) & 1.6$\pm0.1$ (63) & N & \nodata \\
{\bf V1663 Aql} & 10.5 (65) & 2453531.5 (65) & 17 (66) & 1900 (67) & 2: (68) & 1.6e+22  & \nodata & 8.9$\pm3.6$ (69) & N & \nodata \\
V1974 Cyg & 4.3 (70) & 2448654.5 (70) & 17 (71) & 2000 (19) & 0.36$\pm0.04$ (71) & 2.7e+21  & 0.081263 (70) & 1.8$\pm0.1$ (72) & N & \nodata \\
{\bf V2361 Cyg} & 9.3 (73) & 2453412.5 (73) & 6 (40) & 3200 (74) & 1.2: (75) & 7.0e+21 & \nodata & \nodata & Y (40) & \nodata \\
{\bf V2362 Cyg} & 7.8 (76) & 2453831.5 (76) & 9 (77) & 1850 (78) & 0.575$\pm0.015$ (79) & 4.4e+21  & 0.06577 (80) & 7.75$\pm3$ (77) & Y (81) & \nodata \\
{\bf V2467 Cyg} & 6.7 (82) & 2454176.27 (82) & 7 (83) & 950 (82) & 1.5 (84) & 1.4e+22  & 0.159 (85) & 3.1$\pm0.5$ (86) & M (87) & \nodata \\
{\bf V2468 Cyg} & 7.4 (88) & 2454534.2 (88) & 10: & 1000 (88) & 0.77 (89) & 1.0e+22  & 0.242 (90) & \nodata & N & \nodata \\
{\bf V2491 Cyg} & 7.54 (91) & 2454567.86 (91) & 4.6 (92) & 4860 (93) & 0.43 (94) & 4.7e+21  & 0.09580: (95) & 10.5 (96) & N & M \\
V2487 Oph & 9.5 (97) & 2450979.5 (97) & 6.3 (98) & 10000 (98) & 0.38$\pm0.08$ (98) & 2.0e+21 & \nodata & 27.5$\pm3$ (99) & N (100) & Y (101) \\
{\bf V2540 Oph} & 8.5 (102) & 2452295.5 (102) & \nodata & \nodata & \nodata & 2.3e+21 & 0.284781 (103) & 5.2$\pm0.8$ (103) & N & \nodata \\
V2575 Oph & 11.1 (104) & 2453778.8 (104) & 20: & 560 (104) & 1.4 (105) & 3.3e+21 & \nodata & \nodata & N (105) & \nodata \\
{\bf V2576 Oph} & 9.2 (106) & 2453832.5 (106) & 8: & 1470 (106) & 0.25 (107) & 2.6e+21  & \nodata & \nodata & N & \nodata \\
{\bf V2615 Oph} & 8.52 (108) & 2454187.5 (108) & 26.5 (108) & 800 (109) & 0.9 (108) & 3.1e+21  & \nodata & 3.7$\pm0.2$ (108) & Y (110) & \nodata \\
{\bf V2670 Oph} & 9.9 (111) & 2454613.11 (111) & 15: & 600 (112) & 1.3: (113) & 2.9e+21  & \nodata & \nodata & N (114) & \nodata \\
{\bf V2671 Oph} & 11.1 (115) & 2454617.5 (115) & 8: & 1210 (116) & 2.0 (117) & 3.3e+21  & \nodata & \nodata & M (117) & \nodata \\
{\bf V2672 Oph} & 10.0 (118) & 2455060.02 (118) & 2.3 (119) & 8000 (118) & 1.6$\pm0.1$ (119) & 4.0e+21  & \nodata & 19$\pm2$ (119) & \nodata & M \\
V351 Pup & 6.5 (120) & 2448617.5 (120) & 16 (121) & \nodata & 0.72$\pm0.1$ (122) & 6.2e+21 & 0.1182 (123) & 2.7$\pm0.7$ (122) & N & \nodata \\
{\bf V382 Nor} & 8.9 (124) & 2453447.5 (124) & 12 (40) & 1850 (23) & \nodata & 1.7e+22 & \nodata & \nodata & \nodata & \nodata \\
V382 Vel & 2.85 (125) & 2451320.5 (125) & 4.5 (126) & 2400 (126) & 0.05: (126) & 3.4e+21  & 0.146126 (127) & 1.68$\pm0.3$ (126) & N & \nodata \\
{\bf V407 Cyg} & 6.8 (128) & 2455266.314 (128) & 5.9 (129) & 2760 (129) & 0.5$\pm0.05$ (130) & 8.8e+21 & 15595 (131) & 2.7 (131) & \nodata & Y \\
{\bf V458 Vul} & 8.24 (132) & 2454322.39 (132) & 7 (133) & 1750 (134) & 0.6 (135) & 3.6e+21 & 0.06812255 (136) & 8.5$\pm1.8$ (133) & N (135) & \nodata \\
{\bf V459 Vul} & 7.57 (137) & 2454461.5 (137) & 18 (138) & 910 (139) & 1.0 (140) & 5.5e+21  & \nodata & 3.65$\pm1.35$ (138) & Y (140) & \nodata \\
V4633 Sgr & 7.8 (141) & 2450895.5 (141) & 19$\pm3$ (142) & 1700 (143) & 0.21 (142) & 1.4e+21  & 0.125576 (144) & 8.9$\pm2.5$ (142) & N & \nodata \\
{\bf V4643 Sgr} & 8.07 (145) & 2451965.867 (145) & 4.8 (146) & 4700 (147) & 1.67 (148) & 1.4e+22 & \nodata & 3 (148) & N & \nodata \\
{\bf V4743 Sgr} & 5.0 (149) & 2452537.5 (149) & 9 (150) & 2400 (149) & 0.25 (151) & 1.2e+21 & 0.281 (152) & 3.9$\pm0.3$ (151) & N & \nodata \\
{\bf V4745 Sgr} & 7.41 (153) & 2452747.5 (153) & 8.6 (154) & 1600 (155) & 0.1 (154) & 9.0e+20  & 0.20782 (156) & 14$\pm5$ (154) & \nodata & \nodata \\
{\bf V476 Sct} & 10.3 (157) & 2453643.5 (157) & 15 (158) & \nodata & 1.9 (158) & 1.2e+22  & \nodata & 4$\pm1$ (158) & M (159) & \nodata \\
{\bf V477 Sct} & 9.8 (160) & 2453655.5 (160) & 3 (160) & 2900 (161) & 1.2: (162) & 4e+21  & \nodata & \nodata & M (163) & \nodata \\
{\bf V5114 Sgr} & 8.38 (164) & 2453081.5 (164) & 11 (165) & 2000 (23) & \nodata & 1.5e+21  & \nodata & 7.7$\pm0.7$ (165) & N (166) & \nodata \\
{\bf V5115 Sgr} & 7.7 (167) & 2453459.5 (167) & 7 (40) & 1300 (168) & 0.53 (169) & 2.3e+21  & \nodata & \nodata & N (169) & \nodata \\
{\bf V5116 Sgr} & 8.15 (170) & 2453556.91 (170) & 6.5 (171) & 970 (172) & 0.25 (173) & 1.5e+21 & 0.1238 (171) & 11$\pm3$ (173) & N (174) & \nodata \\
{\bf V5558 Sgr} & 6.53 (175) & 2454291.5 (175) & 125 (176) & 1000 (177) & 0.80 (178) & 1.6e+22  & \nodata & 1.3$\pm0.3$ (176) & N (179) & \nodata \\
{\bf V5579 Sgr} & 5.56 (180) & 2454579.62 (180) & 7: & 1500 (23) & 1.2 (181) & 3.3e+21 & \nodata & \nodata & Y (181) & \nodata \\
{\bf V5583 Sgr} & 7.43 (182) & 2455051.07 (182) & 5: & 2300 (182) & 0.39 (30) & 2.0e+21 & \nodata & 10.5 & \nodata & \nodata \\
{\bf V574 Pup} & 6.93 (183) & 2453332.22 (183) & 13 (184) & 2800 (184) & 0.5$\pm0.1$  & 6.2e+21 & \nodata & 6.5$\pm1$  & M (185) & \nodata \\
{\bf V597 Pup} & 7.0 (186) & 2454418.75 (186) & 3: & 1800 (187) & 0.3 (188) & 5.0e+21  & 0.11119 (189) & \nodata & N (188) & \nodata \\
{\bf V598 Pup} & 3.46 (14) & 2454257.79 (14) & 9$\pm1$ (190) & \nodata & 0.16 (190) & 1.4e+21 & \nodata & 2.95$\pm0.8$ (190) & \nodata & \nodata \\
{\bf V679 Car} & 7.55 (191) & 2454797.77 (191) & 20: & \nodata & \nodata & 1.3e+22  & \nodata & \nodata & \nodata & \nodata \\
{\bf V723 Cas} & 7.1 (192) & 2450069.0 (192) & 263 (2) & 600 (193) & 0.5 (194) & 2.35e+21  & 0.69 (195) & 3.86$\pm0.23$ (196) & N & \nodata \\
V838 Her & 5 (197) & 2448340.5 (197) & 2 (198) & \nodata & 0.5$\pm0.1$ (198) & 2.6e+21  & 0.2975 (199) & 3$\pm1$ (198) & Y (200) & \nodata \\
{\bf XMMSL1 J06} & 12 (201) & 2453643.5 (202) & 8$\pm2$ (202) & \nodata & 0.15 (203) & 8.7e+20 & \nodata & 50 & \nodata & \nodata \\
\enddata
\end{deluxetable*}
\end{longrotatetable}

A handy "cheat sheat" that provides the necessary LaTeX to produce 17 
different types of tables is available at \url{http://journals.aas.org/authors/aastex/aasguide.html#table_cheat_sheet}.

%% The reference list follows the main body and any appendices.
%% Use LaTeX's thebibliography environment to mark up your reference list.
%% Note \begin{thebibliography} is followed by an empty set of
%% curly braces.  If you forget this, LaTeX will generate the error
%% "Perhaps a missing \item?".
%%
%% thebibliography produces citations in the text using \bibitem-\cite
%% cross-referencing. Each reference is preceded by a
%% \bibitem command that defines in curly braces the KEY that corresponds
%% to the KEY in the \cite commands (see the first section above).
%% Make sure that you provide a unique KEY for every \bibitem or else the
%% paper will not LaTeX. The square brackets should contain
%% the citation text that LaTeX will insert in
%% place of the \cite commands.

%% We have used macros to produce journal name abbreviations.
%% \aastex provides a number of these for the more frequently-cited journals.
%% See the Author Guide for a list of them.

%% Note that the style of the \bibitem labels (in []) is slightly
%% different from previous examples.  The natbib system solves a host
%% of citation expression problems, but it is necessary to clearly
%% delimit the year from the author name used in the citation.
%% See the natbib documentation for more details and options.

\begin{thebibliography}{}

\bibitem[Astropy Collaboration et al.(2013)]{2013A&A...558A..33A} Astropy Collaboration, Robitaille, T.~P., Tollerud, E.~J., et al.\ 2013, \aap, 558, A33 
\bibitem[Bertin \& Arnouts(1996)]{1996A&AS..117..393B} Bertin, E., \& Arnouts, S.\ 1996, \aaps, 117, 393 
\bibitem[Corrales(2015)]{2015ApJ...805...23C} Corrales, L.\ 2015, \apj, 805, 23
\bibitem[Ferland et al.(2013)]{2013RMxAA..49..137F} Ferland, G.~J., Porter, R.~L., van Hoof, P.~A.~M., et al.\ 2013, \rmxaa, 49, 137
\bibitem[Hanisch \& Biemesderfer(1989)]{1989BAAS...21..780H} Hanisch, R.~J., \& Biemesderfer, C.~D.\ 1989, \baas, 21, 780 
\bibitem[Lamport(1994)]{lamport94} Lamport, L. 1994, LaTeX: A Document Preparation System, 2nd Edition (Boston, Addison-Wesley Professional)
\bibitem[Schwarz et al.(2011)]{2011ApJS..197...31S} Schwarz, G.~J., Ness, J.-U., Osborne, J.~P., et al.\ 2011, \apjs, 197, 31  
\bibitem[Vogt et al.(2014)]{2014ApJ...793..127V} Vogt, F.~P.~A., Dopita, M.~A., Kewley, L.~J., et al.\ 2014, \apj, 793, 127  

\end{thebibliography}

%% This command is needed to show the entire author+affilation list when
%% the collaboration and author truncation commands are used.  It has to
%% go at the end of the manuscript.
%\allauthors

%% Include this line if you are using the \added, \replaced, \deleted
%% commands to see a summary list of all changes at the end of the article.
%\listofchanges

\end{document}

% End of file `sample62.tex'.
