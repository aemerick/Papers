\chapter[Conclusion]{Conclusion}
\label{ch:conclusion}

%Summarize key results

This \dissertation has focused on building an understanding of galactic chemical evolution with the point of view that the same feedback processes that regulate star formation and drive galactic winds in galaxies also play a fundamental role in their metal enrichment. It is understood that stellar feedback plays an important role in driving metals out into the CGM and beyond in galaxies, but the exact coupling of metals to galactic winds is not well understood. This is in part because how metals mix within a multi-phase ISM to eventually be swept up in galactic outflows or mix within cold, star forming gas is not well constrained. Even more, how metals from the full range of nucleosynthetic sources, including AGB winds, stellar winds, core collapse and Type Ia supernovae, and NS-NS mergers, participate in these processes -- mixing and galactic winds -- differently has only recently been examined in efforts to better understand results from the ever increasing availability of high-quality stellar abundance measurements of stars in our own Milky Way and in nearby dwarf galaxies. In order to build a complete picture of metal evolution in the Universe, we have developed a new model for star formation, stellar feedback, and chemical enrichment to follow these processes in simulations of isolated dwarf galaxies. Using this model, we have made critical advancements in understanding both how stellar feedback governs the evolution of low mass dwarf galaxies and how individual metals are distributed in (and beyond) these galaxies over time. We summarize these results below, and conclude with a discussion of future research.


\section{A New Model for Galactic Chemical Enrichment}
\label{conclusion:sec:ch1}

In order to address these outstanding issues in galactic evolution, we developed a new model for galactic chemical evolution that -- for the first time in galaxy-scale simulations -- follows star particles as individual stars along with their detailed feedback and metal yields. As detailed in Chapter~\ref{ch:chapter2} we use the AMR hydrodynamics code \textsc{Enzo} to produce a set of simulations using this model of an idealized, isolated low-mass dwarf galaxy ($M_{\rm gas} \sim 10^{6}$~M$_{\odot}$, $M_{\rm halo} \sim 3 \times 10^{9}$~M$_{\odot}$) embedded in a static dark matter potential. Using a detailed model for gas chemistry and radiative cooling, we follow the evolution of this galaxy and its multi-phase ISM as it forms stars, depositing individual star particles sampled over 1 to 100 M$_{\odot}$. We follow the feedback processes from each of these stars, including the AGB winds from low mass stars, Lyman-Werner radiation, FUV radiation (and photoelectric heating), and ionizing radiation from massive stars, and both core collapse and Type Ia supernovae. We track the metal yields for each of these stars, following 15 individual metal species as they mix within the ISM and imprint their abundances onto the sites of future star formation or are ejected from the galaxy altogether.

We find that this model -- which contains no independently tuned free parameters -- is sufficient to generate a multi-phase ISM with star formation, gas scale-height, and outflow properties that match expectations from observations. The global metal retention fraction for this galaxy is low ($\lesssim$ 5\%), with typical outflow velocities of $\sim$100~km~s$^{-1}$, and up to 1000~km~s$^{-1}$. We find that stellar radiation feedback is important, but that the ISRF is highly variable -- driven by stochastic IMF sampling in this low SFR galaxy. The mass fraction of the ISM is dominated by the cold (T$\sim$100~K) and warm-neutral (T$\sim$1000~K) components, while the warm-ionized and hot phases dominated the volume fraction. The WIM and HIM in the galaxy also exhibit significant time variation due to stochastic IMF sampling and a bursty SFH. Finally, we find that H$_2$ comprises $\lesssim$~5\% of the galaxy by mass, formed primarily through H$-$ gas-phase reactions.

In ongoing work we are extending these methods by implementing them in cosmological simulations of UFD galaxy evolution. Our detailed model tracking the feedback and yields from individual stars will build substantially upon recent works investigating metal enrichment at high redshift and in UFDs and their progenitors \citep[e.g.][]{Ritter2015,Jeon2017,Corlies2018,Wheeler2018,Agertz2019}. This work includes additional improvements to the simulations presented in this \Dissertation, including a model for Pop III star formation and chemical enrichment adapted from \cite{Wise2012a}, a new IMF sampling scheme that allows us to follow very low mass stars ($M_* \lesssim 2$~M$_{\odot}$) -- which have no significant feedback or metal enrichment, but are important long-lived tracers of galactic chemical enrichment -- aggregated together into a single particle during each star formation event, improvements to the numerical implementation to ease memory consumption in these significantly larger simulations, and new metal tracer fields to follow the contribution of AGB, core collapse SNe, Type Ia SNe, and PopIII stars to the total metal content of each cell independently.

Finally, as part of this \dissertation we additionally produced a significant set of companion simulations at $2 \times$ lower resolution (3.6~pc instead of 1.8~pc) turning on and off each independent feedback process (ionizing radiation, radiation pressure, LW and FUV radiation, and supernovae). With these runs, we sought to build upon recent works examining the various effects of each stellar feedback process in driving the evolution of low mass dwarf galaxies \citep{Hu2016,Hu2017,Hu2018,Forbes2016}. In future work we will examine these completed runs in more detail to better understand the complex, interconnected role multi-channel stellar feedback plays in galactic evolution. With our simulations, we can place greater emphasis on how each feedback channel drives the mixing and ejection of individual metals within and beyond low mass dwarf galaxies.

\section{Stellar Radiation Feedback in Dwarf Galaxies}
\label{conclusion:sec:ch2}

Stellar radiation has recently been invoked as an important source of feedback -- in addition to supernovae -- that helps to regulate star formation and drive galactic winds in galaxies across the mass scale. This feedback channel acts immediately after the formation of massive stars and is thus an important source of pre-SN feedback. In addition to destroying cold gas in star forming regions, this acts to dramatically reduce the typical ISM densities in which SNe explode, increasing their impact. Modeling stellar radiation in galaxies in detail can be computationally expensive, with a variety of approximate methods available to reduce this expense. In our simulations we track the ionizing radiation from massive stars accurately using an adaptive ray-tracing radiative transfer method with sufficient resolution to capture the HII regions from individual stars. In Chapter~\ref{ch:chapter2} we demonstrate that stellar ionizing radiation dramatically reduces the star formation rate and generally increases the effectiveness of galactic winds in our low-mass dwarf galaxy simulations.

In addition, we explore the mechanism by which stellar radiation acts to regulate star formation and drive outflows by comparing our fiducial simulations to simulations that only account for localized ($< 20$~pc) ionization around massive stars. This second set of runs mimics the behavior of various approximations for stellar radiation feedback which do not follow radiation propagation in detail, but rather deposit energy / ionize gas in the immediate vicinity of newly formed stars. We find that star formation is regulated by ionizing radiation locally -- destroying cold gas around young, massive stars that would otherwise have formed stars -- as the $< 20$~pc restricted run (initially) shares a very similar SFR to our fiducial run. However, we find that these simulations exhibit dramatically different outflow properties, which drive long-term differences in galactic properties including SFR and metal retention. Stellar radiation at larger distance scales is important for carving out diffuse channels of gas in the ISM and into the CGM of our dwarf galaxy through which SNe can effectively blow out mass and metals from the ISM. This suggests that -- at least in low mass dwarf galaxies -- localized approximations of radiation feedback are likely insufficient for capturing the full effects of stellar radiation.

%sAlthough we include the effects of photoelectric heating, Lyman-Werner radiation, and radiation pressure, we have not investigated their individual importance in this work. Recent works have demonstrated the importance of photoelectric heating in dwarf galaxies \citep{Hu2016,Forbes2016},

\section{Mixing and Ejection of Individual Metals}
\label{conclusion:sec:ch3}

As argued in \cite{KrumholzTing2018}, differences in how metals are released into the ISM (due to the variation in energetics between their nucleosynthetic sites) should drive discernible differences in their mixing behavior in the ISM. We demonstrate these differences for the first time in hydrodynamics simulations in Chapter~\ref{ch:chapter3} by examining the evolution of each of the 15 metals species tracked in our simulations. Although we are not the first to follow a large number of individual metal species from different nucleosynthetic channels in a galaxy-scale simulation, that we follow individual stars and follow the feedback for each independently has allowed us to capture these effects. Notably, we find that -- for low mass dwarf galaxies -- metals released in AGB winds are retained in the galaxy's ISM at a much higher fraction than metals released in either core collapse or Type Ia supernovae. This implies that if one were to construct a chemical evolution model of a low mass dwarf galaxy and adopt the high metal ejection fractions as seen in observations\footnote{Which typically trace O, an element released predominately in core collapse supernovae.} \citep{Kirby2011-metals,McQuinn2015}, one could \textit{underestimate} the mass of AGB-wind dominated metals (such as Sr and Ba, and possibly C and N) in the ISM by a factor of $\sim$ 4-5. This effect is important in matching observations of stellar abundances in low mass dwarf galaxies, but how this ratio depends on galaxy properties (e.g. $M_*$, $M_{\rm halo}$, $\Sigma_{\rm SFR}$, $\Sigma_{\rm gas}$) is uncertain and warrants future research.

In agreement with predictions from \cite{KrumholzTing2018}, we find that the abundances of the AGB-wind metals are more inhomogeneous than the abundances of metals from supernovae. This spread can be significant, over 1~dex for the AGB-wind elements. We predict then that -- at least in the low metallicity environments of low mass dwarf galaxies -- the observed stellar abundance distributions for AGB-wind elements should be characterized by markedly greater spreads than abundance distributions for SNe elements (e.g. O, Mg). Indeed, this is observed in the stellar abundance patterns of dwarf galaxies in the Local Group \citep{Suda2017}, but it remains to be seen if this is a conclusive signature of these mixing differences or perhaps is due to larger differences in star-to-star metal yields for these elements. Finally, we were able to characterize the metal mass fraction PDFs for each element across each phase of the ISM, and found that they generically well-fit by a combined log-normal PDF with a power-law tail towards higher metal mass fractions. We hope that this characterization lends itself well to developing a physically-motivated semi-analytic prescription for stellar abundance spreads to be used in semi-analytic (and one-zone) models of galactic chemical evolution.

\section{Mixing Behavior of Single-Sources}
\label{conclusion:sec:ch4}

To expand upon the results of Chapter~\ref{ch:chapter3} and understand how metals are distributed throughout the ISM on a source-by-source level, we carry out a series of ``mixing experiments'' in Chapter~\ref{ch:chapter4} to test how metal mixing and metal ejection behaviors vary with the injection energy of the source. We confirm that lower energy events (e.g. AGB-winds) are retained at a higher fraction in the ISM and mix poorly compared to higher energy events (e.g. SNe) \textit{on average}. We demonstrate this by following the evolution of metals of individual sources at injection energies of 10$^{46}$~erg (AGB winds), 10$^{49}$~erg and 10$^{50}$~erg (NS-NS mergers), 10$^{51}$~erg (SNe and possibly NS-NS mergers), and 10$^{52}$~erg (hypernovae). However, we find that the behavior of any single source exhibits a substantial amount of variance -- particularly at lower energies -- as it is clear that the injection energy is not the only dominant factor in determining mixing behavior. We investigate other possible factors, including radial position, global galaxy SFR, and local ISM density, and find that the global galaxy SFR has the greatest effect on the mixing and ejection behavior of each metal, particularly for lower energy events. While these results suggest that these behavioral differences can be accounted for in an average sense, the full picture is subject to substantial stochasticity based purely on the randomness with which both when and where different events occur in a galaxy. Abundances associated with rare enrichment sources -- such as the r-process enrichment candidates -- which may only occur once (if at all) in the lowest mass dwarf galaxies should therefore exhibit the most significant galaxy-to-galaxy abundance variations for galaxies at fixed stellar mass.

\section{Future Work}
\label{conclusion:sec:future}

One of the primary goals of this \dissertation was to develop a physical understanding of the processes that contribute generally to galactic chemical evolution and specifically to metal mixing and ejection from galaxies -- a process that was previously not well understood. With a novel model for star formation and stellar feedback we have made substantial contributions to this understanding, but continued work is necessary. Specifically, the conclusions reached in this \dissertation have all been made in the context of the evolution of a low-mass dwarf galaxy with idealized initial conditions and limited cosmological context. Although we can speculate, it remains to be seen how these results can vary over cosmological timescales for a single galaxy as well as how they vary across the galaxy mass scale. Besides conducting a suite of simulations sampling a variety of galaxy parameters, which could prove too computationally expensive at higher galaxy masses, semi-analytic and one-zone models provide the ideal testing ground for identifying what regimes may be sensitive to the metal mixing behaviors identified in this work. The ultimate goal would be to use a suite of hydrodynamics simulations to construct physically motivated prescriptions for use in these one-zone models. Until then, however, future work can make significant strides with the results presented here alone.

To aid in this development, we have built a one-zone model, \textsc{OneZ}, that has been designed to match the physical model used in our hydrodynamics simulations.\footnote{https://github.com/aemerick/OneZ}. This model uses an equivalent star formation recipe as our hydrodynamics simulations, stochastically forming individual star particles and following their yields. Given an input SFH and $f_{\rm ej}$ for each metal from our hydrodynamics simulations, we can match the mean metal abundance evolution from these simulations and can explore how varying $f_{\rm ej}$ (element-by-element) affects the resultant stellar abundance patterns. In future work, we plan to implement a two-phase model like that in \cite{SchonrichWeinberg2019} that includes parameters for how each metal first enters the hot and cold phases, how each metal mixes between the two phases, and a stochastic mixing-zone method as used in \cite{Cescutti2008}. With these additions, we will: 1) explore how these physics manifest themselves in galaxies with a wide range of star formation and halo properties, 2) use these insights to begin to match observed stellar abundance patterns in low-mass dwarf galaxies, 3) make predictions for expected multi-element abundance spreads in dwarf galaxies for comparison to future observations in upcoming surveys, and 4) better constrain the relative importance of stochastic sampling of the IMF and inhomogeneous mixing in setting these abundance spreads. In addition, we can more easily use this model to explore a wide variety of nucleosynthetic yield tables to better constrain the -- highly uncertain -- yields themselves. These insights can be used to improve more complex chemical evolution models that account for the cosmological and hierarchical evolution of galaxies over time. This is a particularly exciting tool that can be used to better constrain the sources of r-process enrichment in low mass dwarf galaxies.

% Write simulation "next steps" here 1) cosmological, 2) feedback variance
