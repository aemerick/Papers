\chapter[Introduction]{Introduction}
\label{ch:intro}
\vspace{-16pt} \begin{chapquote}{----} \singlespacing Some cool quote (optional)
\end{chapquote} \vspace{-8pt}
\noindent\makebox[\linewidth]{\rule{0.5\textwidth}{0.5pt}} \vspace{1pt}


%
% \beq
% \frac{\mathrm{d}}{\mathrm{d}t} \int_S \mathbf{B} \cdot \mathrm{d}S = 0,f
% \eeq

Text

\section{Ingredients of Galactic Evolution}

Very broadly give a picture of galactic evolution, focusing in on the physics needed to simulate galaxies (cooling, heating, chemistry, feedback). Touch on inflows, outflows, mixing, ISM physics, etc. Bring up CE and feedback obviously but reserve for next section(s). But maybe this can all go in the intro-intro

\section{Galactic Chemical Evolution}
\label{sec:CE}

\subsection{Nucleosynthetic Sources}

Cover (briefly) each enrichment source and dominant production elements. 

% maybe a section on our own galaxy

\subsection{Metals in Local Group Dwarfs}

\subsubsection{Metal Retention / Ejection}

\subsubsection{Stellar Abundances}

\subsubsection{Sources of Enrichment}

\subsection{Modelling Chemical Evolution}

\subsubsection{Hydrodynamics Simulations}

\subsubsection{Semi-analytic Models / One-zone models}

Definitely have a lot here. Motivate need for better understanding of mixing processes in galaxies. \citep{SchonrichWeinberg2019}


\section{Stellar Feedback}\label{sec:section1}

Overview

% maybe a section on each FB method? R

\subsection{Star Particles in Simulations}

%
%
%


\section{Structure of Dissertation}\label{sec:structure}

In Chapter \ref{ch:chapter1} I...

%In Chapter \ref{ch:chapter2}

We conclude in Chapter \ref{ch:conclusion} and briefly discuss future work as extensions of this Dissertation.
