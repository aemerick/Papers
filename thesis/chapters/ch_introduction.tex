\chapter[Introduction]{Introduction}
\label{ch:intro}
\vspace{-16pt} 
\begin{chapquote}{Isaac Asimov} \singlespacing The most exciting phrase to hear in science, the one that heralds discoveries, is not ‘Eureka!’ but ‘Now that’s funny…
\end{chapquote} \vspace{-8pt}
\begin{chapquote}{Brian W. Kernighan} \singlespacing Debugging is twice as hard as writing the code in the first place. Therefore, if you write the code as cleverly as possible, you are, by definition, not smart enough to debug it
\end{chapquote} \vspace{-8pt}

\noindent\makebox[\linewidth]{\rule{0.5\textwidth}{0.5pt}} \vspace{1pt}

\newcommand{\code}{\textsc}

%
% \beq
% \frac{\mathrm{d}}{\mathrm{d}t} \int_S \mathbf{B} \cdot \mathrm{d}S = 0,f
% \eeq

Galaxies are amalgamations of gas and stars embedded in dark matter halos, formed over cosmic time as the products of the hierarchical assembly that follows from the collapse of primordial density fluctuations that arose after the Big Bang. The gravitational pull of dark matter controls the growth of structure in the Universe which, as is predicted from $\Lambda$CDM cosmology, TEXT HERE. While, broadly speaking, the evolution of baryons is dominated by the gravitational pull of dark matter, the beautiful simplicity of the evolution of dark matter and $\Lambda$CDM is muddied by their existence. The complexities of hydrodynamics, magnetohydrodynamics, thermodynamics, chemistry, radiative processes, and nucleosynthesis -- or, ``astrophysics" for short -- drives these deviations and gives rise to the Universe that we observe and interact with and -- not least of all -- us. In spite of a concerted effort spanning nearly a century, we are far from a self-consistent theory of galaxy evolution \citep[see ][ for a recent review of outstanding problems in galactic evolution]{NaabOstriker2017}. It is understanding the rich set of physics that formed our own Galaxy and the countless galaxies scattered throughout the Universe that motivates this work. 

A substantial slice of modern astrophysics has been devoted towards understanding the chemical evolution -- the abundances of individual elements in space and time -- of the Universe. With the exception of H, He, and trace amounts of light elements, all of the elements in the Universe are produced in nuclear reactions in the cores of stars, massive star stellar winds, asymptotic giant branch (AGB) winds of low mass stars, and more exotic sites like nuetron star - neutron star mergers (NS-NS). See \cite{Nomoto2013}, \cite{Thielmann2017}, and \cite{Frebel2018} for recent reviews on this topic as it applies to studies of galactic and stellar evolution. Much work has been done in studying the global metallicity evolution of galaxies through the mass-metallicity relationship \citep[e.g.][]{Lequeux1979,Tremonti2004,Lee2006,Zahid2012}, the metallicity gradient in our own and nearby galaxies \citep[e.g.][]{}, and detailed stellar abundances in galaxies with spectroscopically resolved stellar populations \citep[e.g.][]{}. 

%Developing a complete model for galactic chemical evolution is a daunting task.
Here is a recipe for constructing a complete model of galactic chemical evolution as it is understood today. One begins first with the simple step of nailing down a complete model of stellar structure, stellar evolution, and learning from nuclear physicists the reaction rates of every nuclear reaction and lifetimes of individual isotopes. One must then perscribe a galaxy's connection to its large scale structure to understand the inflow of pristine, metal-free gas into the disk of the galaxy, and the accretion (via mergers) of stars formed previously in galaxies hosted by different dark matter halos. One can then worry about knowing where, how, and with what masses stars form in the first place from this gas, which requires understanding the hydrodynamic 
%and magnetohydrodynamic 
properties of the multi-scale, multi-phase turbulent interstellar medium (ISM). This in turn requires one to toss in a detailed understanding of the stellar feedback physics -- stellar radiation, stellar winds, and supernovae -- that helps to regulate star formation by destroying cold, star forming gas, driving the multi-phase structure of the ISM, and by driving outflows of gas into the circumgalactic medium (CGM) around galaxies. Once this is understood, one can then color in the gas in the galaxy over time with the individual metal yields of stellar populations released over their lifetime. With a complete understanding of stellar feedback and the ISM, one can then be sure that they completely capture the mixing of these metals in the ISM over time, and will produce a stellar population with accurate metal abundance ratios. In addition, one would then produce realistic galactic winds, removing the correct amount of metals from the galaxy and enriching the CGM and the intergalactic medium (IGM). Finally, one can layer this model ad nauseum on top of a $\Lambda CDM$ model for the cosmological evolution of many galaxies, reproducing all observed properties of galactic chemical evolution as a function of both mass and redshift..... Wait... I forgot about magnetic fields and cosmic rays...

It should be obvious now to the reader why there remains so much to be learned about galactic chemical evolution. Developing a complete model for galactic chemical evolution is a daunting task. Yet it is clear that this field offers an incredibly exciting test of an incredible range of physics.

The observational landscape today is primed for developing a more detailed understanding these processes. Recent observational campaigns such as SEGUE \citep{Yanny2009}, RAVE \citep{Kunder2017}, the Gaia-ESO survey \citep{Gaia}, APOGEE and APOGEE2 \citep{APOGEE2010,APOGEE}, GALAH \citep{GALAH,Buder2018}, as well as upcoming observations, such as the Local Volume Mapper as part of SDSS-V, have generated tremendous amounts of information on detailed stellar abundnaces and stellar kinematics in our Milky Way and nearby Local Group dwarf galaxies. One of the most powerful proposed uses of this enormous trove of data is chemical tagging \citep{Freeman2002}, whereby stellar populations are analyzed in chemical space and 6D phase space to identify co-eval and co-natal groups of stars. This process of galactic archeaology aims to break down and identify each distinct stellar component of our Galaxy, explaining the process of their formation and evolution. Substantial work has been made recently to determine the efficacy of this approach \citep[e.g.][]{Armillotta2018}, yet the physical processes that give rise to stellar abundances as we observe them today are still uncertain. The excitement at the availability of this data does not even mention 

Mention study of dwarf galaxies instead of MW first as way to simulate physics in detail and build up undserstanding to more massive regimes. Lead into physics that is important below.

\section{Ingredients of Galactic Chemical Evolution}
\label{intro:sec:ingredients}

We give a primer on how to model galactic evolution in hydrodynamics simulations. We focus on the physics needed to simulate galaxies (cooling, heating, chemistry, star formation, feedback, etc.) as it pertains to the aspects of galactic chemical evolution examined in this dissertation. This is meant to be a broad overview of the physical processes modelled in such simulations, focusing on aspects of their implementation rather than the derivation of theory behind such processes. While important, the latter is better reserved for a much more detailed work (i.e. textbook), and focusing on the former gives a much more accurate represenation of the actual work done for this dissertation. This discussion includes a very brief overview of hydrodynamics and the numerical methods used in this work in Section~\ref{intro:sec:hydrodynamics}, nucleosynthesis and stellar evolution in Section~\ref{intro:sec:nucleosynthesis}, radiative cooling and chemistry (actual chemistry) in Section~\ref{intro:sec:cooling}, star formation and star particles in Section~\ref{intro:sec:sf} and Section~\ref{intro:sec:stars}, and stellar feedback in Section~\ref{intro:sec:feedback}.

But first, we note that Astronomy jargon is full of idiosyncrisies. As metioned before, the word ``chemical" in the ``galactic chemical evolution" is a bit of a misnomer. This is commonly applied simply to refer to studies interested in the evolution of metal abundances (either as a whole, or for individual isotopes) in galaxies over time, and less often to the astrochemistry that those elements may participate in. Likewise, ``metals" refers broadly to every element except H and He. Throughout this dissertation we often refer to the metallicity ($Z$) of a galaxy, its gas, or its stars. This quantity represents the total mass fraction of all metals -- and is computed as such in our simulations -- but we note that this is impossible to measure the abundances of all metals in astrophysical contexts outside our own Solar System. The metallicity of our own Sun ($Z_{\odot}$), for example, is still uncertain \citep{Asplund2009}. Instead, most observational works adopt Fe as a proxy for total metallicity in stars (due to its many, strong absorption lines) and O as a proxy in gas abundances (as it is the most abundant metal in the Universe and has convenient, strong emisson lines). These metallicity proxies are commonly reported in some form of normalized abundance ratio. For stellar metallicities, this is given in the form [A/B], where [A/B] = log($N_A / N_B$) - log($N_A / N_B$)$_{\odot}$, and $N$ refers to the number of atoms of a given element (usually as [Fe/H]). Confusingly, gas-phase abundances are commonly reported as just log($N_A / N_B$), often with the somewhat arbitrary normalization of 12, or log(O/H) + 12, for example.\footnote{And I am not going to mention issues in normalizatiion across methods of deriving these abundances... \citep[e.g.][]{KewleyEllison2008}.} 

\subsection{Hydrodynamics}
\textit{(This will be a sub-section of something, but putting here for now as placeholder).}

Astrophysical hydrodynamics simulations are first distinguished by the numerical methods employed to solve the Euler equations that describe the time ($t$) evolution of the energy density ($E$), density ($\rho$), pressure ($p$), and peculiar velocity ($\bm{v}$) of a fluid:
\begin{equation}
  \frac{\delta \rho}{\delta t} + \nabla \cdot \left(\rho \bm{v}\right)  = 0,\\
  \frac{\delta \rho \bm{v}}{\delta t} + \nabla \cdot \left(\rho \bm{v}\bm{v} + \bm{I}p \right) = 0,\\
  \frac{\delta E}{\delta t} + \nabla \cdot \left[\left(E +p\right)\bm{v}\right] = 0.
\end{equation}

Historically, codes fall into one of two camps: 1) Eulerian grid-based codes, including the popular codes / algorithms commonly in use (in some way) today such as \code{Zeus} \citep{StoneNorman1992}, \code{FLASH} \citep{FLASH}, \code{RAMSES} \citep{Teyssier2002}, \code{Athena} \citep{Athena}, \code{ART-II} \citep{Rudd2008}, and \code{Enzo} \citep{Enzo2014}, and 2) particle-based Lagrangian methods known as smooth particle hydrodynamics (SPH), such as \code{Gadget} \citep{Springel2005}, \code{PKDGRAV-2} \citep{Stadel2001}, \code{Gasoline} \citep{Wadsley2004}, and \code{Changa} \citep{Menon2015}. However, recent codes blur the lines between these distinctions with new algorithms for solving the fluid equations on a moving-mesh (e.g. \code{AREPO} \citep{Springel2010}), or meshless finite-mass / finite-volume methods (such as those implemented in \code{Gizmo} \citep{Hopkins2015}). Given the large variance in numerical methods, recent studies have examined the differences between these numerical implementations \citep[e.g.][]{Agertz2007}, including large-scale code comparison projects \citep[e.g.][]{AGORA,AGORA2}. While developing a notion of which code is ``correct" is an ill-posed problem, these studies have allowed for improvement across implementations and an insight into how numerical methods themselves drive uncertainty in our understanding of astrophysical problems.

In this Dissertation, we make use of the adaptive-mesh refinement (AMR) hydrodynamics code \code{Enzo}, as described in greater detail in Chapter~\ref{ch:chapter1}. As it pertains to galactic chemical evolution, however, the use of a grid-based code has the advantage that metals (which are advected as passive scalars that follow the fluid-flow in the simulations) are allowed to mix and diffuse naturally across fluid elements (grid cells). This is not possible in native SPH implementations, yet is necessary to reproduce realistic galactic chemical evolution properties as shown by multiple recent works testing implementations of diffusion in SPH simulations \citep[e.g.][]{Shen2010,Su2017a,Escala2018}. The disadvantage of the diffusion in \code{Enzo}, however, is that it is entirely numerical, and therefore its exact properties are challenging to characterize and are strongly resolution dependent. Throughout this dissertation we attempt to account for the effects of resolution on our results, but do not explicitly examine the properties of numerical diffusion itself. 


\subsection{Nucleosynthesis and Stellar Evolution} \label{intro:sec:nucleosynthesis}


\subsection{Radiative Processes and Chemistry}
\label{intro:sec:cooling}

Radiative processes are of fundamental importance in properly modelling the diversity of density, temperature, and ionization states in a multi-phase ISM. It is via radiative cooling that gas, heated by gravitational accretion onto dark matter halos, can cool, condense, become self-gravitating, and (eventually) collapse to form stars. Conversly heating, via radiative absorption or scatterings, from stellar sources within galaxies and the extragalactic sources that comprises the cosmic UV background (UVB, e.g. \cite{HM2001,HM2012,FG2011}).

\subsection{Star Formation} \label{intro:sec:sf}

In galaxy-scale hydrodynamics, star formation is typically treated as a process that occurs by a sub-grid model, with an assumed rate and assumed efficiency, in fluid elements that surpass a variety of threshold conditions. Usually, simulators set these thresholds to fully encompass gas whose Jeans-length (and thus dynamical behavior) becomes unresolved according to the widely-used \cite{Truelove1997} criterion. Often, this means a simple mass or number density threshold, on top of which simulators can also require that the local gas cells be in a converging flow ($\div \bm{v} < 0$), that gas surpass some virial threshold \citep[e.g.][]{}, or that gas also meet a minimum $H_2$ mass fraction \citep[e.g.][]{}. One a fluid element is flagged as star forming, simulators have a few options to choose from in how to determine what fraction and with what rate that mass is converted into stars. Briefly, some form of the following equation is usually adopted,
\begin{equation}
\label{intro:eq:sfr}
 \rm{SFR} \propto 
\end{equation}
,. Some works adopt modifications of this simple perscription, with varying star formation efficiency ($\eta$) (refs), or XXXX. Others, for application in high-resolution simulations where the mass of stars that would form in a single timestep ($dt \times \rm{SFR}$) is far less than a single star particle mass, allow stars to form out of star forming fluid elements stochastically \citep[e.g.][]{Goldbaum2015}.

Recent work has demonstrated that variations in these implemenations can lead to dramatic differences in the outcome of the simulated galaxies (refs, incl brooks/munshi work). However, it is generally found that the differences between implementations decrease in simulations with sufficient resolution ($\sim$ 10~pc or better spatial, or $\sim$ $100$ M$_{\odot}$ or better in mass) and with a sufficiently high density threshold for star formation ($n > 100$~cm$^{-3}$) (fire ref). 

% maybe a section on each FB method? R

\subsection{Stars in Simulations} \label{intro:sec:stars}

Stars, particularly large-scale cosmological simulations, are commonly represented as simple stellar populations (SSPs), or particles with IMF-averaged properties of stellar feedback and metal enrichment. As demonstrated in \cite{Revaz2016}, this approach is reasonable so long as the mass of the star particle is large enough to fully sample the IMF ($> 10^{4}$~M$_{\odot}$). In high resolution simulations, particularly simulations of small, low-mass systems like dwarf galaxies, this is no longer the case. This leads to inconsistencies in both the effective metal yield of the modelled stars, and (often) an artificial reduction of the effects of stellar feedback. Recent works have developed new models to address this issue, discretizing (rather than averaging over) individual feedback events \citep[e.g.][]{MUGS2010,FIRE,Hopkins2018,Rosdahl2018} or accounting for stochastic sampling of the IMF \citep{Hu2016,Hu2017,Applebaum2018,Su2018}. However, each of these methods utilizes a sub-grid recipe in some fashion to improve upon the SSPs. On the other extreme are highly resolved simulations which model star formation with ``sink particles" \citep[see for example ][]{Krumholz2004,Federrath2010,GongOstriker2013,BleulerTeyssier2014,Sormani2017} that account for the gradual accretion of gas in the growth of a single star (or group of stars) and protostellar feedback. These recipes are often computationally expensive, however, restricting their use over long timescales (100's to 1000's of Myr or more) in galaxy-scale simulations.

In this dissertation, we develop the first implementation that breaks the SSP formalism entirely by following stars as individual star particles sampled from an adopted IMF. This allows us to place unprecidented detail on both the stellar feedback and metal yields associated with each star, spatially and temporally resolving differences in both of these channels as a function of stellar mass and metallicity.

\subsection{Stellar Feedback} \label{intro:sec:feedback}

The first models of galacitc evolution produced galaxies with stellar masses far in excess of what is observed in the Universe, converting a substantial fraction of their baryons into stars. From these initial works, it was clear that some form of feedback process must take place to self-regulate star formation within galaxies. With observations of metal abundances in the CGM around galaxies \citep[see][ for a recent review]{Tumlinson2017}, and of fast-moving outflows from star forming galxaies \citep[see][ for a review]{Veilleux2005}, it was additionally clear that some mechanism must exist to drive these outflows. Stellar feedback became the clear physical mechanism to account for both of these phenomena.\footnote{Except maybe for more massive galaxies, where active galactic nuclei likely plan some role in regulating star formation and certainly play a role in driving outflows \citep[e.g.][]{Fabian2012}.}

Initial models for stellar feedback relied primarily on the energy injection from supernovae (SNe, see Section~\ref{intro:sec:supernovae}) to regulate star formation and drive outflows, but it has become clear that this mechanism alone cannot completely account for all of the star formation, ISM, outflow, CGM, and even intergalactic medium (IGM) properties of observed galaxies (see \cite{SomervilleDave2015} and \cite{NaabOstriker2017} for recent reviews and more detailed discussion of how feedback impacts outstanding problems in galactic evolution). Other sources of effective stellar feedback have been identified since \citep[e.g.][]{Agertz2013}, which serve to regulate star formation on different timescales and help develop and ISM and CGM with different phase and ionization properties, and includes stellar radiation (Section~\ref{intro:sec:radiation}) and stellar winds (Section~\ref{intro:sec:stellarwinds}). 

Although not included in this dissertation, for the sake of completeness, we mention a few additional sources of stellar feedback that likely contribute to the full picture of galactic evolution. A substantial amount of recent research has been dedicated towards the examination of supernova-driven cosmic ray feedback (see Section~\ref{ch1:sec:CRs} for a more detailed discussion and references). In brief, cosmic rays can be a dominant source of heading in very dense gas and act as a source of non-thermal pressure in the ISM, which can drive galactic winds with qualitatively different phase structures \citep[e.g.][]{SalemBryanCorlies}. In addition, binarity is often ignored in models of stellar feedback, but can dramatically extend the timescales over which ionizing radiation and supernovae act in a newly formed stellar populations, typically increasing their total energy output and decreasing the typical ISM densities in which SNe explode, increasing their effectiveness (see Section~\ref{ch1:sec:binary stars}). Finally, exoctic sources of stellar feedback, such as hupernovae, high-mass X-ray binaries, and neutron star neutron star mergers can also be a potentially important source of feedback, particularly in the lowest mass halos \citep[e.g.][]{Artale2015}. However, the uncertainties in the frequency and delay time distributions for when these events should occur, and there relative rarity preclude including these effects in this dissertation.

\subsubsection{Supernova Feedback}
\label{intro:sec:supernovae}

In the simplest model, supernova feedback is treated as the injection of pure thermal energy (and mass) into the fluid element(s) that are host / closest to the chosen site of a supernova explosion. In self-consistent models of star formation and feedback these sites are usually actual star particles, but this does not have to be the case. The first models which included supernova feedback lacked the spatial / mass resolution to resolve individual (or even collective) SN events. This is the source of the commonly known ``overcooling" problem, whereby too little energy is injected into too much mass such that the affected region only reaches modest temperatures ($T \lesssim 10^5$~K, as opposed to $T > 10^7$~K, or so). This temperature is right around the peak of the cooling curve, and thus the energy is rapidly radiated away before the injected energy has any time to make a significant dynamical impact of the galaxy. This is still a problem today in large-box simulations of cosmological volumes which have insufficient resulution to resolve individual SNe. While some of the initial, ad-hoc fixes for this problem (e.g. turning off cooling for some time in SN affected fluid elements, decoupling SN affected SPH particles from the hydrodynamics for some time) are still in use today, some notable advancements have been made to employ some combination of kinetic, momentum, and thermal energy feedback to attempt to capture SNe in a more consistent fashion \citep[e.g.][]{Hopkins2014,Simpson2015,Hopkins2018}.

High resolution simulations, like those developed in this dissertation, can typically resort to utilizing pure thermal energy injection into the fluid elements surrounding the supernova site and avoiding artificial ``overcooling". This is true for simulations with typical spatial / mass resolutions better than a few pc / 10 M$_{\odot}$ \citep{Simpson2015,Hu2018,Smith2018b}. However, a spatial resolution requirement is a density dependent statement; in this case supernovae that occur in the densest gas ($n > 10^{2-3}$~cm$^{-3}$) may still be under-resolved and ineffective. In this case, accounting for feedback processes from massive stars that occur before the first core collapse SN in a given star formation event becomes critical \citep{Hu2016}. Including these physics, as discussed below, greatly reduces the typical ISM densities in which SNe occur, greatly increasing the likelihood that these events are well resolved.

\subsubsection{Stellar Radiation}
\label{intro:sec:radiation}

We only briefly discuss this method of feedback here, as it is discussed in much more detail in both Chapters~\ref{ch:chapter1} and \ref{ch:chapter2}. In brief, \cite{Leitherer1999}, \cite{Agertz2013}, and others demonstrated that stellar radiation, accounts for a substantial fraction of the total energy output of a stellar population. Ionizing photons from young, massive stars can photoionize and photoheat the gas surrounding sites of recent star formation. This effectively pre-processes the local environment within which SNe occur, often increasing the effectiveness of SNe in regulating star formation and driving outflows \citep{Hu2016}. In addition, radiation pressure in the single-scattering limit (from UV photons) and rescattered infrared photons has been used as an additional source of pre-SN feedback \citep[e.g.][]{FIRE}. However, due in part to differences in how ionizing radition is followed across simulations, there is disagreement as to exactly how and in what conditions this acts as a source of feedback (see \cite{Krumholz2018} for a detailed examination of this problem).

Non-ionizing radiation also contributes to the regulation of star formation and the multi-phase ISM. Far ultraviolet (FUV) photons contribute to gas heating via the photoelectric heating of dust grains. This is a significant heating mechanism in higher metallicity environments, like the Milky Way \citep{Parravano2003,Wolfire2003}, but may still be important in setting temperature floor of cold dense gas in lower mass galaxies \citep{Forbes2016,Hu2017}. Lyman-Werner (LW) band radiation regulates the $H_2$ abundance, which is particularly important in low-metallicity environments where $H_2$ is a dominant coolant \citep[e.g.][]{Wolcott-Green2012}.\footnote{And is arguably even more important when using star formation perscriptions that depend upon the local $H_2$ mass fraction.} As the goal of this dissertation is to develop an as-complete-as-possible model of galactic chemical evolution, we account for each of the processes in some fashion, including the use of an adaptive ray-tracing radiative transfer scheme to track the stellar ionizing photons.

\subsubsection{Stellar Winds}
\label{intro:sec:stellarwinds}

Stellar winds, in addition to stellar radiation, are a potentially important source of pre-SN feedback that helps regulate star formation in / around the birth clouds of newly formed stars. However, the hot ($T>10^{6}$~K), fast ($v \sim 10^{3}$~km~s$^{-1}$) winds from massive stars \citep{Weaver1977} are challenging to model self-consistently in galaxy-scale simulations. For this reason, there has been comparatively little research on how they affect global galaxy properties, even though some models do include their energy injection in an integrated / approximate fashion \citep[e.g.][]{FIRE}. However, stellar winds, from both massive stars and weaker winds from the AGB phase of low mass stars, are important sources of metal enrichment. For these reasons we do include a simplified model for stellar wind feedback in this disseration.\footnote{Although we do not have conclusive results to this end, initial examination of our simulations has shown that our stellar wind model is only globally important in experimental simulations where radiation feedback is ignored. It is thus likely subdominant to radiation feedback.}

\section{Observations of Local Group Dwarf Galaxies}

We are motivated by using a properly formed model for galactic chemical evolution in dwarf galaxies to interpret and explain observations of Local Group dwarfs. As any good theoretical work should maintain a close connection to the observations that motivate the study in the first place in order to better make predictions for future observations, we devote some time here to summarize relevant observations of Local Group dwarfs.

\subsection{Milky Way and Local Group Dwarf Galaxies}

The Milky Way and its local environment is host to a diverse collection of dwarf galaxies. The properties of these galaxies have been nicely summarized in a recent review \citep{Tolstoy2009}, and again in \cite{McConnachie2012}. However, the number of known Local Group dwarf galaxies has increased dramatically since that time, particularly for the Milky Way, whose known satellite population has more than doubled. This is due in large part to tremendous improvements in our ability to detect galaxies on the faintest end of the luminosity function, called ultrafaint galaxies (UFDs, \cite{Willman2005}).\footnote{See \cite{Simon2019} for a recent review}. The Dark Energy Survey \citep[e.g.][]{Drlica-Wagner2015} and Hyper Supbrime-Cam \citep[e.g.][]{Greco2018} have made tremendous advances to this end, with the upcoming Large Synoptic Survey Telescope producing an even greater increase in the known population of faint galaxies in the Local Group \citep{Haynes2019,Weisz2019}. 

While there is no real sharp transition where a galaxy is considered a ``dwarf" galaxy, historical definitions are based on luminosity, using the SMC and LMC as rough anchor points for the upper-end of the dwarf galaxy scale. \cite{Bullock2017} suggests a deliniation based on stellar mass. For this dissertation, we follow this definition and consider galaxies with $M_* < 10^9$~M_${\odot}$ as dwarf galaxies, and galaxies with $M_* < 10^5$~M$_{\odot}$ as UFDs.



\subsection{Metals in Local Group Dwarfs}

Until recently, the study of chemical evolution in Local Group dwarf galaxies has focused predominantly on the stellar component. This is due in large part because, though dwarf galaxies have higher gas fractions than more massive galaxies \citep{Geha2006}, many of these galaxies have undergone environmental stripping processes, rendering them devoid of a measurable gas component \citep{GrcevichPutman2009}. 

% mention difficulties of measuring gas phase abundances in dwarf galaxies?

\subsection{Metals in the ISM and CGM of Dwarf Galaxies}

The proximity of Milky Way dwarf galaxy satellites allows us to make some of the most detailed extragalactic stellar abundances measurements. 

Stellar feedback drives outflows, galactic winds, from galaxies across all scales (references). Dwarf galaxies in particular, with their relatively shallow potential wells, drive effective galactic winds, with mass loading factors ($\eta$ = $\dot{\rm{M}}_{\rm out} / {\rm SFR}$) ranging from XXX to XXX (cite, muratov etc.). How exactly stellar feedback couples to the ISM to drive these stellar winds is still uncertain, and is the topic of substantial ongoing research, but effective winds 

\subsection{Leo P: A Case Study}

We dedicate this section to a specific Local Group dwarf galaxy, Leo P, as its observed properties motivate the initial conditions of the simulations developed in this dissertation. This galaxy is remarkable for being one of the lowest mass dwarf galaxies with observed, ongoing star formation. It is located at a distance of $1.62\pm 0.15$~Mpc \citep{McQuinn2015a}, and was first characterized as part of the ALFALFA survey \citep{Giovanelli2013,Rhode2013,Skillman2013,McQuinn2013,Bernstein-Cooper2014}. These studies have characterized its star formation history as low and roughly continuous over the age of the Universe (SFR = $4.3\times 10^{-5}$~M$_{\odot}$~yr$^{-1}$, \cite{McQuinn2015a}), its stellar mass ($M_{*} = 5.7 \times 10^{5}$~M$_{\odot}$, \cite{McQuinn2013}), its HI and dynamical mass ($M_{\rm HI}(r<r_{\rm HI}) = 9.5\times 10^{5}$~M$_{\odot}$, $M_{\rm dyn}(r<r_{\rm HI}) = 2.6 \times 10^{7}$~M$_{\odot}$, $r_{\rm HI} \sim 500$~pc, \cite{Bernstein-Cooper2014}), and gas-phase abundances from an HII region (12 + log(O/H) = 7.17$\pm$0.04 \cite{Skillman2013}). Finally, we chose to focus on a Leo P like galaxy in our simulations in part to allow for comparisons to the work in \cite{McQuinn2015}, which makes an accounting of the total metals (as traced by O) contained in the ISM and stars in Leo P, estimating the fraction that must have been ejected over the lifetime of the galaxy. This is an important measurement as prior estaimtes of this number for low mass dwarf galaxies were restricted to Milky Way dSph's, which are devoid of gas and contaminated by environmental affects. As of this writing, Leo P is one of the only low mass dwarf galaxies with both gas-phase and stellar metal abundance measurements, but there is ongoing work to greatly expand this sample in the near future.

\section{Onezone Models of Galactic Chemical Evolution}
\label{intro:sec:onezone}

This dissertation utilizes high-resolution hydrodynamics simulations that are expensive to run in both walltime ($\sim$ weeks to months) and computational time ($\sim 10^{5-6}$ CPU hours). For this reason, 

 One of the most common ways to model and understand the stellar populations of galaxies in high-dimensional chemical space is through simplified one-zone (or many-zone) models. This treatment extends back many decades \citep[e.g.][]{Schmidt1963,TalbotArnett1971,Lynden-Bell1975}, where galaxies and their total metal abundance were treated as closed-box systems of gas and stars. Over the interveaning decades these models have improved, including perscriptions for gas outflow (a ``leaky-box") or gas inflow (``accreting box"), or both (a ``bathtub" model, e.g. \cite{FinlatorDave2008,Bouche2010}). The assumptions that go into these models vary dramatically with complexity. Simple models can assume, for example, that long-lived low mass stars do not contribute to metal enrichment while metals from massive stars are instantaneously recylced into the galaxy's gas reservoir (i.e. ignoring stellar lifetimes) or that all metals mix instantly and homogenously. More complex models account for individual stellar lifetimes (to some degree), or may try and account for some aspects of inhomogenous mixing by constructing mulit-zone models that may separate a galaxy in radial bins (with or without mixing between bins), may account for a hot and cold component separately, with mixing between the two, or account for separate ISM and CGM components. The output of these models is typically the mean metal abundance evolution within each of the tracked components. There is generally no self-consistent perscription to account for the scatter in any abundance relationship in these models, which, when needed, is often added ad-hoc in post-processing. Regardless of the assumptions that go into these models, they are powerful tools that can be used to probe the general physical processes that govern galactic chemical evolution, and explore the vast, uncertain parameter space associated with the models that characterize these processes \citep[e.g.][]{Cote2017a}.
 
One final motivation of this dissertation is to improve the assumptions within these models by better charaterizing the process of inhomogenous mixing in the ISM, and the coupling of metals to galactic winds and outflows. This is particularly important if the behavior differs between elements from different nucleosynthetic sites. In addition, this work could be used to account for the spread and higher-order statistics of stellar abundance distributions, which would be a powerful tool for better leveraging the wealth of observations in the Milky Way and Local Group. Yet, as discussed, the current state of the art perscriptions for these processes are ad-hoc, lacking physical motivation. Recent analytic work in \cite{KrumholzTing2018} lays down a framework to evaluate the correlation between individual metals in different galactic environments and predicts that the correlations will differ for different nucleosynthetic sites (e.g. AGB winds vs. SNe). We characterize these differences in detail in Chapter~\ref{ch:chapter3}, laying the groundwork for how one could potentially incorporate this insight into improving these models. We build upong this idea more in Chapter~\ref{ch:chapter4}, and plan to explore the connection between our simulations and analytic models in the future. Recent work by \cite{SchonrichWeinberg2019} is a great example of how this could greatly impact current models for galactic chemical evolution. They how adopting a two-phase onezone model (hot and cold ISM), along with parameters for how metals are injected and mix between the phases whose values differ depending on nucleosynthetic origin of the metal, can reconcile long-standing problems in understanding the r-process abundance evolution of stars in the Milky Way. And in fact, they argue that it would be impossible to reproduce observed r-process abundances in the Milky Way (namely [Eu/Fe], [Eu/Si], and [Eu/Mg] as functions of [Fe/H]) with a standard one-phase model.


\section{Structure of Dissertation}\label{intro:sec:structure}

In this work we investigate the role of both stellar feedback and mixing in a multi-phase ISM in driving galactic chemical evolution. Using a novel method for following star formation and stellar feedabck in galaxy-scale hydrodynamics simulations, we provide significant insight into how individual metals from distinct nucleosynthetic sites enrich the ISM, how their abundances are set and imprinted upon newly formed stars, and how they are ejected from galaxies through galactic winds. 

In Chapter \ref{ch:chapter1} \citep[published as ][]{Emerick2019}, we motivate the need for a new model of star formation and stellar feedback to address the open uncertainties in galactic chemical evolution. We implement such a model, which follows stars as individual star particles over a fully sampled IMF, in hydrodynamics simulations of an isolated, low-mass dwarf galaxy. We describe in detail the physics included in these simulations, including a stellar feedback model that follows stellar radiation, stellar winds, and supernovae. We find that RESULTS

In Chapter \ref{ch:chapter2} \citep[published as ][]{Emerick2018a}, we investigate how stellar ionizing radiation feedback impacts the evolution of our simulated low mass dwarf galaxy. We confirm the findings in recent works, as dicsussed in Section~\ref{intro:sec:radiation}, that find that stellar radiation is an important source of pre-supernova feedback that regulates star formation and can help drive galactic outflows outflows. We additionally investigate how radiation feedback couples to the ISM to drive galactic winds by demonstrating that the ionization of gas far from individual ionzing sources, out into the disk-halo interface of our galaxy, creates low-density channels through which supernovae can readily drive gas and metals into and beyond the galaxy's CGM. This suggests that not only is the local deposition of energy from stellar radiation important, which quickly destroys dense gas in stellar birth-sites, but also long-range effects are important to consider in a consistent model of stellar feedback.

In Chapter \ref{ch:chapter3} \citep[published as ][]{Emerick2018b}, we dive in detail into the evolution of each of the 15 individual metal species that we follow in our simulation. We focus primarily on the evolution of the gas-phase abundances of these metals, and how the site of nucleosynthesis (e.g. AGB winds or SNe) drives the subsequent evolution of each species. We find that elements released in lower energy enrichment sources (e.g. AGB winds) mix less effectively in the ISM across all phases than elements released in higher energy enrichment sources (e.g. SNe). Likewise, energy difference associated with their enrichment directly impacts their ability to couple to galactic outflows. We find that elements released in AGB winds are retained at a higher fraction (by a factor of $\sim$ 4-5) than elements released in SNe. 

In Chapter \ref{ch:chapter4} we discuss recent, unpublished results which build upon the work in Chapter~\ref{ch:chapter3} by investigating in more detail how energy, spatial location within the galaxy, and galaxy SFR affect the mixing and ejection of metals for individual enrichment events. With a more detailed study, we again confirm that metal mixing in the ISM and ejection in galactic winds depends upon the energy of the enrichment event, sampling event energies from AGB winds ($\sim$ erg) to hypernovae ($\sim 10^{52}$~erg).

Finally, we conclude in Chapter \ref{ch:conclusion} and briefly discuss future work as extensions of this Dissertation.
