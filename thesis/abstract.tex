%
% Abstract
%

\thispagestyle{empty}
\begin{center}

{\Large \bf ABSTRACT}

\vskip.35in
{\Large \bf \thesistitle}

\vskip.35in
{\large Andrew J. Emerick} \\
\vskip.35in
\end{center}
Motivated by the desire to better understand two of the largest outstanding problems in galactic evolution -- stellar feedback and galactic chemical evolution --
%In this \thesis
we develop the first set of galaxy-scale simulations that simultaneously follow star formation with individual stars and the multi-channel stellar feedback and multi-element metal yields associated with each star. We developed these simulations as a means to better understand the way in which stellar feedback, including stellar winds, stellar radiation, and supernovae, couples to the interstellar medium (ISM), regulates star formation, and drives outflows in dwarf galaxies. We follow the evolution of the individual metal yields associated with these stars in order to better understand how metals mix within the ISM and are ejected into the circumgalactic and itergalactic media (CGM, IGM) through outflows. This study is directed with the goal of better understanding the ever increasing quality of stellar abundance measurements within our own Milky Way galaxy and in nearby dwarf galaxies.

Our simulations follow the evolution of an idealized, isolated low mass dwarf galaxy ($M_{\rm vir} \sim 10^{9}$~M$_{\odot}$) over $\sim$ 500 Myr timescales using the adaptive mesh refinement hydrodynamics code \textsc{Enzo}. We implemented a new star formation routine which deposits stars individually from 1~M$_{\odot}$ to 100~M$_{\odot}$. Using tabulated stellar properties, we follow the stellar feedback from each star. For massive stars ($M_* > 8$~M$_{\odot}$) we follow their stellar winds, ionizing radiation (using an adaptive ray-tracing radiative transfer method), the FUV radiation which leads to photoelectric heating of dust grains, Lyman-Werner radiation, which leads to H$_2$ dissociation, and core collapse supernovae. In addition, we follow the asymptotic giant branch (AGB) winds of low-mass stars ($M_* < 8$~M$_{\odot}$) and Type Ia supernovae. We investigate how this detailed model for stellar feedback drives the evolution of low mass galaxies. We find agreement with previous studies that these low mass dwarf galaxies exhibit bursty, irregular star formation histories with significant feedback-driven winds.

Using these simulations, we investigate the role that stellar radiation feedback plays in the evolution of low mass galaxies.
% a fiducial, full-physics simulation, second, a simulation with ionizing radiation restricted to the region immediately around massive star particles, and third, a simulation with no ionizing radiation feedback. I
In the regime of low mass dwarf galaxies, we find that the local effects of stellar radiation (within $\sim$ 10~pc of the massive, ionizing source star) act to regulate star formation by rapidly destroying cold, dense gas around newly formed stars. For the first time, we find that the long-range radiation effects far from the birth sites are vital for carving channels of diffuse gas in the ISM which dramatically increase the effect of supernovae. We find this effect is necessary to drive strong winds with significant mass loading factors and has a significant impact on the metal content of the ISM.

Focusing on the evolution of individual metals within this galaxy, it remains an outstanding question as to what degree (if any) metal mixing processes in a multi-phase ISM influence observed stellar abundance patterns. To address this issue, we characterize the time evolution of the metal mass fraction distributions of each of the tracked elements in our simulation in each phase of the ISM. For the first time, we demonstrate that there are significant differences in how individual metals are sequestered in each gas phase (from cold, neutral gas up to hot, ionized gas) that depends upon the energetics of the enrichment sources that dominate the production of a given metal species. We find that AGB wind elements have much broader distributions (i.e. are poorly mixed) as compared to elements released in supernovae. In addition, we demonstrate that elements dominated by AGB wind production are retained at a much higher fraction than elements released in core collapse supernovae (by a factor of $\sim$ 5).

We expand upon these findings with a more careful study of how varying the energy and spatial location of a given enrichment event changes how its metal yields mix within the ISM. We play particular attention to events that could be associated with different channels of r-process enrichment (for example, neutron star - neutron star mergers vs. hypernovae) as a way to characterize how mixing / ejection differences may manifest themselves in observed abundance patters in low mass dwarf galaxies. \textbf{Note to self to revisit once this chapter is written...}

Finally, we summarize how this new understanding of galactic chemical evolution -- that metal mixing and ejection from galaxies is not uniform across metal species -- can be used to improve significantly upon current state of the art galactic chemical evolution models. These improvements stand to help improve our understanding of galactic chemical evolution and reconcile outstanding disagreements between current models and observations.
