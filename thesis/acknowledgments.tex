%
%
% Acknoledgements
%
%

\newpage

\begin{center}

{\large \bf ACKNOWLEDGMENTS } % leave this title

\end{center}

\vspace{0.8cm}

To my wife Ellen, who moved across the country for me to NYC
(and across the country again in the next few months), who has patiently put up with
all of my travel and time away at conferences in the past few years,
who has worked longer hours at much more demanding jobs than the work I have put
into this \dissertation, and who has continued to support me through this
entire journey, all just so that I could pursue my dream. A simple ``thank you" is
entirely insufficient to repay her for these sacrifices, but its a good thing
I have the rest of my life to show how grateful I am.

To my family who have continued to support me in countless ways.
In many ways my parents are directly responsible for me pursuing a career in Astronomy.
They enabled me to have a good education throughout my life that prepared me
well for graduate school. But maybe more importantly, they provided me with my
first exposure to the wonders of the Universe which gave me the desire to
pursue the path that led me to where I am today.

I have had many research advisors over the years, but am most indebted to
my thesis advisors, Greg and Mordecai, and my committee, Mary and Kathryn. Their
thoughts, advice, and encouragement over the past six years have been invaluable.
This \dissertation  is as much a testament of my formation as a scientist as it
is of the quality of their mentorship. I would also like to thank my various
undergraduate thesis advisors at the University of Minnesota,
Larry Rudnick, Tom Jones, and Priscilla Cushman, my REU advisor at
Texas A\&M University, Ralf Rapp, and Pablo Yepes at Rice University.

I want to thank all of the other graduate students at Columbia that I have overlapped
with over the years. This includes especially Sarah, Steven, and Dan,
and also Susan, Andrea, Munier, Alejandro, Jeff, and Adrian.
Their support and advice -- and shared laughs and beers --
already make me look back on this time with fond memories.

While I have spent much of my time working in my office(s) at either Columbia,
AMNH, or the Flatiron Institute, a surprising fraction of this \dissertation
and the published works it is based upon were written outside of these
places. I want to thank and acknowledge the many coffee shops, bars, breweries,
and the work space in my climbing gym in which I have been
surprisingly productive in times when I found my office uninspiring.

Thank you to everyone I have met in the NYC climbing community who have collectively
helped keep me sane throughout my time in graduate school. Over the past few years
I have learned the importance of a proper work-life balance, and climbing has helped
me keep that balance (... maybe a bit too well...). Thank you especially to my
regular climbing partners -- Daniel Negless, Hunter Whaley, Dennis Kramer, and Lauren Anderson -- who I've trusted
with my life countless times, and the handful of other astronomer-climbers that I've
climbed with throughout graduate school (Jeff Andrews, Matt Wilde, Lauren Anderson, Dan D'Orazio,
Dan Foreman-Mackey, and Andrea Derdzinski).

I would like to thank the numerous developers and contributors to the
open-source software packages without which this \dissertation would not be
possible. This includes more general software, such as \textsc{Python}, \textsc{IPython}, \textsc{NumPy},
\textsc{SciPy}, \textsc{Matplotlib}, \textsc{HDF5}, \textsc{h5py},
and \textsc{deepdish}, and more astronomy-specific projects,  \textsc{yt}, \textsc{Enzo},
\textsc{Grackle}, \textsc{AstroPy}, and \textsc{Cloudy}. In particular I would like
to thank the many developers of \textsc{yt}, \textsc{Enzo}, and \textsc{Grackle}
who have helped answer questions, debug code, review my pull requests, and implement
new features on my behalf throughout this process. This includes: Britton Smith, Nathan Goldbaum,
John Wise, Matt Turk, Greg Bryan, Simon Glover, Munier Salem, Cameron Hummels,
Christine Simpson, Brian O'Shea, and others who I may have forgotten.

Finally, I would like to thank you -- the reader -- for deciding that this
\dissertation is worth your attention.

\vspace{1.8cm}
2019, New York, NY

%\setlength{\baselineskip}{1.11111 \baselineskip}
