%mm substantive changes marked like this, others made silently

\documentclass[twocolumn]{aastex61}
\pdfoutput=1 %for arXiv submission
\usepackage{amsmath,amstext}
\usepackage[T1]{fontenc}
\usepackage{apjfonts}
\usepackage[figure,figure*]{hypcap}

\renewcommand*{\sectionautorefname}{Section} %for \autoref
\renewcommand*{\subsectionautorefname}{Section} %for \autoref

\shorttitle{Differential Metal Mixing}

\begin{document}

%\title{Evolution of Individual Metals in Galactic Winds and the ISM}

% GB(AE): Make sure to specify DG in title and at minimum hint at results of paper (include log-normal some how??)
% AE: Had a hard time coming up with a succinct and interesting sounding title that
%     fit both of the interesting results:
%          1) analytic PDF fits and 2) variations in metal mixing...
\title{Metal Mixing
%mm
    and Ejection
in Dwarf Galaxies is Dependent on Nucleosynthetic Source}

%Non-Uniform Metal Distributions in a Dwarf Galaxy Simulation}
%Variations in the Distribution and Evolution of Individual Metals in Dwarf Galaxies}

\author{Andrew Emerick}
\affiliation{Department of Astronomy, Columbia University, New York, NY, 10027, USA}
\affiliation{Department of Astrophysics, American Museum of Natural History, New York, NY, USA}
\author{Greg L. Bryan}
\affiliation{Department of Astronomy, Columbia University, New York, NY, 10027, USA}
\affiliation{Center for Computational Astrophysics, Flatiron Institute, 162 5th Ave, New York, NY, 10003, U.S.A}
\author{Mordecai-Mark Mac Low}
\affiliation{Department of Astrophysics, American Museum of Natural History, New York, NY, USA}
\author{Benoit C{\^o}t{\'e}}
\affiliation{T.B.D.}
\author{Kathryn V. Johnston}
\affiliation{Department of Astronomy, Columbia University, New York, NY, 10027, USA}
\author{Brian O'Shea}
\affiliation{T.B.D.}



\begin{abstract}
Using a high resolution simulation of an isolated dwarf galaxy, accounting for
%mm
    multi-channel
stellar feedback and chemical evolution on a star-by-star basis, we investigate how each of 15 metal species are distributed within our multi-phase interstellar medium (ISM) and ejected from our galaxy
%mm through
    by
galactic winds. For the first time, we demonstrate that the mass fraction probability distribution functions (PDFs) of individual metal species in the ISM are well described by a piecewise log-normal
%mm +
   and
power-law distribution. The log-normal component generally describes gas at lower metal fractions undergoing enrichment towards the mean metal fraction, driven by recent enrichment events in the power-law tail. The PDF properties, and relative significance of these two components, vary within each ISM phase. Hot gas is dominated by recent enrichment, with a significant power-law tail to high metal fractions, while cold gas is dominated by the log-normal component. In addition, elements dominated by asymptotic giant branch (AGB) wind enrichment (such as N and Ba) mix less efficiently than elements dominated by supernova enrichment (such as $\alpha$ elements and Fe). This difference is driven by the differences in
%mm energetics between each source, particularly the poor coupling of AGB winds to the ISM.
    source locations, particularly the higher chance compared to massive stars for AGB stars to eject material into cold gas.
Nearly all of the produced metals are ejected from the galaxy (only 4\% are retained), but metals dominated by AGB enrichment are retained at a fraction of 20\%. In dwarf galaxies, therefore, elements synthesized predominately through AGB winds should be both overabundant and have a larger spread than elements synthesized in either core collapse or Type Ia supernovae. We discuss the observational implications of these results, their potential use in developing improved models of galactic chemical evolution, and their generalization to more massive galaxies.
\end{abstract}

\keywords{}

\section{Introduction}
Understanding galactic chemical evolution for all metal species across galaxy mass scales remains one of the most challenging aspects of modeling galaxy evolution (\textit{references}). One of the most pressing difficulties is a lack of understanding in exactly how metals propagate from their injection sites (from stellar winds or supernovae) and mix through the phases of the interstellar medium (ISM) into star forming gas.
%
% get a sentence with a better punch at the end
%

One-zone chemical evolution models often assume instantaneous mixing of metals from recent star formation into gas available for future star formation (references). While this incorrectly assumes that metal mixing in the ISM plays no role in delaying future enrichment in star formation, how to properly account for this process is poorly understood. This is in large part because both these models and large-scale cosmological simulations lack the necessary fidelity to capture the detailed, multi-phase mixing process of metals in the ISM directly. Recent hydrodynamics simulations have employed parameteric models to account for unresolved sub-grid metal mixing (\textit{need references}), which plays an important role in determining the chemical properties of galaxies (\textit{need references}). Consequently, following metal mixing, if treated in these models, requires assumptions about how metals mix within the ISM and how metals couple to galactic outflows. When metals are tracked individually, as opposed to a global metallicity field, their injection, mixing, and outflow properties are often treated uniformly. However, these processes are critical components in understanding galactic chemical evolution. Detailed hydrodynamics simulations incorporating a realistic multi-phase ISM and detailed stellar feedback are required to understand metal mixing and metal outflows \citep[e.g.][]{Armillotta2018}. In addition, it remains to be seen to what degree, if at all, metals of different nucleosynthetic origins may couple differently to the ISM.

The use of low mass dwarf galaxies, both observationally in the Local Group and in theoretical models, has been critical in improving our understanding of galactic chemical evolution. That dwarf galaxies are efficient metal polluters has been demonstrated for some time both theoretically \citep[e.g.][]{DekelSilk1986,MacLowFerrara1999,Fragile2004}, and with direct observational evidence from the metal retention fractions of Local Group dwarf galaxies \citep[e.g.][]{Kirby2011-metals,McQuinn2015}. It is commonly assumed that all metals couple equally to galactic outflows, however, which may not be a valid assumption if metals of different nucleosynthetic origins couple differently to the ISM \citep{KrumholzTing2018}. Relaxing this assumption has implications for both interpreting observations of stellar abundances in nearby dwarf galaxies and in modeling galactic chemical evolution in both semi-analytic models and lower resolution cosmological simulations. However, testing this assumption has only become possible recently as it requires high resolution, galaxy-scale hydrodynamics simulations that can resolve ISM mixing and self-consistently drive galactic winds through multi-channel stellar feedback.

It is becoming increasingly important to develop a concrete theoretical understanding of galactic chemical evolution as a result of multiple, recent observational campaigns to probe detailed stellar abundances in our Milky Way and the Local Group. (\textit{ references to APOGEE, GAIA, GALAH, RAVE, TGAS}). Stellar abundances are directly imprinted with the enrichment pattern of their star forming cloud, whose chemical properties are determined by the process of turbulent metal mixing in the ISM. The degree to which we can associate "chemically tagged" stars as co-eval depends directly upon our understanding of metal enrichment and metal mixing in the ISM. However, this understanding is critical for using these observations to further deduce properties of a galaxy's evolutionary history.

In this paper we present the first detailed chemical evolution results from a set of high-resolution hydrodynamics simulations of an isolated, low-mass, dwarf galaxy performed with the adaptive mesh refinement code \textsc{Enzo} \citep{Enzo2014}. The simulation discussed here was introduced in detail in \cite{Emerick2018} (hereafter Paper I). To address the outstanding questions discussed above, these simulations follow star formation using individual star particles, including stellar feedback from massive star and AGB-phase stellar winds, photoelectric heating, Lyman Werner dissociation, ionizing radiation tracked through an adaptive ray-tracing radiative transfer method, and core collapse and Type Ia supernovae. This is in addition to a detailed model for ISM physics using the \textsc{Grackle} library, as discussed below. We show that metals are strongly ejected via galactic winds, but that the retention of metals in the ISM and their mixing through phases varies significantly depending on the production source of the given elemental species. We show how these elements are distributed in the ISM and conclude with a discussion on the implications of these results.

The physical processes that drive galactic chemical evolution are complex, driven by the details of stellar feedback, turbulence and diffusion in a multi-phase ISM, and variations in stellar yields with nucleosynthetic source and stellar metallicity. Uncertainties in each of these processes make reproducing and interpreting observations of gas and stellar abundances challenging. [list examples of modelling problems]. These uncertainties, combined with the difficulty in simulating a fully self-consistent galaxy in detail, motivates this current study. By focusing on a low mass dwarf galaxy, with small size and low star formation rate, we can capture detailed feedback and ISM physics at high resolution, while following individual stars. In this work we focus on theoretical quantities, namely the probability density distributions (PDFs) of metals in the ISM as a function of metal mass fraction, rather than the common observables of stellar and gas abundance ratios, for two reasons. First, we would like to build our understanding of galactic chemical evolution from a fundamental level. Second, there are currently no direct observational comparisons we can make to galaxies of this size, and it is computationally infeasible to simulate galaxies
%mm with observational analogs
  of easily observable size
in as much detail as done here. This work will be the first of several attempting to bridge this gap. With a more fundamental understanding of metal mixing and stellar enrichment in galaxies, we can construct better one-zone chemical evolution models and physically motivated sub-grid physics models for lower resolution simulations of more massive galaxies.

We summarize our methods and physics models in Section~\ref{sec:methods}. We begin our analysis by presenting the only direct observable comparison we can make at this galaxy mass, discussing the metal retention fractions of our galaxy in Section~\ref{sec:ejection}. In Section~\ref{sec:log-normal} we focus on how each of our 15 individual metal species are distributed and evolve within each phase of the ISM.
%mm
   We discuss our results in Section~\ref{sec:discussion}, and conclude in
   Section~\ref{sec:conclusions  }.

\section{Methods}
\label{sec:methods}
We refer the reader to Paper I for a detailed description of our numerical methods and feedback models. We briefly summarize the relevant details here.

%mm \textit{Hydrodynamics ---}
\paragraph{Hydrodynamics} We use the adaptive mesh refinement astrophysical hydrodynamics and N-body code \textsc{Enzo}\footnote{} \citep{Enzo2014}. Hydrodynamics are solved using a direct-Eulerian piecewise parabolic method and a two-shock approximate Riemann solver with progressive fallback to more diffusive solvers. We include gas self-gravity and evolve collisionless star particles using a particle mesh N-body solver. We use a 128$^{3}$ base grid measuring 2.16~R$_{vir}$ on a side, where R$_{\rm vir}$ = 27.4~kpc, and 8 levels of refinement, for a maximum spatial resolution of 1.8~pc. Refinement occurs when either: 1) a cell contains more than 50~M$_{\odot}$ of gas
%mm and
    or
2) a cell's local Jeans length becomes resolved by less than eight cells. Also, if the cell is within four zones of a star particle with active feedback, it is refined to the maximum resolution. At the maximum resolution, we use a pressure floor to prevent artificial fragmentation when the Jeans length is unresolved.

%mm \textit
\paragraph{Chemistry, Heating, and Cooling Physics} We use the the astrophysical chemistry and cooling package \textsc{Grackle} \citep{GrackleMethod} to evolve a nine species non-equilibrium primordial chemistry model (including molecular hydrogen), follow approximate metal line cooling using a \textsc{Cloudy} look-up table, and apply heating from a metagalactic UV background \citep{HM2012}. We account for approximate self-shielding of H~{\sc i} against the UV background following \cite{Rahmati2013}. We assume He~{\sc i} self-shields in the same fashion as H~{\sc i}, and ignore He~{sc ii} heating from the UVB entirely. Approximate H$_2$ self-shielding from background Lyman-Werner (LW) radiation is accounted for using the Sobolov-like method from \cite{Wolcott-Green2011}. Finally, we use the updated metal line cooling tables which self-consistently account for the decrease in metal line cooling rates due to lower ionization fractions in self-shielding gas, as compared to metal cooling tables computed under the optically thin assumption.

%mm \textit
\paragraph{Star Formation} Stars are followed as individual star particles from 1~M$_{\odot}$ to 100~M$_{\odot}$. Stars are able to form in dense gas with: 1) n~$>$~200~cm$^{-3}$, 2) T~$<$~200~K, and 3) $\nabla \cdot v < 0$. Given the short time-steps ($dt \sim 500$~yr) and high resolution (1.8~pc) in these simulations, the local star formation rate in any single zone is very small ($\ll$ 1~M$_{\odot}$ dt$^{-1}$). We therefore form stars stochastically, depending upon the local gas mass, free-fall time, and star formation efficiency, $\epsilon_{\rm f}$, taken to be 2\%. Stellar masses are randomly sampled from an assumed \cite{Salpeter1955} IMF with metallicities and metal fractions set by the local gas environment. We use the zero age main sequence properties of stars from the \textsc{PARSEC} stellar evolution tables \citep{Bressan2012,Tang2014} to determine individual stellar lifetimes and properties which are used to set their LW band, far ultraviolet (FUV) band, and ionizing radiation luminosities (see below).

%mm \textit
\paragraph{Stellar Feedback and Stellar Yields} We track the feedback and yields of 15 metal species for each star individually. Stars between 8 M$_{\odot} < M_{*} < 25 M_{\odot}$ explode as core collapse supernovae at the end of their life, injecting their mass and 10$^{51}$~erg of thermal energy into a spherical region with radius of 5.4~pc, or 3 times the maximum resolution. Stars above this mass are assumed to direct collapse with no mass or energy injection. For all stars above 8~M$_{\odot}$ we follow their stellar winds assuming continuous mass loss over their lifetimes, their LW and FUV radiation as optically thin radiation which contributes to H$_2$ dissociation and photoelectric heating respectively, and their H~{\sc i} and He~{\sc i} ionizing radiation using an adaptive ray-tracing radiative transfer method \citep{WiseAbel2011}. We interpolate over the OSTAR2002 \citep{Lanz2003} grid to set the luminosities of each of these stars. Low mass stars, $M_{*} < 8~M_{\odot}$, do not produce feedback during their main sequence lifetimes, but end their lives injecting a short-lived, low velocity (10 km~s$^{-1}$) AGB wind. Stars with $3 < M_{*,o} < 8$ are tracked after their death as possible Type Ia supernova progenitors, using a delay time distribution model to assign when (if at all) they will explode as a Type Ia, injecting 10$^{51}$~erg of thermal energy with yields. Stellar yields are computed using the NuGrid stellar yield database \citep{Ritter2017} for all stars with $M_{*} < 25~M_{\odot}$, \textbf{Slemer et. al. in prep} for the stellar winds of stars with $M_{*} > 25~M_{\odot}$, and \cite{Thielemann1986} for Type Ia supernovae.

\paragraph{Initial Conditions} Our dwarf galaxy is initialized to approximate, but not reproduce, the $z = 0$ properties of an ultrafaint dwarf galaxy (UFD) as informed by the observed properties of Leo P \citep[see ][]{Giovanelli2013,McQuinn2015a,McQuinn2015}. We initialize a M$_{\rm gas} = 1.8 \times 10^{6}$~M$_{\odot}$ disk as an expotential profile with $Z = 4.3 \times 10^{-4}$ centered on a static \cite{Burkert1995} dark matter potential with M$_{\rm vir} = 2.5 \times 10^{9}$. The gas scale radius and scale height are set to 250~pc and 100~pc respectively, with a maximum radial extent of 600~pc. Both the gas temperatures and velocities are set iteratively to enforce initial hydrostatic equilibrium. The galaxy contains no initial background stellar population, limiting the number of Typa Ia supernovae in our model, but we do provide initial supernova driving to limit the initial transient burst of star formation from the initial cooling and collapse of the galaxy. Although the total metallicity field is initially non-zero, we only track the self-consistently produced metal enrichment for each individual metal species in our simulation, setting their initial abundances to zero. Our analysis is based on the evolution of this galaxy during the first 500~Myr after the formation of the first star particle.

\section{Results}
For context, in Paper I we focused on the global properties of the evolution of this dwarf galaxy, including an analysis of the galaxy's gas mass and star formation evolution, the ISM properties in terms of mass fractions, volume fractions, and phase diagrams, the interstellar radiation field in each tracked radiation band, the gas outflow rates and galactic wind velocities, and the retention / ejection of metals from the galaxy. This galaxy has an average star formation rate of $1.2~\times 10^{4}$ M$_{\odot}$ yr$^{-1}$ and is consistent with observed low mass dwarfs galaxies in the Kenicutt-Schmidt relation.  The galaxy exhibits strong, feedback driven outflows that eject a significant amount of gas and metals from the galaxy. The mass loading factor at 0.25~R$_{\rm vir}$ was found to be $\eta \sim 50$. These winds eject 95\% of the metals produced in the galaxy, with 50\% of all metals leaving the virial radius by the end of the simulation time.

In this work we focus in detail on how each of the 15 individual metal species evolve in this galaxy. We address differences between the ejection fractions of each metal in Section~\ref{sec:ejection} and analyze for the first time the mass-fraction PDFs of the metals retained by the ISM in Section~\ref{sec:mixing}.

\subsection{Preferential Ejection of Metals from the ISM}
\label{sec:ejection}

\begin{figure}
\centering
\includegraphics[width=0.99\linewidth]{species_bar.png}\\
\includegraphics[width=0.99\linewidth]{species_bar_ISM.png}
\caption{The fraction of each metal species in the full simulation box contained in the halo, gas in the galaxy disk, and stars (top), and the fraction of species within the disk alone in each phase of the ISM (bottom). The leftmost bar in each plot shows the sum of all metals.}
\label{fig:species_fractions}
\end{figure}

The top panel of Figure~\ref{fig:species_fractions} gives the mass fraction of each element at the end of the simulation (500~Myr) in each of four reservoirs: locked in stars (yellow), in the ISM (blue), outside the galaxy but within the virial radius (purple), and outside the virial radius (salmon), including gas that has left the domain. We subdivide the ISM by phase in the bottom panel, giving the mass fractions of each element in the cold neutral medium (CNM, $f_{\rm H_2} < 0.5$,  T$< 100$~K), warm neutral medium (WNM, $10^2~\rm{K}\le \rm{T} < 10^4~\rm{K}$), warm ionized medium (WIM, $10^4~\rm{K}\le \rm{T} < 10^{5.5}~\rm{K}$), the hot ionized medium (H~{\sc i}M T$\ge 10^{5.5}$~K), and locked in stars (yellow). Again, with the exception of H and He, we only consider metals produced self-consistently through our star formation and stellar feedback methods; the initial mass of each metal species is zero.

We find two major results. First, nearly all of the metals reside within neutral gas, mostly in the CNM; only a few percent reside in the hot phases. The exact fraction varies with each species, most notably for carbon; these fluctuations are at most $\sim 10$\%. This is not surprising, as the cold phases represent the majority of the mass in the ISM, but, as shown in Section~\ref{sec:statistics}, even though the cold phases harbor most of the metals, the hot phases have significantly higher metal mass fractions. Second, only a small fraction of produced metals are retained within the dwarf galaxy, in agreement with observations of nearby dwarf galaxies \citep[see][]{Kirby2011-metals, McQuinn2015}; however,the retention factor varies. The top panel shows a qualitative disagreement between the retention fractions of N and Ba (about 20\% for each) as compared to the rest of the metals ($\sim$ 4 -- 5\%). This suggests that individual metals \textit{do not} share the same dynamical evolution. Clearly, metal enrichment in galaxies is a phenomenon that cannot be fully captured using single, global metallicity field. In this particular case, assuming that all metals behave the same would underestimate the N and Ba enrichment by a factor of up to five, at least for low mass dwarf galaxies.

\begin{figure}
\centering
\includegraphics[width=0.95\linewidth]{species_bar_sources.png}\\
\caption{The fraction of total mass in each metal species produced by each of the four possible nucleosynthetic channels in our model. These channels differ in both when metals are ejected, as determined by stellar evolution, and the energy with which they are ejected into the ISM. We note that the minimal contribution from Type Ia supernovae for the iron-peak elements is because only XX of them have exploded by the end of our 500~Myr simulation.}
% GB(AE): Why are H and He different? Either discuss or remove from plot.
%  ------ AE removed from plot
%
\label{fig:species_sources}
\end{figure}

The only physics that separates the dynamical evolution of these elements in our simulations is the individual sources of enrichment: AGB winds (stars less than 8~M$_{\odot}$), stellar winds (stars above 8~M$_{\odot}$), core collapse supernovae, and Type Ia supernovae. These channels differ in: 1) how long after a given star formation event they occur, 2), by consequence, the typical ISM properties in which they occur, 3) how much energy is associated with each event, which determines ejecta temperature and velocity. To understand where each of the elements in Figure~\ref{fig:species_fractions} originated, we show the mass fraction of metals produced through each channel in Figure~\ref{fig:species_sources}. Core collapse supernovae are responsible for over 90\% of the total metal enrichment in our galaxy, but this is clearly not true for all elements. In particular, a majority of N (74\%) and Ba (65\%) are ejected by AGB winds; both are retained at a higher rate than the rest of the metals. That this behavior exists for N and Ba in our simulations is dependent upon the metallicity of our galaxy and choice of stellar yields tables, but we can generalize this result to say that, for any assumed set of yields, low mass galaxies should more easily retain \textit{any} elements synthesized predominately in AGB winds, as compared to elements synthesized through supernovae.\footnote{One might expect that much of the N and Ba that is ejected by galactic winds is comprised mostly of the $\sim$30\% of each species produced through the stellar winds of more massive stars and supernovae. However, this cannot be determined as we lack Lagrangian information about individual gas elements.}
% AE: Actually if there are significant isotope variations between nucleosynthetic sources (i.e. N produced in AGB winds is a different isotope than the trace N produced in SNe) then that could be a very cool thing to distinguish if we had good gas-phase abundance measures of various isotopes... but maybe not if they aren't very stable....
 We discuss how this result may extend towards other metal yields that are not well sampled on our relatively short (500 Myr) simulation timescales in Section~\ref{sec:discussion:metal yields}.

AGB winds have low energy and velocities (10~km~s$^{-1}$) as compared to the energy and typical expansion velocities of supernovae ($\sim$1000~km~s$^{-1}$). In addition, their longer timescales (40-100~Myr as compared to $\sim$10~Myr) means that AGB stars are typically removed from their birth regions. The changes in typical ISM density and height of these events contributes to these variations. We show histograms of the average number density within 20~pc of any given enrichment source (top) and height above/below the disk (bottom) within 1~Myr of each event (as limited by our output cadence) in Figure~\ref{fig:spatial distribution}. As shown, SN peak at very low densities, indicating that most explode in superbubbles, regions carved out by previous SNe. AGB stars predominately release their metals close to the average ISM density. The scale height distributions for both events show no significant differences. We do not expect these differences to be the dominant effect in driving the differential evolution of elements ejected by AGB winds vs. supernovae, compared to the energetics, but can certainly play a significant role in determining the mixing behavior of individual enrichment events. The changes to mixing behavior as a function of ISM properties will be investigated in more detail in a future work.

\begin{figure}
\centering
\includegraphics[width=0.95\linewidth]{SE_density.png}\\
\includegraphics[width=0.95\linewidth]{SE_z.png}
\caption{The average gas number densities within 20~pc of a given event (top) and vertical position above/below the disk (bottom) within 1~Myr before the event.}
\label{fig:spatial distribution}
\end{figure}


\subsection{Mixing and Distribution of Metals in the ISM}
\label{sec:mixing}

Chemical evolution models make varying approximations as to what fraction and on what timescales newly formed metals populate a galaxy's ISM and become available to enrich future generations of stars. In the simplest models, new metals are instantaneously available upon the formation of new stars, ignoring delays due to stellar lifetimes (\textit{refs, but maybe not if included in introduction}). Accounting for delays due to individual stellar lifetimes, loss from galactic winds, and dilution from accretion of pristine gas leads to models that can more accurately capture a variety of mean chemical properties of galaxies with some ability to account for spreads in abundances. In hydrodynamics simulations, there has been substantial work understanding how metals, often characterized as passive scalar fields, diffuse in both turbulent boxes and global galaxy models. However, there is not  yet a complete model for how metals should mix and evolve in the complex, multi-phase ISM of real galaxies. To build towards this, we characterize the metal evolution and distributions in our galaxy below.

% Classic chemical evolution models often adopt the simplifying assumption that metals from newly formed stars are instantly deposited back into the ISM \textbf{(examples)}. With added complication, this instantaneous recycling approximation (IRA) can be expanded upon by taking into account individual stellar lifetimes (references) which have been shown to produce significant variations in stellar abundances (references) as compared to IRA models. However, even in this new model it is unlikely that whatever fraction of a star's yields that remains in the ISM instantaneously enriches star forming gas. Rather, these yields are first injected through a hot phase (stellar winds, H~{\sc ii} regions, supernovae), cooling over time into star forming gas. (refs for models accounting for this). We investigate this process in our simulations below, first by outlining a new analytic description of metal distributions in the ISM.

\subsubsection{A Functional Form for Metal PDFs}
\label{sec:log-normal}

The log-normal distribution is found often in nature, generally describing processes with non-negative values that grow with time. In astrophysics, for example, the log-normal distribution can be used to describe the time evolution of the star formation rate density \citep[see ][]{Gladders2013,Abramson2016,Diemer2017}. In addition, as expected from analytic theory \citep{Vazquez-Semadeni1994}, isothermal turbulence gives rise to log-normal density probability distribution functions \citep[PDFs;][]{Padoan1997, Passot1998, Ostriker1999,PadoanNordlund2002,KrumholzMcKee2005,Federrath2008}. Although these PDFs are only log-normal in simulations containing a more realistic, multi-phase ISM \citep{Scalo1998} if the disk is very stable \citep{WadaNorman2007}, individual phases within the ISM do exhibit some log-normality \citep{Tasker2009, Tasker2011,Joung2009,PriceFederrathBrunt2011, HopkinsQuataertMurray2012}. It has been shown that the 3D density PDF and the column density PDF, in both simulations and observations, have a characteristic shape. This includes a log-normal component, generated by multiplicative processes (shocks and the turbulent cascade in the ISM) and a power-law component at high densities arising from the additive combination of individual, self-gravitating cloud structures \citep{Vazquez-Semadeni1994,Burkhart2009, FederrathKlessen2013, Collins2012, Myers2015, Burkhart2017, Chen2017}.

The physics that drives the density PDF is directly related to the process of metal mixing and diffusion. However, there is no a priori reason why gas density and metallicity PDFs should have similar functional forms. We demonstrate here for the first time that the mass fraction PDFs for each metal species in our simulation can indeed be well fit using a piecewise log-normal + power-law PDF. We use a simple conceptual model in Section~\ref{sec:interpretation} to motivate the emergence of this distribution.

We follow \cite{Collins2012}, \citet{Burkhart2017}, and \citet{Chen2017} in constructing our piecewise PDF. We define the distribution of metals in the ISM as a function of the fraction of mass contained at a given metallicity, $Z$, or metal mass fraction, $Z_i$, where $i$ denotes an individual element. This PDF is given as
\begin{align*}
  p(Z) =
  \begin{cases}
    \frac{N}{\sigma Z \sqrt{2\pi}} \rm{exp}\left[-\frac{(\rm{ln}(Z) - \mu)^2}{2\sigma^2}\right],
    & Z < Z_{\rm{t}} \\
    % \multicolumn{1}{@{}c@{\quad}}{1} % variant with \multicolumn
    N p_o Z^{-\alpha},
    & Z > Z_{\rm{t}}
\end{cases}
\end{align*}
where $N$ is a normalization constant, $p_o$ ensures continuity between the two components, $\mu$ and $\sigma$ are the log-mean and width of the log-normal component, $\alpha$ is the power-law slope,  and Z$_{\rm t}$ is the metal fraction at the transition between the log-normal and power-law components. When fitting this PDF, we only enforce continuity and $\int_0^{\infty} p(Z) dZ = 1$, leaving $\alpha$, $\mu$, $\sigma$, and $Z_{\rm t}$ as free parameters. These conditions set $N$ and $p_o$ to
\begin{equation}
N = \frac{1}{2} \left[ 1 + \rm{erf}\left( \frac{ln\left(Z_{\rm t}\right) - \mu}{\sigma\sqrt{2}}\right)\right] + \frac{p_o}{\alpha-1}Z_{\rm t}^{-\alpha + 1}
\end{equation}
and
\begin{equation}
p_o = \frac{1}{\sigma Z_{\rm t} \sqrt{2 \pi}} {\rm exp}\left[-\frac{({\rm ln}(Z_{\rm t}) - \mu)^2}{2 \sigma^2} + \alpha {\rm ln}(Z_{\rm t}) \right]
\end{equation}.

We compute the numerical metal mass fraction PDFs for each metal species in our simulation using a fixed bin with of 0.05 dex. We fit $p(Z)$ to each of these using a Levenberg-Marquardt algorithm as implemented in \textsc{SciPy} \cite{SciPy}, stepping through possible values for $Z_{\rm t}$, set to the centers of each of these bins. The best of these fits is then compared to best fits using only a log-normal component or only a power-law component, and the best of these three is accepted. The log-normal + power-law PDF produces the best fit in nearly all cases.

We show in Figure~\ref{fig:log-normal} the numerical PDFs (solid histograms) and log-normal + power-law fits (dashed lines) across individual gas phases. These PDFs have been computed at an arbitrary single point in time, at 270 Myr. For clarity, we only show a subset of the 15 elements we follow. As shown, there are clear differences in the PDFs across elements of different nucleosynthetic origins and between each phase. However, each of these distributions are characterized by a power-law tail towards high metal fractions and a turnover of varying width at low metal fractions. We discuss the differences among each phase in more detail in the next section, but note here that the significance of each of these components varies notably across phases. In many of these cases, the piecewise log-normal + power-law distribution fits the numerical PDF quite well. However, there are often deviations, particularly at low metal fractions, from pure log-normal behavior. This manifests as either a very broad, flat PDF at low metal fraction (see N and Ba, for example), or large peaks not well described by $p(Z)$ at low metal fraction. In these situations, it is unclear what, if any, portion should be considered as a log-normal, or if there are multiple components within this region.

We argue here that the log-normal + power-law PDF can be a powerful tool for modelling the the metal fraction  PDFs of individual elements in galaxy models. The fits are not uniformly perfect but some deviation from a simple analytic model is expected in a complex, multi-phase ISM. In addition, some of this deviation, particularly in the WNM and WIM, could be caused by grouping together qualitatively distinct gas in a single phase; this would tend to broaden the distributions. Most importantly, however, the PDFs of the CNM, are indeed well fit by the adopted $p(Z)$. As this is the source of star forming gas in the ISM, the log-normal + power-law PDF appears useful to account for intrinsic scatter in stellar abundances in galaxy evolution models.

\begin{figure*}
\centering
\includegraphics[width=0.95\linewidth]{DD0390_element_by_element.png}
\caption{The numerical PDFs (solid histograms) and the associated log-normal + power-law fits (dashed lines) for a subset of the elements tracked in our simulation in each of the four gas phases defined in Section~\ref{sec:ejection}: CNM (blue), WNM (green), WIM (purple), H~{\sc i}M (red), and all the gas in the ISM (black). For clarity, each distribution is normalized to the mode of the full-disk PDF (black) and is centered on the median value of the full-disk PDF. We note the vertical axis normalization is such that integrating over the shown PDF gives the mass fraction of that phase in the disk. Since the CNM dominates the mass fraction of our galaxy, the black curve is often obscured at low metal fractions.}
\label{fig:log-normal}
\end{figure*}

\subsubsection{PDF Variation Across Gas Phase}
\label{sec:phase-pdfs}

The various phases represented in Figure~\ref{fig:log-normal} involve variations in density and temperature of more than six orders of magnitude. The evolution of each phase is qualitatively different, and the metal mixing behavior of each should vary. Mixing timescales over a given length scale should be related to the local sound speed; hot gas, with higher sound speeds, should mix more rapidly than the dense, disconnected clumps of cold gas in the ISM. In addition, one would expect a metallicity gradient with gas temperature as enrichment occurs first in the hot phases, cooling and enriching denser gas over time. We examine the PDF variations among elements in each phase, as shown in Figure~\ref{fig:log-normal}, in more detail here.

Unsurprisingly the diffuse H~{\sc i}M contains the most metal enriched gas, as it is comprised predominately of metal enriched supernova ejecta. Clearly this leads to long power-law tails towards high metal fractions for each species, with a very narrow, poorly defined peak at low metal fractions. Generally, in colder gas the metal PDFs become less enriched with broader low metal fraction components and steeper, power-law tails. Although the extended power-law tail of the H~{\sc i}M leads to a large range of metal fractions, the H~{\sc i}M represents very little mass, and this tail represents recent, un-mixed enrichment. The comparatively narrow width in the log-normal regimes of CNM is perhaps surprising. Although this gas is at lower metal fraction than the WNM and WIM, it would appear that it is more well mixed. This runs counter to the idea that mixing times should be long in colder gas unless mixing first proceeds rapidly in the hot phases before mixing in with cold, disconnected structures across the galaxy.

Although these trends across phases hold for all elements, there are qualitative differences between elements, particularly between those ejected predominantly by AGB winds (e.g. Ba and N) and those ejected by supernovae (e.g. O and Mg). In all phases, except the H~{\sc i}M, the AGB-wind elements have broader distributions that are less well described by our adopted $p(Z)$ than the metals dominated by supernova enrichment. The power-law component of N and Ba is generally shallower than all other metals across each phase (except the H~{\sc i}M), particularly in the CNM. Ba and N do not show significant differences among the rest of the metals in the H~{\sc i}M, though this is likely because the Ba and N present in this phase is dominated by the Ba and N produced in supernovae yields. Again these differences between yield sources could be driven both by differences in their enrichment timescales and therefore differences in the typical ISM environment each event encounters, and the differences in energetics between AGB winds and supernovae. AGB wind elements enrich the WNM and WIM directly, rather than the H~{\sc i}M, which would increase their mixing timescales and broaden their PDFs.

\subsubsection{The Time Evolution of Metal PDFs}
\label{sec:statistics}
We focus on the time evolution of the full numerical PDFs in this section. Our results here do not depend upon the choice of functional form for $p(Z)$. Figure~\ref{fig:phase-statistics} shows the evolution of four different statistics for the O (top) and Ba (bottom) PDFs. These two elements are treated as representative elements for supernova and AGB wind production respectively. The difference between the mean and median masses of the PDF (second column) is a measure of the skew of the PDF. Positive values indicate that metals are preferentially sequestered in metal-rich gas, and are less well mixed throughout the given phase. The skew is always positive in these distributions.

\begin{figure*}
\centering
\includegraphics[width=0.95\linewidth]{O_Ba_distribution_evolution.png}
\caption{Time evolution of three different statistics for the full distributions of O (top) and Ba (bottom) in each phase of our simulation. The panels show log$_{10}$ of the median (left), the difference, in dex, between the mean and the median (middle), and the 90$^{\rm th}$ decile and 10$^{\rm th}$ decile difference in dex (right).}
\label{fig:phase-statistics}
\end{figure*}

For O, the median mass fraction is ordered by phase temperature. The H~{\sc i}M is significantly more enriched than the cooler phases by anywhere from 0.1 dex to 4 dex, fluctuating by $\pm$ 1 dex over the simulation time. The frequent large skew in the H~{\sc i}M (see second panel) and spread in the H~{\sc i}M (right two panels), coupled with its continual fluctuation, suggests that the H~{\sc i}M is not in equilibrium.
%This phase is poorly mixed because it is short lived, either cooling into different phases of the ISM or being ejected from the galaxy, and is continually enriched by new SN events.
Each cooler phase in O is progressively less enriched (lower median), with smaller skew and spread. The offset between phases and increasingly well mixed gas from hot to cold indicates that metal enrichment in the ISM of galaxies proceeds first through mixing on large scales in a hot phase, before progressively cooling through multiple phases until enriching star forming gas. This is the only explanation for how the cold gas can rapidly homogenize over the whole galaxy within $\sim$50 Myr, roughly when the cold phase exhibits a nearly constant spread (right panels). Individual enrichment events will have significantly higher metallicities than the ambient ISM in any phase, and will drive an increase in the difference between the mean and median of the PDF. These can be seen as the obvious spikes in the H~{\sc i}M and WIM. For O, the lack of these spikes in the two cold phases suggests again that enrichment does not occur directly in these phases, but proceeds more gradually through the warmer phases first.

% In each phase, enrichment events will tend to drive larger differences between mean and median mass fraction,

Although these trends are generally true for Ba, its evolution is much more complicated.
%In addition, the H~{\sc i}M is much less offset from the rest of the phases than is seen in O, with much less variability.
Unlike O, the spread and skew in Ba for the CNM, WNM, and WIM \textit{increase} in during the evolution, reaching differences in 90$^{\rm th}$ and 10$^{\rm th}$ percentiles of nearly 2 dex in the CNM and over 2 dex in the WNM and WIM. The H~{\sc i}M is seemingly unaffected by this trend, and simply fluctuates throughout the simulation. Given their lower energies, AGB winds more directly enrich the WNM and WIM, not the H~{\sc i}M as in supernovae. The consequence of this is clear in Figure~\ref{fig:log-normal} by the wider PDFs and longer power-law tails in Ba in these phases. These tails represent the most recently enriched gas, and is clearly much more locally confined than O.

%{\bf (move to discussion?)}:
The positive skew in all of the PDFs presented here implies that most of the mass of the galaxy has a metal fraction below what one would normally adopt as the average metallicity (i.e. the ratio between the total mass of metals and the total mass). Assuming star forming gas follows the same properties as the cold gas, this distinction is small ($\sim$ 0.2 dex), but significant, for elements released during SNe explosions, but can be very significant, up to $\sim$ 0.8 dex, for elements released in AGB winds. Chemical evolution models, especially one-zone models, follow the mean metal fraction, not median. These results indicate that these models are biased to overestimate gas and stellar elemental abundances.
. % However star formation may not be unbiased, and may preferentially come from gas sitting on the higher metal fraction tails.

To summarize, this figure demonstrates: 1) there are qualitative differences in how supernova injected elements (e.g. O, Mg) and AGB wind injected elements (e.g. N, Ba) are distributed through the ISM, with the latter having a broader range of variation and being less well mixed in all phases except the H~{\sc i}M, 2) hotter phases are more metal enriched, both because the cooler phases make up most of the initially unenriched mass of the ISM, and the hotter phases are more directly populated by recent enrichment events, 3) the cooler, denser phases, particularly for supernova injected elements, are more well mixed than the hot phases of the ISM, 4) the PDFs of metal mass fraction are best fit by a log-normal + power law function, and therefore 5) the median metallicity available for star formation rates lies below the mean galactic value. In the case of supernova injected elements, enrichment proceeds quickly through the H~{\sc i}M over the entire galaxy. For AGB injected elements, enrichment proceeds through the WIM and WNM, leading to longer mixing timescales and larger metal fraction variations in the ISM.

\section{Discussion}
\label{sec:discussion}
We begin with a simple toy model that motivates the power-law tail at high metal fractions of the metal fraction PDFs and a subsequent turnover at low metal fractions in Section~\ref{sec:interpretation}. This work is placed in context with previous works focusing on metal mixing in the ISM in Section~\ref{sec:context}. Finally, in Section~\ref{sec:stellar abundances} we discuss how these results relate to stellar abundances, make generalizations to more massive galaxies in Section~\ref{sec:massive galaxies}, and discuss possible impact of these results on chemical enrichment from more exotic nucleosynthetic sources in Section~\ref{sec:exotic enrichment}.

\subsection{Physical Interpretation of the PDF}
\label{sec:interpretation}
Take the simple case of an initially primordial, uniform, isothermal medium of mass $M_o$ with initial metallicity $Z_o = 0$, containing a single, un-mixed enrichment event of mass $M_{\rm ej}$ whose size is small compared to the system and mass $M_{\rm ej} / M_o \ll 1$. Thus the background medium represents a virtually inexhaustible (but finite) source of un-enriched gas. In this case, $p(Z)$ takes the form of a double-delta function
\begin{equation}
p(Z) = \delta(Z) + \frac{M_{\rm ej}}{M_o}\delta(Z - Z_{\rm ej}).
\end{equation}
If the enriched material mixes continually with the primordial gas at a constant rate, after some time this gas will have mixed with an equal amount of primordial gas. At this point, the high metal fraction delta term becomes $\frac{2M_{\rm ej}}{M_o} \delta(Z - \frac{1}{2}Z_{\rm ej})$. This implies an inverse relationship between the mass of enriched gas and the metallicity of the enriched gas, and thus a power-law evolution in time of the high metallicity portion of $p(Z)$ with slope $\alpha = 1$. If the gas is continually enriched by identical, well-separated enrichment events of mass $M_{\rm ej}$ and metallicity $Z_{\rm ej}$, $p(Z)$ will become a power-law in $Z$, truncated at some time dependent minimum $Z$. The slope of the power-law is determined by the rate of injection versus mixing with the ambient medium. A power-law index $\alpha < 1$ can occur when injection occurs more rapidly than the newly enriched gas can mix with the ambient medium. Steeper power-laws, $\alpha > 1$, develop when mixing occurs more rapidly than enrichment. If the ambient, primordial gas were truly infinite, the power-law would never completely encompass the delta function at $Z = Z_o$.

In reality, however, $M_o$ is not an inexhaustible reservoir of primordial gas. Eventually the entire ambient medium will become enriched to some $Z > Z_o$, and $p(Z)$ will consist entirely of a continuous, truncated power-law. As enrichment proceeds, gas near the low metallicity truncation of $p(Z)$, which still comprises much of the mass of the system, enriches towards higher metallicities in the power-law tail. This will produce a turnover at the low $Z$ limit of $p(Z)$. The low-turnover limit is produced by diffusion from many different sources, and is thus an additive process, which, as we noted above, tends to produce log-normal distributions. The physical interpretation of these two components is that the power-law tail represents newly enriched and poorly mixed gas that is above the average gas metallicity and is undergoing dilution, while the log-normal component represents the ambient medium that lies below the average gas metallicity and is undergoing enrichment.

This toy model provides a physical intuition for the general trends in the PDFs presented here across ISM phases. The H~{\sc i}M is comprised predominantly of gas from individual enrichment events that, due to its low density, easily create large power-law tails beginning at very high metal fractions. Whatever ambient component of the H~{\sc i}M that exists is well mixed, leading to a narrow and sub-dominant log-normal component at low-metallicities. Cold, dense gas, which is almost never directly impacted by these individual enrichment events, is enriched almost entirely by diffusion, and thus has an almost completely log-normal PDF with very little, if any, power-law tail.

%
%
%
%Although this model is useful to explain these very general features, it is certainly incomplete in practice. Enrichment proceeds irregularly through the mulit-phase ISM with a variety of ejection masses and metallicities. Phases are not independent, and the hot phase cools and enriches cooler phases in the ISM, distorting $p(Z)$ within each phase. Cold gas is heated up and destroyed through feedback, changing the content of the low-metallicity end of the warm and hot phases. Finally, gas is removed from the galaxy through feedback driven winds that preferentially remove metal enriched gas. Each of these processes has a non-trivial effect on the evolution of the PDF that will cause deviations from the simple toy model outlined here. Determining the exact physical processes that drive and set the properties of each PDF is the subject of ongoing work.

%
% A.E. paragraph may need work
%
Mixing timescales are likely proportional to the eddy turnover times at the injection scale \citep{PanScannapieco2010, Colbrook2017} and the properties of turbulence in the ISM \citep{YangKrumholz2012}. Mixing within a phase is likely dependent upon the phase's sound speed and velocity dispersion. This would imply rapid mixing timescales in the WIM and H~{\sc i}M, with typical velocity dispersions of $\sim$~30 km~s$^{-1}$ and $\sim$~100~km~s$^{-1}$ respectively, and long mixing timescales in cold, dnese gas ($\sim$ 1 km s$^{-1}$). Our results show, however, that in general the WIM and H~{\sc i}M are the least well mixed, while the CNM is the most well mixed across the galaxy. This is particularly curious as the sound crossing time of the H~{\sc i}M across the galaxy ($\sim 1$ kpc) is $\sim$10~Myr, compared to $\sim$~1~Gyr for the coldest gas. It must be that the H~{\sc i}M is far from equillibrium throughout the simulation, in part due to the continual enrichment by ongoing supernovae. Newly enriched gas must mix through the H~{\sc i}M on galaxy scales, becoming well mixed and less enriched by the time it cools into the CNM and eventually star forming gas. Since elements produced in AGB winds do not directly enter the H~{\sc i}M, mixing is less efficient driving larger variations across the galaxy. The idea of metals processing first through the hot ISM has been proposed before to explain observed metallicity trends in the outskirts of more massive galaxies \citep{Tassis2008,Werk2011}.

\subsection{Context}
\label{sec:context}
That metal fraction distributions can be described using a simple analytic log-normal + power-law PDFs, even in a complex, multi-phase ISM, has not been demonstrated prior to this work.   In addition, this is the first work to demonstrate differential mixing behavior of individual metal species using 3D hydrodynamics simulations. However, there does exist a significant body of work investigating the mixing behavior of passive scalars in a variety of contexts, as discussed below. \cite{KrumholzTing2018} predict differential behavior for AGB wind and core collapse supernova synthesized elements as a direct consequence of the differences in size of typical planetary nebulae ($\sim$ 0.1~pc) and supernova remnants ($\sim$ 100~pc).
%We discuss our results further in the context of previous research on metal mixing in the ISM and speculate on the physical processes that drive the evolution of these PDFs across phases.

Previous work on the evolution of metals in the ISM varies from studies concerning the advection of passive scalars in idealized turbulent boxes \citep[e.g.][]{Pope1991, PanScannapieco2010, PanScannapiecoScalo2012, PanScannapiecoScalo2013, YangKrumholz2012, SurPanScannapieco2014, Colbrook2017}
% cite passive scalar reviews by Warhaft 2000 in fluid mech and scalo + Elmegreen 2004?
to global galaxy models studying generalized advection and mixing of passive scalars \citep[e.g.][]{deAvillez2002,Petit2015,Goldbaum2016} to models with more detailed self-consistent metal enrichment \citep[e.g.][]{Revaz2009,Escala2018}.
% summarize conclusions from these works:

The evolution of metallicity PDFs has been investigated previously in some of these works \citep[see ][ and references therein]{PanScannapiecoScalo2012,PanScannapiecoScalo2013}, with effort towards developing closure models to describe the evolution of the PDFs of passive scalars in turbulent media \citep[e.g.][]{EswaranPope1988,Chen1989,Pope1991}. The astrophysical context in much of this work was enrichment from the first stars, so the focus was on the low-metallicity tail of the PDF and the timescales over which gas is polluted \citep[e.g.][]{PanScannapiecoScalo2013,Sarmento2017}. These works often use isothermal turbulent-box simulations initialized with a double-delta function PDF of pristine gas and enriched gas in some varying spatial distribution, ignoring ongoing enrichment. Initial PDFs of this form were demonstrated some time ago to evolve into a Gaussian distribution at late times \citep{EswaranPope1988}, but it is unclear if these could also be described with a log-normal distribution. However, these works uniformly do not contain the high metallicity power-law tails shown in our work. As suggested by our toy model in Section~\ref{sec:interpretation}, this is due to the lack of ongoing enrichment in these studies.

More detailed models of global galaxy evolution have generally focused on spatial correlations and mixing timescales of initially asymmetric fields, without concern for ongoing enrichment \citep[e.g.][]{deAvillez2002,Petit2015} or focused primarily on the evolution of stellar abundance patterns and metallicity distribution functions \citep[e.g][need more]{Jeon2017,Hirai2017,Escala2018}, without directly examining the evolution of gas-phase metallicity PDFs.

%
% MM: remove from this section:
%This is the first work examining the gas-phase properties of metallicity PDFs in a global galaxy simulation with a detailed model for stellar feedback, chemical enrichment, and a multi-phase ISM. Since we are far from a complete understanding of galactic chemical evolution, investigating the physics behind what sets the statistical properties of these PDFs, particularly what drives the differences across metal species and phases, is important. Although this is beyond the scope of this current work, we use scaling arguments to speculate on the meaning of these results. %Metal mixing in the ISM is a diffusive process driven by turbulence.

% Recent works investigating dwarf galaxy chemical enrichment with Lagrangian simulations have shown that the sub-grid diffusion models required to reproduce observed dwarf galaxy abundance properties imply fairly rapid mixing timescales \citep{Escala2018,Hirai2017}. In contrast to this work, \cite{Escala2018} finds dwarf galaxy
%The sub-grid diffusion models in Langrangian simulations of dwarf galaxy chemical enrichment that match the abundance properties of observed dwarf galaxies imply rapid mixing timescales

%
%
%  Recent works specifically examining metal mixing in dwarf galaxies. Saving for potential future discussion:
% \cite{Hirai2017} uses constrains the diffusion parameter in their SPH simulations comparing their simulated dwarf galaxies to observations in [Mg/Fe] vs. [Fe/H] and [Ba/Fe] vs. [Fe/H] space. Suggests Ba/Fe is more powerful constraint and briefly discusses potential differences in mixing between elements but does not do work with this
% \cite{Jeon2017}
% \cite{Escala2018} tests of role of metal diffusion models in producing realistic MDFs in dwarf galaxies. Able to reproduce things like MDF width and [alpha/Fe]vs[Fe/H] scatter.

% The properties of log-normal density PDFs in isothermal simulations are governed by the statistical properties of turbulence (refs) and the relative importance of compressive and solenoidal driving. However, the density PDFs in simulations of global galaxies with a resolved, multiphase ISM show significant departures from log-normallity \citep[see][]{Hopkins2013} but are still approximately log-normal within individual phases of the ISM \citep{HopkinsQuataertMurray2012}. The properties of the density PDF in multiphase simulations have not been well studied which offers little insight into what sets the mass fraction PDFs for each element. We speculate on this here to motivate future study, but note that a detailed characterization is beyond the scope of this work.



% The log-normal density PDFs found in ISM simulations have a variance that is proportional to the mach number \citep[e.g.][(maybe more)]{PNJ1997, Passot1998, Fedderath2008, LemasterStone2009}. In our simulations, however,




%We discuss possible caveats to this in Section~\ref{sec:caveats}. Does this extend to more massive? how does DM potential affect this? Are general trends still true in much larger galaxies? or are things very different once H~{\sc i}M crossing times are more like 100 Myr (aka MW size galaxies).

\subsection{Timescale Dependence of AGB Ejecta}
\label{sec:discussion:metal yields}
We can generally say that metals produced in AGB winds evolve qualitatively differently in dwarf galaxies than metals produced through supernovae. However, exactly which metals exhibit these differences, and to what degree, is timescale and metallicity dependent. Short-timescales ($\lesssim 100$~Myr) only samples enrichment from the most massive AGB stars, which stars on order of a few solar masses enrich on 1 Gyr timescales or longer. In our simulations, which only sample 500 Myr of evolution, N and Ba are dominated from AGB wind ejecta. C is commonly used to track AGB wind enrichment, but as it originates from lower mass AGB stars, C enrichment operates on $\sim$ Gyr timescales, longer than we follow in this work. This additionally varies with metallicity. Sr, for example, is only significantly ejected through AGB winds at our metallicity on $\sim$~Gyr timescales; stellar winds from more massive stars dominate the production of Sr on shorter timescales. At higher metallicity, much more Sr is produced through massive AGB stars, decreasing the timescale over which Sr should exhibit a differential chemical evolution as compared to supernovae ejected elements.

To illustrate some of these differences, Figure~\ref{fig:agb evolution} gives the fraction of a given metal ejected through AGB winds relative to the total amount of that metal produced in a single-age stellar population with a metal fraction of $Z = 10^{-4}$ at four different times. This model was run using \textsc{SYGMA} \citep{Ritter2017}. As shown, metal enrichment from AGB winds only begins to dominate for a given species at $\sim$ 100 Myr. This includes N and Ba, as followed in our simulations, but this plot indicates that Li should show the strongest differences with respect to metal ejected through supernovae, even on relatively short ($< 100~$~Myr) timescales. It takes over a Gyr for C to be dominated from AGB wind ejecta, which is the case for many of the elements shown. F, which shows very little contribution from AGB winds at short timescales is dominated by enrichment on longer timescales. These dependences on metallicity and timescales certainly adds complications in generalizing our work and in interpreting observations in the context of the results presented here, but these difference could be leveraged to better understand galactic chemical evolution on multiple, distinct timescales. Clearly this motivates future work covering Gyr timescales to fully understand how metal mass fraction PDFs evolve with time.

\begin{figure*}
\centering
\includegraphics[width=0.95\linewidth]{Half_AGB_Fraction_elements_s0.png}
\caption{The fraction of a given metal ejected through AGB winds at various times for a model of a single-age stellar population at $Z = 10^{-4}$ metal mass fraction, with no continual star formation. We only show a sample of some of the elements dominated most by AGB enrichment. The horizontal line marks 50\% contribution.}
\label{fig:agb evolution}
\end{figure*}


%\begin{figure*}
%\centering
%\includegraphics[width=0.95\linewidth]{C_N_evolution.png}
%\caption{The fractional contribution of AGB winds to the total yields of C and N as a function of time for a simple stellar population at three different metallicities. Since the nucleosynthetic source governs the mixing dynamics of a given metal, this plot illustrates how this can change with both time and metallicity for a given species. These models were generated using \textsc{SYGMA} \citep{Ritter2017-sygma}}
%\label{fig:agb evolution}
%\end{figure*}


\subsection{Impact on Stellar Abundance Patterns}
\label{sec:stellar abundances}
% AE: Possibly need better general introduction to CE models / sims and importance of understanding stellar abundances (obs+theory)
In order to better understand the physics driving stellar abundance patterns and distributions it is important to characterize the chemical evolution of star forming gas in our simulations. This could be used to help disentangle sources of scatter in observed stellar abundance, including radial/azimuthal abundance gradients and stellar migration, redshift evolution, asymmetric accretion of pristine gas, and the intrinsic scatter in ISM abundances. If the star forming gas in our simulations can also be well-fit by a log-normal + power-law PDF, and if we can parameterize their evolution as functions of global galaxy properties, this can be used as a powerful and analytic tool for modeling stellar abundance patterns in semi-analytic models. This would be a physically motivated way to account for both the intrinsic spread in stellar abundances due to inhomogeneities in the ISM and variations in mixing for different metal species. We reserve an analysis of these distributions in abundance space in both the gas and stars for a future work. However, we briefly discuss the connection to star forming gas and stellar enrichment below.

Unfortunately, there is insufficient star forming gas at any one time in these simulations to construct a PDF of metal distributions. However, star forming gas originates from the CNM, so it is reasonable to expect that this gas, and therefore stars themselves, follow a similar PDF to the CNM. To verify this, Figure ~\ref{fig:stars} shows the difference between the oxygen mass fraction of stars and the median mass fraction from the CNM PDF at the time that particle formed. The large scatter at early times ($<$ 120~Myr) is a result of the early enrichment phase, when the initial gas oxygen mass fraction was zero. Stars seem to be sampled evenly around the median of the CNM distribution, with only a slight bias (52\% of stars) towards values below the median. However, for stars formed after the initial phase, the median separation from the CNM median is 0.31~dex, and can reach up to $\sim$~0.5~dex. Though a significant deviation, this is smaller than the typical IQR of the CNM (see Figure~\ref{fig:phase-statistics}). Additionally, if we further subdivide the CNM by density, the metal fraction PDFs narrow and tend towards the median value as a function of increasing density. High density gas, from which star formation occurs, is not biased towards higher metallicity in cold gas.

%

\begin{figure}
\centering
\includegraphics[width=0.95\linewidth]{stellar_CNM_distance.png}
\caption{The separation (in dex) of each star's oxygen fraction from the median value of the CNM oxygen mass fraction PDF for the within 1~Myr (our time resolution) of each star's formation time. The median deviation is given in the plot for stars formed after the initial star formation and enrichment period (120~Myr.)}
\label{fig:stars}
\end{figure}

\subsection{Do these results apply to more massive galaxies?}
\label{sec:massive galaxies}
One important caveat about our work is that these results are derived from simulations of an isolated, low mass dwarf galaxy whose properties vary dramatically from a more massive galaxy like the Milky Way. It is unclear how much of these results apply to more massive galaxies given their deeper potential wells and higher star formation rates. Feedback driven galactic outflow properties do vary significantly with halo mass \citep[e.g.][]{MacLowFerrara1999,Muratov2017}, as more massive galaxies more easily retain and re-accrete gas. We expect the results presented in Figure~\ref{fig:species_fractions} to be the most susceptible to the particular star formation history and halo depth of a given galaxy. With gas from individual enrichment events more easily contained, we would expect the retention fractions to be more similar across metal species with increasing halo mass or decreasing star formation rate. If true, this difference in metal retention can be a key observable in verifying our result that the dynamical evolution of metals in the ISM is not uniform. If massive galaxies retain metals equally, we would expect dwarf galaxies to exhibit larger abundance ratios between AGB wind elements and SN enriched elements than stars in more massive galaxies, like the Milky Way. This difference would be greater at later times (higher metallicity) once AGB enrichment becomes significant. For example, at metallicities of [Fe/H] $\gtrsim$ 1, Ba is predominantly produced via s-process and ejected in AGB winds \citep{Travaglio1999,Travaglio2004}. At this metallicity, we would expect an elevated [Ba/$\alpha$] at fixed [$\alpha$/H] for low mass dwarfs as compared to the Milky Way. Indeed, this is found in the dSph's of the Milky Way \citep[see ][]{Tolstoy2009} and has been argued to either be the result of variations in s-process enrichment in metal poor environments and/or the result of enrichment timescale differences given the different star formation histories of dwarf galaxies and the Milky Way. In addtion, recent observations find elevationd [Ba/Fe] vs. [Fe/H] in local dwarf galaxies \citep{DugganKirbyAAS},
% AE TODO: Update duggand and kirby if / when the paper comes out
which could possibly be explained by this effect, but could also point towards significant Ba enrichment from NSMs. Given the lack of definitive explanation, clearly generalizing our results and detailing their observational consequences is a valuable area of future research.

In contrast, we expect the properties of the enrichment PDFs, and the variations between AGB enriched elements and supernova enriched elements (as presented in Figures~\ref{fig:log-normal} and ~\ref{fig:phase-statistics}), to be general. We expect metals to exhibit log-normal + power-law distributions in the ISM of all galaxies, with similar trends with enrichment source and gas phase as outlined here. However, the detailed properties of the PDFs (e.g, log-normal width, power-law slope) likely depend non-trivially on global galaxy properties. How these results vary as a function of galaxy properties will be investigated in future work.

% Talk about CGM at all???

%\textit{In addition, hot gas in a galaxy with a deeper potential well is less likely to be ejected from the galaxy, allowing more time for homogenization and possibly smaller metal fraction variation than seen here.}.

%
% moved to possible to-do list below,
%
\subsection{Implications for Exotic Enrichment Sources}
\label{sec:exotic enrichment}
We have shown that the dynamical evolution of metals depends upon their nucleosynthetic source, focusing here on AGB synthesized elements and supernova synthesized elements. However, these differences should also apply to exotic enrichment sources, such as hypernovae (HNe), neutron-star neutron-star mergers (NSM), and neutron star - black hole mergers. These sources have energies that differ significantly from typical supernova (10$^{51}$ erg), reaching $> 10^{52}$ for HNe \citep{Nomoto2004} and $\sim$10$^{49}$~erg for NSMs \citep{Goriely2011}. We would therefore expect different mixing behaviors for metals synthesized through these channels as compared to supernovae. Based on our results here, we would expect elements from HNe to be more well mixed in the ISM of all galaxies, but more readily ejected in dwarf galaxies, as compared to elements from SNe. Elements from NSMs, a significant source of r-process enrichment, would be less well-mixed in the ISM of all galaxies and more easily retained than SNe elements in dwarf galaxies. These differences could provide important signatures for distinguishing individual, exotic enrichment sources from observed stellar abundance patterns in our own Milky Way and in nearby dwarf galaxies. In particular, low mass dwarf galaxies with unusual r-process enrichment \citep[e.g.][]{Ji2015}, are valuable for constraining the source of these elements, their frequency, and typical yields. The differential metal evolution presented in our work both opens up an additional avenue by which elements from distinct nucleosynthetic sources may be distinguished in observations and challenges current assumptions used in interpreting these observations.



%\subsection{Extra things I could discuss:}
%\textit{Things I can / will additionally discuss}
%\begin{itemize}
%\item Observed N/O in stellar abundances exhibits large spreads at low metallicity. Argument has been that this is driven by timescale differences in AGB vs. massive star nucleosynthesis. However, our results would predict that N/O have a large spread due to mixing efficiency diffein rences between AGB winds and SNe.
%\end{itemize}

\section{Conclusions}
\label{sec:conclusions}
For the first time we present a detailed analysis of galactic chemodynamics and metal mixing on an element-by-element basis in a low mass dwarf galaxy with hydrodynamics simulations that simultaneously capture multi-channel stellar feedback in detail with a multi-phase ISM. This high resolution simulation, coupled with our star-by-star modelling of stellar nucleosynthetic yields, has allowed us to analyze how individual metal species couple to the galactic winds and ISM of this galaxy. We find that individual metal species do not share the same dynamical evolution, with differences directly related to nucleosynthetic origin (AGB winds, winds from massive stars, core collapse supernovae, and Type Ia supernovae). This difference is most significant in our model between elements ejected predominately through AGB winds and those ejected predominantly by core-collapse supernovae. In addition, we find for the first time that the metal mass fraction PDFs of each metal in the ISM can be described using an analytic piece-wise log-normal + power-law PDF. The properties of these PDFs vary with both ISM phase and metal species, again driven mostly by differences in enrichment sources.

We summarize our results as follows:
\begin{itemize}


%\item Elements from lower energy enrichment sources (i.e. AGB winds) are preferentially retained in the ISM of low mass dwarf galaxies as compared to those from higher energy sources (supernovae), which are more readily launched in galactic winds. This result is likely sensitive to global galaxy properties, becoming less significant with increasing halo mass. We predict the abundance ratios between AGB wind dominated metals (s-process elements) and SN dominated metals ($\alpha$ elements) to be higher at fixed metallicity in low mass dwarf galaxies than in more massive galaxies, like the Milky Way.

\item Power-law tails in metal fraction PDFs are a natural consequence of ongoing chemical enrichment, with the power-law slope related to the rate at which mixing dilutes newly enriched gas.

\item Hotter phases have metal fraction PDFs that are more enriched with significant power-law tails as compared to cold phases, which have a more dominant log-normal structure. The lack of significant tails in the cold-phase PDFs indicates that metal mixing occurs rapidly in hotter phases before cooling into dense gas. % missing text???????????

\item Metal outflow in low mass dwarf galaxies depends upon nucleosynthetic origin. Metals from lower energy enrichment events (e.g. AGB winds) are preferentially retained in the ISM as compared to those from higher energy events (e.g. SNe). The degree to which this is true likely depends upon global galaxy properties such as star formation rate, dark matter potential well, and gas geometry.

\item Likewise, metals originating in AGB winds are less well mixed in the ISM compared to metals injected through SNe, with spreads of over 1 dex in cold gas, as compared to

\item Metal distributions exhibit positive skew, such that the median metal fraction can be anywhere from 0.1 to 1.0 dex above the mean metal fraction. Simple chemical evolution models, which generally follow the mean abundance, that are unable to account for complex metal mixing physics are likely biased towards higher enrichment values.


\end{itemize}

We expect these results to be general for more massive galaxies and galaxy properties. However, it may be the case that metals couple more uniformly to galactic winds in more massive galaxies as wind driving occurs increasingly through entrainment of the ambient ISM rather than the ejection of individual enrichment events. This warrants further examination. But interpreting these results for low mass dwarf galaxies, we expect that: 1) s-process elements from AGB winds should exhibit larger spreads than $\alpha$ elements, particularly at lower metallicities, 2) these elements should be overabundant in dwarf galaxies at fixed age, as compared to massive galaxies like the Milky Way, and 3) that r-process enrichment from NS-NS mergers should behave similarly to s-process elements from AGB winds, as compared to $\alpha$ elements.

%\section{Things to cite}
%Hou, Yu, Lu 2014: Need to read closer, but CE model playing around with SN energy feedback effectiveness finds that weaker feedback leads to more metal enrichment, higher FB, lower metallicities. Suggestion of importance of FB driving CE.



\acknowledgments
Acknowledgments.

\facilities{Yeti, CCA cluster, Stampede, Pleiades}
\software{Numpy, Enzo, yt, astropy}

\bibliographystyle{yahapj}
\bibliography{msbib}





\appendix
\renewcommand\thefigure{\thesection.\arabic{figure}}
\setcounter{figure}{0}

\section{Density PDF}
\label{appendix:density PDF}
We show the density PDF in Figure~\ref{fig:density_pdf} to illustrate our result in comparison to comparable works that have computed the density PDF in global galaxy simulations with a multi-phase ISM \citep{Joung2009, Tasker2009, Tasker2011, Tasker2015,HopkinsQuataertMurray2012}. We show the mass-weighted PDF ($dm/Md\rm{log}n$) on the left, and the volume-weighted PDF ($dv/Vd\rm{log}n$) on the right. As has been demonstrated in previous work, the full density PDF (black) is not well described as a log-normal distribution. The mass-weighted PDF is broad and flat at low densities, with a large tail through to high densities. The volume weighted PDF is much better described as a multi-component power law. The other phases do also show some log-normality (more for the mass weighted PDFs than the volume weighted PDFs), but all exhibit power-law tails towards higher densities. These deviations from a log-normal may still be the result of grouping together qualitatively different types of ISM gas. Finally, \cite{Hopkins2013} suggests a different functional form for describing these PDFs across a range of idealized simulations, but it still may be insufficient to fully describe the density PDF in realistic galaxy simulations. Clearly, we are far from a general understanding of mass-weighted and volume-weighted density PDFs in the case of a turbulent, self-gravitating, multi-phase ISM.

\begin{figure}
\centering
\includegraphics[width=0.475\linewidth]{density_PDF.png}
\includegraphics[width=0.475\linewidth]{density_PDF_volume_weighted.png}\caption{The mass-weighted (left) and volume weighted (right) density PDFs of our dwarf galaxy at an arbitrarily chosen time of 250~Myr. The total distribution is given in black, sub-divided by the contributions of the individual phases in the ISM.}
\label{fig:density_pdf}
\end{figure}

\section{Resolution Comparison}
We perform a resolution test to confirm that the key results of this study are convergent, at least qualitatively. Given the variations in star formation rate, feedback effectiveness, and stochasticisty in our model, we do not expect exact numerical convergence in any one quantity. We conduct two lower resolution simulations with a maximum physical resolution of 3.6 pc and 7.2 pc. We refer the reader to Paper I for a previous comparison of these simulations to our fiducial run. In Figure~\ref{fig:resolution-phase} we demonstrate that O and Ba abundances behave qualitatively similar in our lower resolution runs as in our fiducial model. In both lower resolution runs O has a generally tighter distribution that narrows over time, while Ba is much less well mixed, in agreement with our fiducial simulation. The exact numerical values for these spreads are not convergent across simulations, but we do note significant variations in the exact SFH of these lower resolution runs could be driving these differences.

\begin{figure*}
\centering
\includegraphics[width=0.95\linewidth]{O_Ba_resolution_study.png}
\caption{Resolution comparison at two lower resolution runs giving the log difference between the 90$^{\rm th}$ decile and 10$^{\rm th}$ decile in dex for the metal PDFS of O and Ba across all phases. Compare to Figure~\ref{fig:phase-statistics}.}
\label{fig:resolution-phase}
\end{figure*}









\end{document}
